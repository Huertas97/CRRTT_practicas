
    




    
\documentclass[11pt]{article}

    
    \usepackage[breakable]{tcolorbox}
    \tcbset{nobeforeafter} % prevents tcolorboxes being placing in paragraphs
    \usepackage{float}
    \floatplacement{figure}{H} % forces figures to be placed at the correct location
    
    \usepackage[T1]{fontenc}
    % Nicer default font (+ math font) than Computer Modern for most use cases
    \usepackage{mathpazo}

    % Basic figure setup, for now with no caption control since it's done
    % automatically by Pandoc (which extracts ![](path) syntax from Markdown).
    \usepackage{graphicx}
    % We will generate all images so they have a width \maxwidth. This means
    % that they will get their normal width if they fit onto the page, but
    % are scaled down if they would overflow the margins.
    \makeatletter
    \def\maxwidth{\ifdim\Gin@nat@width>\linewidth\linewidth
    \else\Gin@nat@width\fi}
    \makeatother
    \let\Oldincludegraphics\includegraphics
    % Set max figure width to be 80% of text width, for now hardcoded.
    \renewcommand{\includegraphics}[1]{\Oldincludegraphics[width=.8\maxwidth]{#1}}
    % Ensure that by default, figures have no caption (until we provide a
    % proper Figure object with a Caption API and a way to capture that
    % in the conversion process - todo).
    \usepackage{caption}
    \DeclareCaptionLabelFormat{nolabel}{}
    \captionsetup{labelformat=nolabel}

    \usepackage{adjustbox} % Used to constrain images to a maximum size 
    \usepackage{xcolor} % Allow colors to be defined
    \usepackage{enumerate} % Needed for markdown enumerations to work
    \usepackage{geometry} % Used to adjust the document margins
    \usepackage{amsmath} % Equations
    \usepackage{amssymb} % Equations
    \usepackage{textcomp} % defines textquotesingle
    % Hack from http://tex.stackexchange.com/a/47451/13684:
    \AtBeginDocument{%
        \def\PYZsq{\textquotesingle}% Upright quotes in Pygmentized code
    }
    \usepackage{upquote} % Upright quotes for verbatim code
    \usepackage{eurosym} % defines \euro
    \usepackage[mathletters]{ucs} % Extended unicode (utf-8) support
    \usepackage[utf8x]{inputenc} % Allow utf-8 characters in the tex document
    \usepackage{fancyvrb} % verbatim replacement that allows latex
    \usepackage{grffile} % extends the file name processing of package graphics 
                         % to support a larger range 
    % The hyperref package gives us a pdf with properly built
    % internal navigation ('pdf bookmarks' for the table of contents,
    % internal cross-reference links, web links for URLs, etc.)
    \usepackage{hyperref}
    \usepackage{longtable} % longtable support required by pandoc >1.10
    \usepackage{booktabs}  % table support for pandoc > 1.12.2
    \usepackage[inline]{enumitem} % IRkernel/repr support (it uses the enumerate* environment)
    \usepackage[normalem]{ulem} % ulem is needed to support strikethroughs (\sout)
                                % normalem makes italics be italics, not underlines
    \usepackage{mathrsfs}
    

    
    % Colors for the hyperref package
    \definecolor{urlcolor}{rgb}{0,.145,.698}
    \definecolor{linkcolor}{rgb}{.71,0.21,0.01}
    \definecolor{citecolor}{rgb}{.12,.54,.11}

    % ANSI colors
    \definecolor{ansi-black}{HTML}{3E424D}
    \definecolor{ansi-black-intense}{HTML}{282C36}
    \definecolor{ansi-red}{HTML}{E75C58}
    \definecolor{ansi-red-intense}{HTML}{B22B31}
    \definecolor{ansi-green}{HTML}{00A250}
    \definecolor{ansi-green-intense}{HTML}{007427}
    \definecolor{ansi-yellow}{HTML}{DDB62B}
    \definecolor{ansi-yellow-intense}{HTML}{B27D12}
    \definecolor{ansi-blue}{HTML}{208FFB}
    \definecolor{ansi-blue-intense}{HTML}{0065CA}
    \definecolor{ansi-magenta}{HTML}{D160C4}
    \definecolor{ansi-magenta-intense}{HTML}{A03196}
    \definecolor{ansi-cyan}{HTML}{60C6C8}
    \definecolor{ansi-cyan-intense}{HTML}{258F8F}
    \definecolor{ansi-white}{HTML}{C5C1B4}
    \definecolor{ansi-white-intense}{HTML}{A1A6B2}
    \definecolor{ansi-default-inverse-fg}{HTML}{FFFFFF}
    \definecolor{ansi-default-inverse-bg}{HTML}{000000}

    % commands and environments needed by pandoc snippets
    % extracted from the output of `pandoc -s`
    \providecommand{\tightlist}{%
      \setlength{\itemsep}{0pt}\setlength{\parskip}{0pt}}
    \DefineVerbatimEnvironment{Highlighting}{Verbatim}{commandchars=\\\{\}}
    % Add ',fontsize=\small' for more characters per line
    \newenvironment{Shaded}{}{}
    \newcommand{\KeywordTok}[1]{\textcolor[rgb]{0.00,0.44,0.13}{\textbf{{#1}}}}
    \newcommand{\DataTypeTok}[1]{\textcolor[rgb]{0.56,0.13,0.00}{{#1}}}
    \newcommand{\DecValTok}[1]{\textcolor[rgb]{0.25,0.63,0.44}{{#1}}}
    \newcommand{\BaseNTok}[1]{\textcolor[rgb]{0.25,0.63,0.44}{{#1}}}
    \newcommand{\FloatTok}[1]{\textcolor[rgb]{0.25,0.63,0.44}{{#1}}}
    \newcommand{\CharTok}[1]{\textcolor[rgb]{0.25,0.44,0.63}{{#1}}}
    \newcommand{\StringTok}[1]{\textcolor[rgb]{0.25,0.44,0.63}{{#1}}}
    \newcommand{\CommentTok}[1]{\textcolor[rgb]{0.38,0.63,0.69}{\textit{{#1}}}}
    \newcommand{\OtherTok}[1]{\textcolor[rgb]{0.00,0.44,0.13}{{#1}}}
    \newcommand{\AlertTok}[1]{\textcolor[rgb]{1.00,0.00,0.00}{\textbf{{#1}}}}
    \newcommand{\FunctionTok}[1]{\textcolor[rgb]{0.02,0.16,0.49}{{#1}}}
    \newcommand{\RegionMarkerTok}[1]{{#1}}
    \newcommand{\ErrorTok}[1]{\textcolor[rgb]{1.00,0.00,0.00}{\textbf{{#1}}}}
    \newcommand{\NormalTok}[1]{{#1}}
    
    % Additional commands for more recent versions of Pandoc
    \newcommand{\ConstantTok}[1]{\textcolor[rgb]{0.53,0.00,0.00}{{#1}}}
    \newcommand{\SpecialCharTok}[1]{\textcolor[rgb]{0.25,0.44,0.63}{{#1}}}
    \newcommand{\VerbatimStringTok}[1]{\textcolor[rgb]{0.25,0.44,0.63}{{#1}}}
    \newcommand{\SpecialStringTok}[1]{\textcolor[rgb]{0.73,0.40,0.53}{{#1}}}
    \newcommand{\ImportTok}[1]{{#1}}
    \newcommand{\DocumentationTok}[1]{\textcolor[rgb]{0.73,0.13,0.13}{\textit{{#1}}}}
    \newcommand{\AnnotationTok}[1]{\textcolor[rgb]{0.38,0.63,0.69}{\textbf{\textit{{#1}}}}}
    \newcommand{\CommentVarTok}[1]{\textcolor[rgb]{0.38,0.63,0.69}{\textbf{\textit{{#1}}}}}
    \newcommand{\VariableTok}[1]{\textcolor[rgb]{0.10,0.09,0.49}{{#1}}}
    \newcommand{\ControlFlowTok}[1]{\textcolor[rgb]{0.00,0.44,0.13}{\textbf{{#1}}}}
    \newcommand{\OperatorTok}[1]{\textcolor[rgb]{0.40,0.40,0.40}{{#1}}}
    \newcommand{\BuiltInTok}[1]{{#1}}
    \newcommand{\ExtensionTok}[1]{{#1}}
    \newcommand{\PreprocessorTok}[1]{\textcolor[rgb]{0.74,0.48,0.00}{{#1}}}
    \newcommand{\AttributeTok}[1]{\textcolor[rgb]{0.49,0.56,0.16}{{#1}}}
    \newcommand{\InformationTok}[1]{\textcolor[rgb]{0.38,0.63,0.69}{\textbf{\textit{{#1}}}}}
    \newcommand{\WarningTok}[1]{\textcolor[rgb]{0.38,0.63,0.69}{\textbf{\textit{{#1}}}}}
    
    
    % Define a nice break command that doesn't care if a line doesn't already
    % exist.
    \def\br{\hspace*{\fill} \\* }
    % Math Jax compatibility definitions
    \def\gt{>}
    \def\lt{<}
    \let\Oldtex\TeX
    \let\Oldlatex\LaTeX
    \renewcommand{\TeX}{\textrm{\Oldtex}}
    \renewcommand{\LaTeX}{\textrm{\Oldlatex}}
    % Document parameters
    % Document title
    \title{Practica\_1\_Sara\_Dorado\_Alfaro\_Alvaro\_Huertas\_Garcia}
    
    
    
    
    
% Pygments definitions
\makeatletter
\def\PY@reset{\let\PY@it=\relax \let\PY@bf=\relax%
    \let\PY@ul=\relax \let\PY@tc=\relax%
    \let\PY@bc=\relax \let\PY@ff=\relax}
\def\PY@tok#1{\csname PY@tok@#1\endcsname}
\def\PY@toks#1+{\ifx\relax#1\empty\else%
    \PY@tok{#1}\expandafter\PY@toks\fi}
\def\PY@do#1{\PY@bc{\PY@tc{\PY@ul{%
    \PY@it{\PY@bf{\PY@ff{#1}}}}}}}
\def\PY#1#2{\PY@reset\PY@toks#1+\relax+\PY@do{#2}}

\expandafter\def\csname PY@tok@w\endcsname{\def\PY@tc##1{\textcolor[rgb]{0.73,0.73,0.73}{##1}}}
\expandafter\def\csname PY@tok@c\endcsname{\let\PY@it=\textit\def\PY@tc##1{\textcolor[rgb]{0.25,0.50,0.50}{##1}}}
\expandafter\def\csname PY@tok@cp\endcsname{\def\PY@tc##1{\textcolor[rgb]{0.74,0.48,0.00}{##1}}}
\expandafter\def\csname PY@tok@k\endcsname{\let\PY@bf=\textbf\def\PY@tc##1{\textcolor[rgb]{0.00,0.50,0.00}{##1}}}
\expandafter\def\csname PY@tok@kp\endcsname{\def\PY@tc##1{\textcolor[rgb]{0.00,0.50,0.00}{##1}}}
\expandafter\def\csname PY@tok@kt\endcsname{\def\PY@tc##1{\textcolor[rgb]{0.69,0.00,0.25}{##1}}}
\expandafter\def\csname PY@tok@o\endcsname{\def\PY@tc##1{\textcolor[rgb]{0.40,0.40,0.40}{##1}}}
\expandafter\def\csname PY@tok@ow\endcsname{\let\PY@bf=\textbf\def\PY@tc##1{\textcolor[rgb]{0.67,0.13,1.00}{##1}}}
\expandafter\def\csname PY@tok@nb\endcsname{\def\PY@tc##1{\textcolor[rgb]{0.00,0.50,0.00}{##1}}}
\expandafter\def\csname PY@tok@nf\endcsname{\def\PY@tc##1{\textcolor[rgb]{0.00,0.00,1.00}{##1}}}
\expandafter\def\csname PY@tok@nc\endcsname{\let\PY@bf=\textbf\def\PY@tc##1{\textcolor[rgb]{0.00,0.00,1.00}{##1}}}
\expandafter\def\csname PY@tok@nn\endcsname{\let\PY@bf=\textbf\def\PY@tc##1{\textcolor[rgb]{0.00,0.00,1.00}{##1}}}
\expandafter\def\csname PY@tok@ne\endcsname{\let\PY@bf=\textbf\def\PY@tc##1{\textcolor[rgb]{0.82,0.25,0.23}{##1}}}
\expandafter\def\csname PY@tok@nv\endcsname{\def\PY@tc##1{\textcolor[rgb]{0.10,0.09,0.49}{##1}}}
\expandafter\def\csname PY@tok@no\endcsname{\def\PY@tc##1{\textcolor[rgb]{0.53,0.00,0.00}{##1}}}
\expandafter\def\csname PY@tok@nl\endcsname{\def\PY@tc##1{\textcolor[rgb]{0.63,0.63,0.00}{##1}}}
\expandafter\def\csname PY@tok@ni\endcsname{\let\PY@bf=\textbf\def\PY@tc##1{\textcolor[rgb]{0.60,0.60,0.60}{##1}}}
\expandafter\def\csname PY@tok@na\endcsname{\def\PY@tc##1{\textcolor[rgb]{0.49,0.56,0.16}{##1}}}
\expandafter\def\csname PY@tok@nt\endcsname{\let\PY@bf=\textbf\def\PY@tc##1{\textcolor[rgb]{0.00,0.50,0.00}{##1}}}
\expandafter\def\csname PY@tok@nd\endcsname{\def\PY@tc##1{\textcolor[rgb]{0.67,0.13,1.00}{##1}}}
\expandafter\def\csname PY@tok@s\endcsname{\def\PY@tc##1{\textcolor[rgb]{0.73,0.13,0.13}{##1}}}
\expandafter\def\csname PY@tok@sd\endcsname{\let\PY@it=\textit\def\PY@tc##1{\textcolor[rgb]{0.73,0.13,0.13}{##1}}}
\expandafter\def\csname PY@tok@si\endcsname{\let\PY@bf=\textbf\def\PY@tc##1{\textcolor[rgb]{0.73,0.40,0.53}{##1}}}
\expandafter\def\csname PY@tok@se\endcsname{\let\PY@bf=\textbf\def\PY@tc##1{\textcolor[rgb]{0.73,0.40,0.13}{##1}}}
\expandafter\def\csname PY@tok@sr\endcsname{\def\PY@tc##1{\textcolor[rgb]{0.73,0.40,0.53}{##1}}}
\expandafter\def\csname PY@tok@ss\endcsname{\def\PY@tc##1{\textcolor[rgb]{0.10,0.09,0.49}{##1}}}
\expandafter\def\csname PY@tok@sx\endcsname{\def\PY@tc##1{\textcolor[rgb]{0.00,0.50,0.00}{##1}}}
\expandafter\def\csname PY@tok@m\endcsname{\def\PY@tc##1{\textcolor[rgb]{0.40,0.40,0.40}{##1}}}
\expandafter\def\csname PY@tok@gh\endcsname{\let\PY@bf=\textbf\def\PY@tc##1{\textcolor[rgb]{0.00,0.00,0.50}{##1}}}
\expandafter\def\csname PY@tok@gu\endcsname{\let\PY@bf=\textbf\def\PY@tc##1{\textcolor[rgb]{0.50,0.00,0.50}{##1}}}
\expandafter\def\csname PY@tok@gd\endcsname{\def\PY@tc##1{\textcolor[rgb]{0.63,0.00,0.00}{##1}}}
\expandafter\def\csname PY@tok@gi\endcsname{\def\PY@tc##1{\textcolor[rgb]{0.00,0.63,0.00}{##1}}}
\expandafter\def\csname PY@tok@gr\endcsname{\def\PY@tc##1{\textcolor[rgb]{1.00,0.00,0.00}{##1}}}
\expandafter\def\csname PY@tok@ge\endcsname{\let\PY@it=\textit}
\expandafter\def\csname PY@tok@gs\endcsname{\let\PY@bf=\textbf}
\expandafter\def\csname PY@tok@gp\endcsname{\let\PY@bf=\textbf\def\PY@tc##1{\textcolor[rgb]{0.00,0.00,0.50}{##1}}}
\expandafter\def\csname PY@tok@go\endcsname{\def\PY@tc##1{\textcolor[rgb]{0.53,0.53,0.53}{##1}}}
\expandafter\def\csname PY@tok@gt\endcsname{\def\PY@tc##1{\textcolor[rgb]{0.00,0.27,0.87}{##1}}}
\expandafter\def\csname PY@tok@err\endcsname{\def\PY@bc##1{\setlength{\fboxsep}{0pt}\fcolorbox[rgb]{1.00,0.00,0.00}{1,1,1}{\strut ##1}}}
\expandafter\def\csname PY@tok@kc\endcsname{\let\PY@bf=\textbf\def\PY@tc##1{\textcolor[rgb]{0.00,0.50,0.00}{##1}}}
\expandafter\def\csname PY@tok@kd\endcsname{\let\PY@bf=\textbf\def\PY@tc##1{\textcolor[rgb]{0.00,0.50,0.00}{##1}}}
\expandafter\def\csname PY@tok@kn\endcsname{\let\PY@bf=\textbf\def\PY@tc##1{\textcolor[rgb]{0.00,0.50,0.00}{##1}}}
\expandafter\def\csname PY@tok@kr\endcsname{\let\PY@bf=\textbf\def\PY@tc##1{\textcolor[rgb]{0.00,0.50,0.00}{##1}}}
\expandafter\def\csname PY@tok@bp\endcsname{\def\PY@tc##1{\textcolor[rgb]{0.00,0.50,0.00}{##1}}}
\expandafter\def\csname PY@tok@fm\endcsname{\def\PY@tc##1{\textcolor[rgb]{0.00,0.00,1.00}{##1}}}
\expandafter\def\csname PY@tok@vc\endcsname{\def\PY@tc##1{\textcolor[rgb]{0.10,0.09,0.49}{##1}}}
\expandafter\def\csname PY@tok@vg\endcsname{\def\PY@tc##1{\textcolor[rgb]{0.10,0.09,0.49}{##1}}}
\expandafter\def\csname PY@tok@vi\endcsname{\def\PY@tc##1{\textcolor[rgb]{0.10,0.09,0.49}{##1}}}
\expandafter\def\csname PY@tok@vm\endcsname{\def\PY@tc##1{\textcolor[rgb]{0.10,0.09,0.49}{##1}}}
\expandafter\def\csname PY@tok@sa\endcsname{\def\PY@tc##1{\textcolor[rgb]{0.73,0.13,0.13}{##1}}}
\expandafter\def\csname PY@tok@sb\endcsname{\def\PY@tc##1{\textcolor[rgb]{0.73,0.13,0.13}{##1}}}
\expandafter\def\csname PY@tok@sc\endcsname{\def\PY@tc##1{\textcolor[rgb]{0.73,0.13,0.13}{##1}}}
\expandafter\def\csname PY@tok@dl\endcsname{\def\PY@tc##1{\textcolor[rgb]{0.73,0.13,0.13}{##1}}}
\expandafter\def\csname PY@tok@s2\endcsname{\def\PY@tc##1{\textcolor[rgb]{0.73,0.13,0.13}{##1}}}
\expandafter\def\csname PY@tok@sh\endcsname{\def\PY@tc##1{\textcolor[rgb]{0.73,0.13,0.13}{##1}}}
\expandafter\def\csname PY@tok@s1\endcsname{\def\PY@tc##1{\textcolor[rgb]{0.73,0.13,0.13}{##1}}}
\expandafter\def\csname PY@tok@mb\endcsname{\def\PY@tc##1{\textcolor[rgb]{0.40,0.40,0.40}{##1}}}
\expandafter\def\csname PY@tok@mf\endcsname{\def\PY@tc##1{\textcolor[rgb]{0.40,0.40,0.40}{##1}}}
\expandafter\def\csname PY@tok@mh\endcsname{\def\PY@tc##1{\textcolor[rgb]{0.40,0.40,0.40}{##1}}}
\expandafter\def\csname PY@tok@mi\endcsname{\def\PY@tc##1{\textcolor[rgb]{0.40,0.40,0.40}{##1}}}
\expandafter\def\csname PY@tok@il\endcsname{\def\PY@tc##1{\textcolor[rgb]{0.40,0.40,0.40}{##1}}}
\expandafter\def\csname PY@tok@mo\endcsname{\def\PY@tc##1{\textcolor[rgb]{0.40,0.40,0.40}{##1}}}
\expandafter\def\csname PY@tok@ch\endcsname{\let\PY@it=\textit\def\PY@tc##1{\textcolor[rgb]{0.25,0.50,0.50}{##1}}}
\expandafter\def\csname PY@tok@cm\endcsname{\let\PY@it=\textit\def\PY@tc##1{\textcolor[rgb]{0.25,0.50,0.50}{##1}}}
\expandafter\def\csname PY@tok@cpf\endcsname{\let\PY@it=\textit\def\PY@tc##1{\textcolor[rgb]{0.25,0.50,0.50}{##1}}}
\expandafter\def\csname PY@tok@c1\endcsname{\let\PY@it=\textit\def\PY@tc##1{\textcolor[rgb]{0.25,0.50,0.50}{##1}}}
\expandafter\def\csname PY@tok@cs\endcsname{\let\PY@it=\textit\def\PY@tc##1{\textcolor[rgb]{0.25,0.50,0.50}{##1}}}

\def\PYZbs{\char`\\}
\def\PYZus{\char`\_}
\def\PYZob{\char`\{}
\def\PYZcb{\char`\}}
\def\PYZca{\char`\^}
\def\PYZam{\char`\&}
\def\PYZlt{\char`\<}
\def\PYZgt{\char`\>}
\def\PYZsh{\char`\#}
\def\PYZpc{\char`\%}
\def\PYZdl{\char`\$}
\def\PYZhy{\char`\-}
\def\PYZsq{\char`\'}
\def\PYZdq{\char`\"}
\def\PYZti{\char`\~}
% for compatibility with earlier versions
\def\PYZat{@}
\def\PYZlb{[}
\def\PYZrb{]}
\makeatother


    % For linebreaks inside Verbatim environment from package fancyvrb. 
    \makeatletter
        \newbox\Wrappedcontinuationbox 
        \newbox\Wrappedvisiblespacebox 
        \newcommand*\Wrappedvisiblespace {\textcolor{red}{\textvisiblespace}} 
        \newcommand*\Wrappedcontinuationsymbol {\textcolor{red}{\llap{\tiny$\m@th\hookrightarrow$}}} 
        \newcommand*\Wrappedcontinuationindent {3ex } 
        \newcommand*\Wrappedafterbreak {\kern\Wrappedcontinuationindent\copy\Wrappedcontinuationbox} 
        % Take advantage of the already applied Pygments mark-up to insert 
        % potential linebreaks for TeX processing. 
        %        {, <, #, %, $, ' and ": go to next line. 
        %        _, }, ^, &, >, - and ~: stay at end of broken line. 
        % Use of \textquotesingle for straight quote. 
        \newcommand*\Wrappedbreaksatspecials {% 
            \def\PYGZus{\discretionary{\char`\_}{\Wrappedafterbreak}{\char`\_}}% 
            \def\PYGZob{\discretionary{}{\Wrappedafterbreak\char`\{}{\char`\{}}% 
            \def\PYGZcb{\discretionary{\char`\}}{\Wrappedafterbreak}{\char`\}}}% 
            \def\PYGZca{\discretionary{\char`\^}{\Wrappedafterbreak}{\char`\^}}% 
            \def\PYGZam{\discretionary{\char`\&}{\Wrappedafterbreak}{\char`\&}}% 
            \def\PYGZlt{\discretionary{}{\Wrappedafterbreak\char`\<}{\char`\<}}% 
            \def\PYGZgt{\discretionary{\char`\>}{\Wrappedafterbreak}{\char`\>}}% 
            \def\PYGZsh{\discretionary{}{\Wrappedafterbreak\char`\#}{\char`\#}}% 
            \def\PYGZpc{\discretionary{}{\Wrappedafterbreak\char`\%}{\char`\%}}% 
            \def\PYGZdl{\discretionary{}{\Wrappedafterbreak\char`\$}{\char`\$}}% 
            \def\PYGZhy{\discretionary{\char`\-}{\Wrappedafterbreak}{\char`\-}}% 
            \def\PYGZsq{\discretionary{}{\Wrappedafterbreak\textquotesingle}{\textquotesingle}}% 
            \def\PYGZdq{\discretionary{}{\Wrappedafterbreak\char`\"}{\char`\"}}% 
            \def\PYGZti{\discretionary{\char`\~}{\Wrappedafterbreak}{\char`\~}}% 
        } 
        % Some characters . , ; ? ! / are not pygmentized. 
        % This macro makes them "active" and they will insert potential linebreaks 
        \newcommand*\Wrappedbreaksatpunct {% 
            \lccode`\~`\.\lowercase{\def~}{\discretionary{\hbox{\char`\.}}{\Wrappedafterbreak}{\hbox{\char`\.}}}% 
            \lccode`\~`\,\lowercase{\def~}{\discretionary{\hbox{\char`\,}}{\Wrappedafterbreak}{\hbox{\char`\,}}}% 
            \lccode`\~`\;\lowercase{\def~}{\discretionary{\hbox{\char`\;}}{\Wrappedafterbreak}{\hbox{\char`\;}}}% 
            \lccode`\~`\:\lowercase{\def~}{\discretionary{\hbox{\char`\:}}{\Wrappedafterbreak}{\hbox{\char`\:}}}% 
            \lccode`\~`\?\lowercase{\def~}{\discretionary{\hbox{\char`\?}}{\Wrappedafterbreak}{\hbox{\char`\?}}}% 
            \lccode`\~`\!\lowercase{\def~}{\discretionary{\hbox{\char`\!}}{\Wrappedafterbreak}{\hbox{\char`\!}}}% 
            \lccode`\~`\/\lowercase{\def~}{\discretionary{\hbox{\char`\/}}{\Wrappedafterbreak}{\hbox{\char`\/}}}% 
            \catcode`\.\active
            \catcode`\,\active 
            \catcode`\;\active
            \catcode`\:\active
            \catcode`\?\active
            \catcode`\!\active
            \catcode`\/\active 
            \lccode`\~`\~ 	
        }
    \makeatother

    \let\OriginalVerbatim=\Verbatim
    \makeatletter
    \renewcommand{\Verbatim}[1][1]{%
        %\parskip\z@skip
        \sbox\Wrappedcontinuationbox {\Wrappedcontinuationsymbol}%
        \sbox\Wrappedvisiblespacebox {\FV@SetupFont\Wrappedvisiblespace}%
        \def\FancyVerbFormatLine ##1{\hsize\linewidth
            \vtop{\raggedright\hyphenpenalty\z@\exhyphenpenalty\z@
                \doublehyphendemerits\z@\finalhyphendemerits\z@
                \strut ##1\strut}%
        }%
        % If the linebreak is at a space, the latter will be displayed as visible
        % space at end of first line, and a continuation symbol starts next line.
        % Stretch/shrink are however usually zero for typewriter font.
        \def\FV@Space {%
            \nobreak\hskip\z@ plus\fontdimen3\font minus\fontdimen4\font
            \discretionary{\copy\Wrappedvisiblespacebox}{\Wrappedafterbreak}
            {\kern\fontdimen2\font}%
        }%
        
        % Allow breaks at special characters using \PYG... macros.
        \Wrappedbreaksatspecials
        % Breaks at punctuation characters . , ; ? ! and / need catcode=\active 	
        \OriginalVerbatim[#1,codes*=\Wrappedbreaksatpunct]%
    }
    \makeatother

    % Exact colors from NB
    \definecolor{incolor}{HTML}{303F9F}
    \definecolor{outcolor}{HTML}{D84315}
    \definecolor{cellborder}{HTML}{CFCFCF}
    \definecolor{cellbackground}{HTML}{F7F7F7}
    
    % prompt
    \newcommand{\prompt}[4]{
        \llap{{\color{#2}[#3]: #4}}\vspace{-1.25em}
    }
    

    
    % Prevent overflowing lines due to hard-to-break entities
    \sloppy 
    % Setup hyperref package
    \hypersetup{
      breaklinks=true,  % so long urls are correctly broken across lines
      colorlinks=true,
      urlcolor=urlcolor,
      linkcolor=linkcolor,
      citecolor=citecolor,
      }
    % Slightly bigger margins than the latex defaults
    
    \geometry{verbose,tmargin=1in,bmargin=1in,lmargin=1in,rmargin=1in}
    
    

    \begin{document}
    
    
    \maketitle
    
    

    
    \hypertarget{pruxe1ctica-1}{%
\section{PRÁCTICA 1}\label{pruxe1ctica-1}}

\hypertarget{autores}{%
\subsection{Autores:}\label{autores}}

\begin{itemize}
\tightlist
\item
  Sara Dorado Alfaro
\item
  Álvaro Huertas García
\end{itemize}

\hypertarget{fecha-06022020}{%
\subsection{Fecha: 06/02/2020}\label{fecha-06022020}}

    \hypertarget{apartado-1}{%
\subsection{Apartado 1}\label{apartado-1}}

    \begin{tcolorbox}[breakable, size=fbox, boxrule=1pt, pad at break*=1mm,colback=cellbackground, colframe=cellborder]
\prompt{In}{incolor}{1}{\hspace{4pt}}
\begin{Verbatim}[commandchars=\\\{\}]
\PY{k+kn}{import} \PY{n+nn}{networkx} \PY{k}{as} \PY{n+nn}{nx}
\PY{k+kn}{import} \PY{n+nn}{matplotlib}\PY{n+nn}{.}\PY{n+nn}{pyplot} \PY{k}{as} \PY{n+nn}{plt}
\PY{k+kn}{import} \PY{n+nn}{numpy} \PY{k}{as} \PY{n+nn}{np}
\end{Verbatim}
\end{tcolorbox}

    \begin{tcolorbox}[breakable, size=fbox, boxrule=1pt, pad at break*=1mm,colback=cellbackground, colframe=cellborder]
\prompt{In}{incolor}{41}{\hspace{4pt}}
\begin{Verbatim}[commandchars=\\\{\}]
\PY{c+c1}{\PYZsh{} Mostramos el grafo dirigido de la práctica}
\PY{n}{G} \PY{o}{=} \PY{n}{nx}\PY{o}{.}\PY{n}{DiGraph}\PY{p}{(}\PY{p}{)}
\PY{n}{G}\PY{o}{.}\PY{n}{add\PYZus{}nodes\PYZus{}from}\PY{p}{(}\PY{p}{[}\PY{l+s+s1}{\PYZsq{}}\PY{l+s+s1}{1}\PY{l+s+s1}{\PYZsq{}}\PY{p}{,} \PY{l+s+s1}{\PYZsq{}}\PY{l+s+s1}{2}\PY{l+s+s1}{\PYZsq{}}\PY{p}{,} \PY{l+s+s1}{\PYZsq{}}\PY{l+s+s1}{5}\PY{l+s+s1}{\PYZsq{}}\PY{p}{,} \PY{l+s+s1}{\PYZsq{}}\PY{l+s+s1}{10}\PY{l+s+s1}{\PYZsq{}}\PY{p}{,} \PY{l+s+s1}{\PYZsq{}}\PY{l+s+s1}{3}\PY{l+s+s1}{\PYZsq{}}\PY{p}{,} \PY{l+s+s1}{\PYZsq{}}\PY{l+s+s1}{4}\PY{l+s+s1}{\PYZsq{}}\PY{p}{,} \PY{l+s+s1}{\PYZsq{}}\PY{l+s+s1}{6}\PY{l+s+s1}{\PYZsq{}}\PY{p}{,} \PY{l+s+s1}{\PYZsq{}}\PY{l+s+s1}{9}\PY{l+s+s1}{\PYZsq{}}\PY{p}{,} \PY{l+s+s1}{\PYZsq{}}\PY{l+s+s1}{7}\PY{l+s+s1}{\PYZsq{}}\PY{p}{,} \PY{l+s+s1}{\PYZsq{}}\PY{l+s+s1}{8}\PY{l+s+s1}{\PYZsq{}}\PY{p}{]}\PY{p}{)}
\PY{n}{G}\PY{o}{.}\PY{n}{add\PYZus{}edges\PYZus{}from}\PY{p}{(}\PY{p}{[}\PY{p}{(}\PY{l+s+s1}{\PYZsq{}}\PY{l+s+s1}{1}\PY{l+s+s1}{\PYZsq{}}\PY{p}{,} \PY{l+s+s1}{\PYZsq{}}\PY{l+s+s1}{2}\PY{l+s+s1}{\PYZsq{}}\PY{p}{)}\PY{p}{,} \PY{p}{(}\PY{l+s+s1}{\PYZsq{}}\PY{l+s+s1}{1}\PY{l+s+s1}{\PYZsq{}}\PY{p}{,} \PY{l+s+s1}{\PYZsq{}}\PY{l+s+s1}{5}\PY{l+s+s1}{\PYZsq{}}\PY{p}{)}\PY{p}{,} \PY{p}{(}\PY{l+s+s1}{\PYZsq{}}\PY{l+s+s1}{5}\PY{l+s+s1}{\PYZsq{}}\PY{p}{,} \PY{l+s+s1}{\PYZsq{}}\PY{l+s+s1}{1}\PY{l+s+s1}{\PYZsq{}}\PY{p}{)}\PY{p}{,} \PY{p}{(}\PY{l+s+s1}{\PYZsq{}}\PY{l+s+s1}{1}\PY{l+s+s1}{\PYZsq{}}\PY{p}{,} \PY{l+s+s1}{\PYZsq{}}\PY{l+s+s1}{10}\PY{l+s+s1}{\PYZsq{}}\PY{p}{)}\PY{p}{,} \PY{p}{(}\PY{l+s+s1}{\PYZsq{}}\PY{l+s+s1}{4}\PY{l+s+s1}{\PYZsq{}}\PY{p}{,} \PY{l+s+s1}{\PYZsq{}}\PY{l+s+s1}{2}\PY{l+s+s1}{\PYZsq{}}\PY{p}{)}\PY{p}{,} \PY{p}{(}\PY{l+s+s1}{\PYZsq{}}\PY{l+s+s1}{4}\PY{l+s+s1}{\PYZsq{}}\PY{p}{,} \PY{l+s+s1}{\PYZsq{}}\PY{l+s+s1}{5}\PY{l+s+s1}{\PYZsq{}}\PY{p}{)}\PY{p}{,}
                  \PY{p}{(}\PY{l+s+s1}{\PYZsq{}}\PY{l+s+s1}{10}\PY{l+s+s1}{\PYZsq{}}\PY{p}{,} \PY{l+s+s1}{\PYZsq{}}\PY{l+s+s1}{6}\PY{l+s+s1}{\PYZsq{}}\PY{p}{)}\PY{p}{,} \PY{p}{(}\PY{l+s+s1}{\PYZsq{}}\PY{l+s+s1}{10}\PY{l+s+s1}{\PYZsq{}}\PY{p}{,} \PY{l+s+s1}{\PYZsq{}}\PY{l+s+s1}{1}\PY{l+s+s1}{\PYZsq{}}\PY{p}{)}\PY{p}{,} \PY{p}{(}\PY{l+s+s1}{\PYZsq{}}\PY{l+s+s1}{6}\PY{l+s+s1}{\PYZsq{}}\PY{p}{,} \PY{l+s+s1}{\PYZsq{}}\PY{l+s+s1}{10}\PY{l+s+s1}{\PYZsq{}}\PY{p}{)}\PY{p}{,} \PY{p}{(}\PY{l+s+s1}{\PYZsq{}}\PY{l+s+s1}{3}\PY{l+s+s1}{\PYZsq{}}\PY{p}{,} \PY{l+s+s1}{\PYZsq{}}\PY{l+s+s1}{4}\PY{l+s+s1}{\PYZsq{}}\PY{p}{)}\PY{p}{,}
                  \PY{p}{(}\PY{l+s+s1}{\PYZsq{}}\PY{l+s+s1}{6}\PY{l+s+s1}{\PYZsq{}}\PY{p}{,} \PY{l+s+s1}{\PYZsq{}}\PY{l+s+s1}{9}\PY{l+s+s1}{\PYZsq{}}\PY{p}{)}\PY{p}{,} \PY{p}{(}\PY{l+s+s1}{\PYZsq{}}\PY{l+s+s1}{7}\PY{l+s+s1}{\PYZsq{}}\PY{p}{,} \PY{l+s+s1}{\PYZsq{}}\PY{l+s+s1}{6}\PY{l+s+s1}{\PYZsq{}}\PY{p}{)}\PY{p}{,} \PY{p}{(}\PY{l+s+s1}{\PYZsq{}}\PY{l+s+s1}{7}\PY{l+s+s1}{\PYZsq{}}\PY{p}{,} \PY{l+s+s1}{\PYZsq{}}\PY{l+s+s1}{9}\PY{l+s+s1}{\PYZsq{}}\PY{p}{)}\PY{p}{,} \PY{p}{(}\PY{l+s+s1}{\PYZsq{}}\PY{l+s+s1}{9}\PY{l+s+s1}{\PYZsq{}}\PY{p}{,} \PY{l+s+s1}{\PYZsq{}}\PY{l+s+s1}{8}\PY{l+s+s1}{\PYZsq{}}\PY{p}{)}\PY{p}{,} \PY{p}{(}\PY{l+s+s1}{\PYZsq{}}\PY{l+s+s1}{9}\PY{l+s+s1}{\PYZsq{}}\PY{p}{,} \PY{l+s+s1}{\PYZsq{}}\PY{l+s+s1}{6}\PY{l+s+s1}{\PYZsq{}}\PY{p}{)}\PY{p}{,} \PY{p}{(}\PY{l+s+s1}{\PYZsq{}}\PY{l+s+s1}{7}\PY{l+s+s1}{\PYZsq{}}\PY{p}{,} \PY{l+s+s1}{\PYZsq{}}\PY{l+s+s1}{8}\PY{l+s+s1}{\PYZsq{}}\PY{p}{)}\PY{p}{]}\PY{p}{)}

\PY{n}{posiciones} \PY{o}{=} \PY{p}{\PYZob{}}\PY{l+s+s1}{\PYZsq{}}\PY{l+s+s1}{1}\PY{l+s+s1}{\PYZsq{}}\PY{p}{:} \PY{p}{[}\PY{l+m+mi}{7}\PY{p}{,} \PY{l+m+mi}{0}\PY{p}{]}\PY{p}{,} \PY{l+s+s1}{\PYZsq{}}\PY{l+s+s1}{2}\PY{l+s+s1}{\PYZsq{}}\PY{p}{:}\PY{p}{[}\PY{l+m+mi}{5}\PY{p}{,} \PY{l+m+mi}{1}\PY{p}{]}\PY{p}{,} 
              \PY{l+s+s1}{\PYZsq{}}\PY{l+s+s1}{5}\PY{l+s+s1}{\PYZsq{}}\PY{p}{:} \PY{p}{[}\PY{l+m+mi}{5}\PY{p}{,} \PY{o}{\PYZhy{}}\PY{l+m+mi}{1}\PY{p}{]}\PY{p}{,} \PY{l+s+s1}{\PYZsq{}}\PY{l+s+s1}{10}\PY{l+s+s1}{\PYZsq{}}\PY{p}{:} \PY{p}{[}\PY{l+m+mi}{9}\PY{p}{,} \PY{l+m+mi}{0}\PY{p}{]}\PY{p}{,} 
              \PY{l+s+s1}{\PYZsq{}}\PY{l+s+s1}{3}\PY{l+s+s1}{\PYZsq{}}\PY{p}{:} \PY{p}{[}\PY{l+m+mi}{1}\PY{p}{,} \PY{l+m+mi}{0}\PY{p}{]}\PY{p}{,} \PY{l+s+s1}{\PYZsq{}}\PY{l+s+s1}{4}\PY{l+s+s1}{\PYZsq{}}\PY{p}{:} \PY{p}{[}\PY{l+m+mi}{3}\PY{p}{,} \PY{l+m+mi}{0}\PY{p}{]}\PY{p}{,} 
              \PY{l+s+s1}{\PYZsq{}}\PY{l+s+s1}{6}\PY{l+s+s1}{\PYZsq{}}\PY{p}{:} \PY{p}{[}\PY{l+m+mi}{11}\PY{p}{,} \PY{l+m+mi}{0}\PY{p}{]}\PY{p}{,} \PY{l+s+s1}{\PYZsq{}}\PY{l+s+s1}{9}\PY{l+s+s1}{\PYZsq{}}\PY{p}{:} \PY{p}{[}\PY{l+m+mi}{13}\PY{p}{,}  \PY{l+m+mi}{1}\PY{p}{]}\PY{p}{,} 
              \PY{l+s+s1}{\PYZsq{}}\PY{l+s+s1}{7}\PY{l+s+s1}{\PYZsq{}}\PY{p}{:} \PY{p}{[}\PY{l+m+mi}{13}\PY{p}{,} \PY{o}{\PYZhy{}}\PY{l+m+mi}{1}\PY{p}{]}\PY{p}{,} \PY{l+s+s1}{\PYZsq{}}\PY{l+s+s1}{8}\PY{l+s+s1}{\PYZsq{}}\PY{p}{:} \PY{p}{[}\PY{l+m+mi}{15}\PY{p}{,} \PY{l+m+mi}{0}\PY{p}{]}\PY{p}{\PYZcb{}}

\PY{n+nb}{print}\PY{p}{(}\PY{l+s+s2}{\PYZdq{}}\PY{l+s+s2}{Grafo dirigido}\PY{l+s+s2}{\PYZdq{}}\PY{p}{)}
\PY{n}{nx}\PY{o}{.}\PY{n}{draw\PYZus{}networkx}\PY{p}{(}\PY{n}{G}\PY{p}{,} \PY{n}{pos} \PY{o}{=} \PY{n}{posiciones}\PY{p}{,} \PY{n}{arrowsize} \PY{o}{=} \PY{l+m+mi}{15}\PY{p}{,} \PY{n}{edge\PYZus{}color} \PY{o}{=} \PY{l+s+s1}{\PYZsq{}}\PY{l+s+s1}{black}\PY{l+s+s1}{\PYZsq{}}\PY{p}{)}
\end{Verbatim}
\end{tcolorbox}

    \begin{Verbatim}[commandchars=\\\{\}]
Grafo dirigido
\end{Verbatim}

    \begin{center}
    \adjustimage{max size={0.9\linewidth}{0.9\paperheight}}{output_3_1.png}
    \end{center}
    { \hspace*{\fill} \\}
    
    \begin{tcolorbox}[breakable, size=fbox, boxrule=1pt, pad at break*=1mm,colback=cellbackground, colframe=cellborder]
\prompt{In}{incolor}{3}{\hspace{4pt}}
\begin{Verbatim}[commandchars=\\\{\}]
\PY{c+c1}{\PYZsh{} Mostramos el grafo no dirigido}

\PY{n}{G} \PY{o}{=} \PY{n}{nx}\PY{o}{.}\PY{n}{Graph}\PY{p}{(}\PY{p}{)}
\PY{n}{G}\PY{o}{.}\PY{n}{add\PYZus{}nodes\PYZus{}from}\PY{p}{(}\PY{p}{[}\PY{l+s+s1}{\PYZsq{}}\PY{l+s+s1}{1}\PY{l+s+s1}{\PYZsq{}}\PY{p}{,} \PY{l+s+s1}{\PYZsq{}}\PY{l+s+s1}{2}\PY{l+s+s1}{\PYZsq{}}\PY{p}{,} \PY{l+s+s1}{\PYZsq{}}\PY{l+s+s1}{5}\PY{l+s+s1}{\PYZsq{}}\PY{p}{,} \PY{l+s+s1}{\PYZsq{}}\PY{l+s+s1}{10}\PY{l+s+s1}{\PYZsq{}}\PY{p}{,} \PY{l+s+s1}{\PYZsq{}}\PY{l+s+s1}{3}\PY{l+s+s1}{\PYZsq{}}\PY{p}{,} \PY{l+s+s1}{\PYZsq{}}\PY{l+s+s1}{4}\PY{l+s+s1}{\PYZsq{}}\PY{p}{,} \PY{l+s+s1}{\PYZsq{}}\PY{l+s+s1}{6}\PY{l+s+s1}{\PYZsq{}}\PY{p}{,} \PY{l+s+s1}{\PYZsq{}}\PY{l+s+s1}{9}\PY{l+s+s1}{\PYZsq{}}\PY{p}{,} \PY{l+s+s1}{\PYZsq{}}\PY{l+s+s1}{7}\PY{l+s+s1}{\PYZsq{}}\PY{p}{,} \PY{l+s+s1}{\PYZsq{}}\PY{l+s+s1}{8}\PY{l+s+s1}{\PYZsq{}}\PY{p}{]}\PY{p}{)}
\PY{n}{G}\PY{o}{.}\PY{n}{add\PYZus{}edges\PYZus{}from}\PY{p}{(}\PY{p}{[}\PY{p}{(}\PY{l+s+s1}{\PYZsq{}}\PY{l+s+s1}{1}\PY{l+s+s1}{\PYZsq{}}\PY{p}{,} \PY{l+s+s1}{\PYZsq{}}\PY{l+s+s1}{2}\PY{l+s+s1}{\PYZsq{}}\PY{p}{)}\PY{p}{,} \PY{p}{(}\PY{l+s+s1}{\PYZsq{}}\PY{l+s+s1}{1}\PY{l+s+s1}{\PYZsq{}}\PY{p}{,} \PY{l+s+s1}{\PYZsq{}}\PY{l+s+s1}{5}\PY{l+s+s1}{\PYZsq{}}\PY{p}{)}\PY{p}{,} \PY{p}{(}\PY{l+s+s1}{\PYZsq{}}\PY{l+s+s1}{1}\PY{l+s+s1}{\PYZsq{}}\PY{p}{,} \PY{l+s+s1}{\PYZsq{}}\PY{l+s+s1}{10}\PY{l+s+s1}{\PYZsq{}}\PY{p}{)}\PY{p}{,} \PY{p}{(}\PY{l+s+s1}{\PYZsq{}}\PY{l+s+s1}{2}\PY{l+s+s1}{\PYZsq{}}\PY{p}{,} \PY{l+s+s1}{\PYZsq{}}\PY{l+s+s1}{4}\PY{l+s+s1}{\PYZsq{}}\PY{p}{)}\PY{p}{,} \PY{p}{(}\PY{l+s+s1}{\PYZsq{}}\PY{l+s+s1}{5}\PY{l+s+s1}{\PYZsq{}}\PY{p}{,} \PY{l+s+s1}{\PYZsq{}}\PY{l+s+s1}{4}\PY{l+s+s1}{\PYZsq{}}\PY{p}{)}\PY{p}{,} \PY{p}{(}\PY{l+s+s1}{\PYZsq{}}\PY{l+s+s1}{10}\PY{l+s+s1}{\PYZsq{}}\PY{p}{,} \PY{l+s+s1}{\PYZsq{}}\PY{l+s+s1}{6}\PY{l+s+s1}{\PYZsq{}}\PY{p}{)}\PY{p}{,} \PY{p}{(}\PY{l+s+s1}{\PYZsq{}}\PY{l+s+s1}{3}\PY{l+s+s1}{\PYZsq{}}\PY{p}{,} \PY{l+s+s1}{\PYZsq{}}\PY{l+s+s1}{4}\PY{l+s+s1}{\PYZsq{}}\PY{p}{)}\PY{p}{,}
                  \PY{p}{(}\PY{l+s+s1}{\PYZsq{}}\PY{l+s+s1}{6}\PY{l+s+s1}{\PYZsq{}}\PY{p}{,} \PY{l+s+s1}{\PYZsq{}}\PY{l+s+s1}{9}\PY{l+s+s1}{\PYZsq{}}\PY{p}{)}\PY{p}{,} \PY{p}{(}\PY{l+s+s1}{\PYZsq{}}\PY{l+s+s1}{6}\PY{l+s+s1}{\PYZsq{}}\PY{p}{,} \PY{l+s+s1}{\PYZsq{}}\PY{l+s+s1}{7}\PY{l+s+s1}{\PYZsq{}}\PY{p}{)}\PY{p}{,} \PY{p}{(}\PY{l+s+s1}{\PYZsq{}}\PY{l+s+s1}{9}\PY{l+s+s1}{\PYZsq{}}\PY{p}{,} \PY{l+s+s1}{\PYZsq{}}\PY{l+s+s1}{7}\PY{l+s+s1}{\PYZsq{}}\PY{p}{)}\PY{p}{,} \PY{p}{(}\PY{l+s+s1}{\PYZsq{}}\PY{l+s+s1}{9}\PY{l+s+s1}{\PYZsq{}}\PY{p}{,} \PY{l+s+s1}{\PYZsq{}}\PY{l+s+s1}{8}\PY{l+s+s1}{\PYZsq{}}\PY{p}{)}\PY{p}{,} \PY{p}{(}\PY{l+s+s1}{\PYZsq{}}\PY{l+s+s1}{7}\PY{l+s+s1}{\PYZsq{}}\PY{p}{,} \PY{l+s+s1}{\PYZsq{}}\PY{l+s+s1}{8}\PY{l+s+s1}{\PYZsq{}}\PY{p}{)}\PY{p}{]}\PY{p}{)}

\PY{n}{posiciones} \PY{o}{=} \PY{p}{\PYZob{}}\PY{l+s+s1}{\PYZsq{}}\PY{l+s+s1}{1}\PY{l+s+s1}{\PYZsq{}}\PY{p}{:} \PY{p}{[}\PY{l+m+mi}{7}\PY{p}{,} \PY{l+m+mi}{0}\PY{p}{]}\PY{p}{,} \PY{l+s+s1}{\PYZsq{}}\PY{l+s+s1}{2}\PY{l+s+s1}{\PYZsq{}}\PY{p}{:}\PY{p}{[}\PY{l+m+mi}{5}\PY{p}{,} \PY{l+m+mi}{1}\PY{p}{]}\PY{p}{,} 
              \PY{l+s+s1}{\PYZsq{}}\PY{l+s+s1}{5}\PY{l+s+s1}{\PYZsq{}}\PY{p}{:} \PY{p}{[}\PY{l+m+mi}{5}\PY{p}{,} \PY{o}{\PYZhy{}}\PY{l+m+mi}{1}\PY{p}{]}\PY{p}{,} \PY{l+s+s1}{\PYZsq{}}\PY{l+s+s1}{10}\PY{l+s+s1}{\PYZsq{}}\PY{p}{:} \PY{p}{[}\PY{l+m+mi}{9}\PY{p}{,} \PY{l+m+mi}{0}\PY{p}{]}\PY{p}{,} 
              \PY{l+s+s1}{\PYZsq{}}\PY{l+s+s1}{3}\PY{l+s+s1}{\PYZsq{}}\PY{p}{:} \PY{p}{[}\PY{l+m+mi}{1}\PY{p}{,} \PY{l+m+mi}{0}\PY{p}{]}\PY{p}{,} \PY{l+s+s1}{\PYZsq{}}\PY{l+s+s1}{4}\PY{l+s+s1}{\PYZsq{}}\PY{p}{:} \PY{p}{[}\PY{l+m+mi}{3}\PY{p}{,} \PY{l+m+mi}{0}\PY{p}{]}\PY{p}{,} 
              \PY{l+s+s1}{\PYZsq{}}\PY{l+s+s1}{6}\PY{l+s+s1}{\PYZsq{}}\PY{p}{:} \PY{p}{[}\PY{l+m+mi}{11}\PY{p}{,} \PY{l+m+mi}{0}\PY{p}{]}\PY{p}{,} \PY{l+s+s1}{\PYZsq{}}\PY{l+s+s1}{9}\PY{l+s+s1}{\PYZsq{}}\PY{p}{:} \PY{p}{[}\PY{l+m+mi}{13}\PY{p}{,}  \PY{l+m+mi}{1}\PY{p}{]}\PY{p}{,} 
              \PY{l+s+s1}{\PYZsq{}}\PY{l+s+s1}{7}\PY{l+s+s1}{\PYZsq{}}\PY{p}{:} \PY{p}{[}\PY{l+m+mi}{13}\PY{p}{,} \PY{o}{\PYZhy{}}\PY{l+m+mi}{1}\PY{p}{]}\PY{p}{,} \PY{l+s+s1}{\PYZsq{}}\PY{l+s+s1}{8}\PY{l+s+s1}{\PYZsq{}}\PY{p}{:} \PY{p}{[}\PY{l+m+mi}{15}\PY{p}{,} \PY{l+m+mi}{0}\PY{p}{]}\PY{p}{\PYZcb{}}

\PY{n+nb}{print}\PY{p}{(}\PY{l+s+s2}{\PYZdq{}}\PY{l+s+s2}{Grafo no dirigido}\PY{l+s+s2}{\PYZdq{}}\PY{p}{)}
\PY{n}{nx}\PY{o}{.}\PY{n}{draw\PYZus{}networkx}\PY{p}{(}\PY{n}{G}\PY{p}{,} \PY{n}{pos} \PY{o}{=} \PY{n}{posiciones}\PY{p}{,} \PY{n}{with\PYZus{}label} \PY{o}{=} \PY{k+kc}{True}\PY{p}{)}
\end{Verbatim}
\end{tcolorbox}

    \begin{Verbatim}[commandchars=\\\{\}]
Grafo no dirigido
\end{Verbatim}

    \begin{center}
    \adjustimage{max size={0.9\linewidth}{0.9\paperheight}}{output_4_1.png}
    \end{center}
    { \hspace*{\fill} \\}
    
    \hypertarget{representad-el-siguiente-grafo-dirigido-mediante-a-una-matriz-de-adyacencia-y-b-una-lista-de-adyacencia.}{%
\subsubsection{1. Representad el siguiente grafo dirigido mediante (a)
una matriz de adyacencia y (b) una lista de
adyacencia.}\label{representad-el-siguiente-grafo-dirigido-mediante-a-una-matriz-de-adyacencia-y-b-una-lista-de-adyacencia.}}

\(\\\)

\textbf{a) Matriz de adyacencia (grafo dirigido):}

\begin{longtable}[]{@{}ccccccccccc@{}}
\toprule
& Nodo 1 & Nodo 2 & Nodo 3 & Nodo 4 & Nodo 5 & Nodo 6 & Nodo 7 & Nodo 8
& Nodo 9 & Nodo 10\tabularnewline
\midrule
\endhead
Nodo 1 & 0 & 1 & 0 & 0 & 1 & 0 & 0 & 0 & 0 & 1\tabularnewline
Nodo 2 & 0 & 0 & 0 & 0 & 0 & 0 & 0 & 0 & 0 & 0\tabularnewline
Nodo 3 & 0 & 0 & 0 & 1 & 0 & 0 & 0 & 0 & 0 & 0\tabularnewline
Nodo 4 & 0 & 1 & 0 & 0 & 1 & 0 & 0 & 0 & 0 & 0\tabularnewline
Nodo 5 & 1 & 0 & 0 & 0 & 0 & 0 & 0 & 0 & 0 & 0\tabularnewline
Nodo 6 & 0 & 0 & 0 & 0 & 0 & 0 & 0 & 0 & 1 & 1\tabularnewline
Nodo 7 & 0 & 0 & 0 & 0 & 0 & 1 & 0 & 1 & 1 & 0\tabularnewline
Nodo 8 & 0 & 0 & 0 & 0 & 0 & 0 & 0 & 0 & 0 & 0\tabularnewline
Nodo 9 & 0 & 0 & 0 & 0 & 0 & 0 & 0 & 1 & 0 & 0\tabularnewline
Nodo 10 & 1 & 0 & 0 & 0 & 0 & 1 & 0 & 0 & 0 & 0\tabularnewline
\bottomrule
\end{longtable}

\(\\\)

\textbf{b) Lista de adyacencia:}

\(1 \rightarrow 2 \rightarrow 5 \rightarrow 10\)

\(2\)

\(3 \rightarrow 4\)

\(4 \rightarrow 2 \rightarrow 5\)

\(5 \rightarrow 1\)

\(6 \rightarrow 9 \rightarrow 10\)

\(7 \rightarrow 6 \rightarrow 8 \rightarrow 9\)

\(8\)

\(9 \rightarrow 8\)

\(10 \rightarrow 1 \rightarrow 6\)

\(\\\)

\hypertarget{responded-a-las-siguientes-preguntas}{%
\subsubsection{2. Responded a las siguientes
preguntas:}\label{responded-a-las-siguientes-preguntas}}

Notación: \(v_i\) es el vértice \(i\)-ésimo del grafo.

\(\\\)

\textbf{a) ¿Es ponderado?}

No, porque las aristas no tienen peso.

\(\\\)

\textbf{b) ¿Es conexo?}

No, porque no hay camino entre cualquier par de vértices. Por ejemplo no
hay camino entre \(v_7\) y \(v_9\).

\(\\\)

\textbf{c) ¿Es débilmente conexo?}

Sí, porque al transformar el grado dirigido en uno no dirigido
observamos que es conexo, es decir, existe un camino entre cualquier par
de vértices en el grafo no dirigido.

\(\\\)

\textbf{d) ¿Cuál es su tamaño y su orden?}

Tamaño \(|E| = 16\), orden \(|V| = 10\).

\(\\\)

\textbf{e) ¿Tiene algún punto de articulación? En caso positivo, indica
cual}

Para el grafo dirigido todos los nodos lo son, porque el grafo no es
conexo. Para el grafo no dirigido, que sí es conexo, los puntos de
articulación son: \(v_1, v_4, v_6, v_{10}\)

\(\\\)

\textbf{f) ¿Tiene lazos?}

No, ningún vértice se conecta consigo mismo.

\(\\\)

\textbf{g) ¿El grafo tiene algún ciclo?}

Teniendo en cuenta que un ciclo es todo paseo cerrado (el nodo de inicio
y fin es el mismo) en el que no se repiten ramas ni vértices, podemos
decir que sí existen ciclos en el grafo dirigido:

Sí, hay varios: \(v_5 \rightarrow v_1 \rightarrow v_5\)\ldots{}

Ciclo 1: \(\ C = \{v_5,v_1,v_5 \} \quad k=2\)

Ciclo 2: \(\ C = \{v_1, v_{10}, v_1\} \quad k=2\)

Ciclo 3: \(\ C = \{v_{10}, v_6, v_{10}\} \quad k=2\)

Ciclo 4: \(\ C = \{v_6, v_9, v_6\} \quad k=2\)

donde \(k\) indica la longitud del camino.

\(\\\)

\textbf{h) ¿Existe algún camino entre los nodos 4 y 7? En caso positivo,
indica cual es y su longitud}

No, porque \(v_7\) es una fuente, así que no podemos alcanzarlo desde
otro nodo.

\(\\\)

\textbf{i) ¿Existe algún camino entre los nodos 3 y 9? En caso positivo,
indica cual es y su longitud}

Sí:

\(C = \{v_3, v_4, v_5, v_1, v_{10}, v_6, v_9 \} \quad k = 6\)

\(\\\)

\textbf{Considera ahora el grafo como un grafo no dirigido:}

\textbf{a) ¿El grafo tiene algún ciclo? En caso positivo, indica cual}

Teniendo en cuenta que un ciclo es todo paseo cerrado (el nodo de inicio
y fin es el mismo) en el que no se repiten ramas ni vértices, podemos
decir que sí existen ciclos en el grafo no dirigido:

Ciclo 1: \(\ C=\{v_4,v_2,v_1,v_5,v_4\}\)

Ciclo 2: \(\ C=\{v_6,v_9,v_8,v_7,v_6\}\)

Ciclo 3: \(\ C=\{v_6,v_9,v_7,v_6\}\)

Ciclo 4: \(\ C=\{v_8,v_7,v_9,v_8\}\)

\(\\\)

\textbf{b) ¿Cuál es el mayor valor de k para el cual existe un k-core?}

Un k-core de un grafo \(G\) es un subgrafo \(G'\) tal que el grado de
cada nodo \(G'\) es al menos \(k\). Por lo tanto,en el grafo hay dos
\(2\)-core:

\begin{itemize}
\tightlist
\item
  \(G'_{1}=(V'_{1},E'_{1})\) donde \(V'_{1}=\{v_6, v_7, v_9\}\) y
  \(E'_{1}= \{(v_6, v_7), (v_7, v_9), (v_9, v_6\}\)
\item
  \(G'_{2}=(V'_{2},E'_{2})\) donde \(V'_{2}=\{v_7, v_8, v_9\}\) y
  \(E'_{2}= \{(v_7, v_8), (v_7, v_9), (v_8, v_9\}\)
\end{itemize}

\(\\\)

\textbf{c) ¿Cuál es el índice de clusterización de \(v_{10}\)?}

El cálculo del índice de clusterización se realiza mediante la siguiente
ecuación:

\[Cv = \frac{Nv}{\frac{kv(kv-1)}{2}}\]

donde \(Nv\) es el número de ramas que hay entre los vecinos del nodo
\(v\); \(kv\) es el número de vecinos del nodo \(v\)

\(v_{10}\) tiene 2 vecinos (\(v_{1}, v_{6}\)), y éstos no se conectan.
Por lo tanto, el índice de clusterización de \(v_10\) es \(0\). Que el
índice de clusterización de \(v_{10}\) sea \(0\) indica que es
importante y un punto débil del grafo, puesto que sin él, sus nodos
vecinos quedan desconectados.

\(\\\)

\textbf{d) Calcula el camino característico del nodo 10}

El cálculo del camino característico se realiza mediante la siguiente
ecuación:

\[ L_{v} =  \frac{\sum_{k=1}^{|V|}d(v, v_{k})}{|V|-1} \]

donde \(d(v, v_{k})\) es la distancia de \(v\) a un nodo \(v_{k}\) del
grafo; \(|V|\) es el orden del grafo.

Camino característico de \(v_{10}\).

\[\frac{1+2+4+3+2+1+2+3+2}{10-1} = \frac{20}{9}\]

\(\\\)

\textbf{e) ¿Existe algún cliqué de orden mayor de 2? En caso positivo,
indica los nodos que lo componen}

Un cliqué es un conjunto de vértices en el que todo par de vértices
distintos son adyacentes, es decir, existe una arista que los conecta.
En otras palabras, un clique es un subgrafo en el que cada vértice está
conectado a todos los demás vértices del subgrafo (a.k.a. completo).

En el grafo no dirigido existen dos cliqués (precisamente los dos
2-cores). Los subgrafos:

\begin{itemize}
\tightlist
\item
  \(G'_{1}=(V'_{1},E'_{1})\) donde \(V'_{1}=\{v_6, v_7, v_9\}\) y
  \(E'_{1}= \{(v_6, v_7), (v_7, v_9), (v_9, v_6\}\)
\item
  \(G'_{2}=(V'_{2},E'_{2})\) donde \(V'_{2}=\{v_7, v_8, v_9\}\) y
  \(E'_{2}= \{(v_7, v_8), (v_7, v_9), (v_8, v_9\}\)
\end{itemize}

No hay cliqués de orden mayor que \(3\).

    \(\\\)

\hypertarget{apartado-2-anuxe1lisis-de-una-red-de-interacciuxf3n-de-proteuxednas-mediantenetworkx.}{%
\subsection{Apartado 2: Análisis de una red de interacción de proteínas
medianteNetworkX.}\label{apartado-2-anuxe1lisis-de-una-red-de-interacciuxf3n-de-proteuxednas-mediantenetworkx.}}

    \textbf{1. Descargad de Moodle el grafo \emph{CaernoElegans-LC\_uw.txt},
el grafo contiene una red de interacción de proteínas correspondiente al
gusano \emph{Caernobidis Elegans}}

    \(\\\)

\textbf{2. El fichero que contiene la red está en formato lista de
ramas, por tanto, cargad el grafo en una variable G\_CE mediante la
función read\_edgelist(``CL-LC\_uw.txt'').}

    \begin{tcolorbox}[breakable, size=fbox, boxrule=1pt, pad at break*=1mm,colback=cellbackground, colframe=cellborder]
\prompt{In}{incolor}{4}{\hspace{4pt}}
\begin{Verbatim}[commandchars=\\\{\}]
\PY{n}{G\PYZus{}CE} \PY{o}{=} \PY{n}{nx}\PY{o}{.}\PY{n}{read\PYZus{}edgelist}\PY{p}{(}\PY{l+s+s2}{\PYZdq{}}\PY{l+s+s2}{CaernoElegans\PYZhy{}LC\PYZus{}uw.txt}\PY{l+s+s2}{\PYZdq{}}\PY{p}{)}
\PY{n}{G\PYZus{}CE}\PY{o}{.}\PY{n}{name} \PY{o}{=} \PY{l+s+s1}{\PYZsq{}}\PY{l+s+s1}{CaernoElegans}\PY{l+s+s1}{\PYZsq{}} \PY{c+c1}{\PYZsh{} añadimos atributo nombre}
\end{Verbatim}
\end{tcolorbox}

    \begin{tcolorbox}[breakable, size=fbox, boxrule=1pt, pad at break*=1mm,colback=cellbackground, colframe=cellborder]
\prompt{In}{incolor}{5}{\hspace{4pt}}
\begin{Verbatim}[commandchars=\\\{\}]
\PY{c+c1}{\PYZsh{}\PYZsh{}\PYZsh{} Exportar a gexf para visualizar}
\PY{n}{nx}\PY{o}{.}\PY{n}{write\PYZus{}gexf}\PY{p}{(}\PY{n}{G\PYZus{}CE}\PY{p}{,} \PY{l+s+s2}{\PYZdq{}}\PY{l+s+s2}{CaernoElegans.gexf}\PY{l+s+s2}{\PYZdq{}}\PY{p}{)}
\end{Verbatim}
\end{tcolorbox}

    Exportamos la red a formato \(\texttt{.gexf}\) para su posterior
visualización con la herramienta \(\texttt{gephi}\). Algoritmo de
visualización: Fruchterman Reingold. Claramente podemos observar algunos
nodos clave, con un grado muy elevado. Este análisis se ejecutará con
más detalle en las próximas secciones de la práctica.

    \(\\\)

\textbf{3. Obtened e imprimid por la salida el orden y el tamaño del
grafo}

    \begin{tcolorbox}[breakable, size=fbox, boxrule=1pt, pad at break*=1mm,colback=cellbackground, colframe=cellborder]
\prompt{In}{incolor}{6}{\hspace{4pt}}
\begin{Verbatim}[commandchars=\\\{\}]
\PY{c+c1}{\PYZsh{}\PYZsh{} Orden y tamaño}
\PY{n}{tam} \PY{o}{=}  \PY{n}{nx}\PY{o}{.}\PY{n}{number\PYZus{}of\PYZus{}edges}\PY{p}{(}\PY{n}{G\PYZus{}CE}\PY{p}{)} \PY{c+c1}{\PYZsh{} numero de aristas}
\PY{n}{orden} \PY{o}{=} \PY{n}{nx}\PY{o}{.}\PY{n}{number\PYZus{}of\PYZus{}nodes}\PY{p}{(}\PY{n}{G\PYZus{}CE}\PY{p}{)} \PY{c+c1}{\PYZsh{} numero de vertices}

\PY{n+nb}{print} \PY{p}{(}\PY{l+s+s2}{\PYZdq{}}\PY{l+s+s2}{Tamaño:}\PY{l+s+s2}{\PYZdq{}} \PY{p}{,} \PY{n}{tam}\PY{p}{)}
\PY{n+nb}{print} \PY{p}{(}\PY{l+s+s2}{\PYZdq{}}\PY{l+s+s2}{Orden:}\PY{l+s+s2}{\PYZdq{}}\PY{p}{,} \PY{n}{orden} \PY{p}{)}
\end{Verbatim}
\end{tcolorbox}

    \begin{Verbatim}[commandchars=\\\{\}]
Tamaño: 1648
Orden: 1387
\end{Verbatim}

    También podría haberse obtenido a partir de la información que aporta la
función ``info()'' ded NetworkX

    \begin{tcolorbox}[breakable, size=fbox, boxrule=1pt, pad at break*=1mm,colback=cellbackground, colframe=cellborder]
\prompt{In}{incolor}{7}{\hspace{4pt}}
\begin{Verbatim}[commandchars=\\\{\}]
\PY{n+nb}{print}\PY{p}{(}\PY{n}{nx}\PY{o}{.}\PY{n}{info}\PY{p}{(}\PY{n}{G\PYZus{}CE}\PY{p}{)}\PY{p}{)}
\end{Verbatim}
\end{tcolorbox}

    \begin{Verbatim}[commandchars=\\\{\}]
Name: CaernoElegans
Type: Graph
Number of nodes: 1387
Number of edges: 1648
Average degree:   2.3764
\end{Verbatim}

    \(\\\)

\textbf{Averiguad si el grafo es dirigido o no}

Entre las funciones que incorpora NetworkX se encuentra
``is\_directed()'' que devuelve True en caso del que sea dirigido y
False en caso contrario

    \begin{tcolorbox}[breakable, size=fbox, boxrule=1pt, pad at break*=1mm,colback=cellbackground, colframe=cellborder]
\prompt{In}{incolor}{8}{\hspace{4pt}}
\begin{Verbatim}[commandchars=\\\{\}]
\PY{n+nb}{print}\PY{p}{(}\PY{n}{nx}\PY{o}{.}\PY{n}{is\PYZus{}directed}\PY{p}{(}\PY{n}{G\PYZus{}CE}\PY{p}{)}\PY{p}{)}
\end{Verbatim}
\end{tcolorbox}

    \begin{Verbatim}[commandchars=\\\{\}]
False
\end{Verbatim}

    Por tanto, el grafo cargado perteneciente a \emph{Caernobidis elegans}
es no dirigido.

    \(\\\)

\textbf{¿Es un grafo denso o disperso?}

La dispersión de un grafo se calcula con la siguiente fórmula:

\[ D =\frac{2|E|}{|V|(|V|-1)}\]

Donde: 
\begin{itemize}
    \item \(|E|\) es el número de aristas
    \item \(|V|\) es el número de vértices
\end{itemize}


    \begin{tcolorbox}[breakable, size=fbox, boxrule=1pt, pad at break*=1mm,colback=cellbackground, colframe=cellborder]
\prompt{In}{incolor}{9}{\hspace{4pt}}
\begin{Verbatim}[commandchars=\\\{\}]
\PY{n}{densidad} \PY{o}{=} \PY{p}{(}\PY{l+m+mi}{2}\PY{o}{*}\PY{n}{tam}\PY{p}{)}\PY{o}{/}\PY{p}{(}\PY{n}{orden}\PY{o}{*}\PY{p}{(}\PY{n}{orden}\PY{o}{\PYZhy{}}\PY{l+m+mi}{1}\PY{p}{)}\PY{p}{)}
\PY{n+nb}{print}\PY{p}{(}\PY{n+nb}{round}\PY{p}{(}\PY{n}{densidad}\PY{p}{,} \PY{l+m+mi}{5}\PY{p}{)}\PY{p}{)}
\end{Verbatim}
\end{tcolorbox}

    \begin{Verbatim}[commandchars=\\\{\}]
0.00171
\end{Verbatim}

    También se puede emplear la función ``density()'' de NetworkX

    \begin{tcolorbox}[breakable, size=fbox, boxrule=1pt, pad at break*=1mm,colback=cellbackground, colframe=cellborder]
\prompt{In}{incolor}{10}{\hspace{4pt}}
\begin{Verbatim}[commandchars=\\\{\}]
\PY{n}{densidad\PYZus{}2} \PY{o}{=} \PY{n}{nx}\PY{o}{.}\PY{n}{density}\PY{p}{(}\PY{n}{G\PYZus{}CE}\PY{p}{)}
\PY{n+nb}{print}\PY{p}{(}\PY{n+nb}{round}\PY{p}{(}\PY{n}{densidad\PYZus{}2}\PY{p}{,} \PY{l+m+mi}{5}\PY{p}{)}\PY{p}{)}
\end{Verbatim}
\end{tcolorbox}

    \begin{Verbatim}[commandchars=\\\{\}]
0.00171
\end{Verbatim}

    Al ser muy cercano a 0 podemos asegurar que se trata de un grafo
disperso. El hecho de se trate de un grafo disperso es bueno y lo hace
interesante de estudiar. Por un lado, los algoritmos empleados en
cálculos de parámetros de grafos proceden mucho más rápido en grafos
dispersos que en grafos densos. Igualmente, las redes biológicas son en
la mayor parte de las ocasiones grafos dispersos, lo que las hace muy
interesante desde el punto de vista de su estudio.

    \(\\\)

\textbf{4. Cread un grafo aleatorio G\_AL que tenga el mismo orden y
tamaño que el grafo que acabáis de cargar mediante la función
gnm\_random\_graph(n,m)}

Esta función nos permite generar un grafo aleatorio con \textbf{n} nodos
y \textbf{m} ramas por nodo.

    \begin{tcolorbox}[breakable, size=fbox, boxrule=1pt, pad at break*=1mm,colback=cellbackground, colframe=cellborder]
\prompt{In}{incolor}{11}{\hspace{4pt}}
\begin{Verbatim}[commandchars=\\\{\}]
\PY{c+c1}{\PYZsh{} Reutulizamos orden y tamaño calculado previamente}
\PY{n}{G\PYZus{}AL} \PY{o}{=} \PY{n}{nx}\PY{o}{.}\PY{n}{gnm\PYZus{}random\PYZus{}graph}\PY{p}{(}\PY{n}{orden}\PY{p}{,} \PY{n}{tam}\PY{p}{,} \PY{n}{seed} \PY{o}{=} \PY{l+m+mi}{1}\PY{p}{)}
\PY{n}{G\PYZus{}AL}\PY{o}{.}\PY{n}{name} \PY{o}{=} \PY{l+s+s2}{\PYZdq{}}\PY{l+s+s2}{Aleatorio}\PY{l+s+s2}{\PYZdq{}} \PY{c+c1}{\PYZsh{} añadimos el atributo nombre}

\PY{c+c1}{\PYZsh{} Comprobamos que las caracteristicas sean iguales}
\PY{n+nb}{print}\PY{p}{(}\PY{n}{nx}\PY{o}{.}\PY{n}{info}\PY{p}{(}\PY{n}{G\PYZus{}AL}\PY{p}{)}\PY{p}{)}
\end{Verbatim}
\end{tcolorbox}

    \begin{Verbatim}[commandchars=\\\{\}]
Name: Aleatorio
Type: Graph
Number of nodes: 1387
Number of edges: 1648
Average degree:   2.3764
\end{Verbatim}

    Igualemente, podemos conocer la probabilidad con la que se ha generado
el grafo, igualando las dos siguientes fórmulas:

\[ <k> \ = \frac{2|E|}{N} \quad (1) \]

\[  |E| \simeq p \cdot \frac{N \cdot (N-1)}{2} \quad (2) \]

\[ \frac{k \cdot N}{2} \simeq p \cdot \frac{N \cdot (N-1)}{2}  \rightarrow   p \simeq \frac{<k>}{N-1} \simeq 0.0017 \]

Donde:

\begin{itemize}
\item
  \(|E|\) es el número de aristas
\item
  \(<k>\) es el grado medio del grafo
\item
  \(N\) es el número d vértices del grafo
\item
  \(p\) es la probabilidad de que una rama esté presente en el grafo
  aleatorio
\end{itemize}

    \(\\\)

\hypertarget{repetimos-los-cuxe1lculos-para-20-grafos-aleatorios}{%
\paragraph{REPETIMOS LOS CÁLCULOS PARA 20 GRAFOS
ALEATORIOS}\label{repetimos-los-cuxe1lculos-para-20-grafos-aleatorios}}

Con el fin de realizar comparaciones justas. Esto se hace para tener
valores robustos de parámetros como el camino característico o el
coeficiente de clusterización. Después se comprobará si los valores
obtenidos para el grafo CaernoElegans y grafos aleatorios con el mismo
orden y tamaño son distintos de manera significativa, teniendo en cuenta
la media y la desviación estándar.

    \begin{tcolorbox}[breakable, size=fbox, boxrule=1pt, pad at break*=1mm,colback=cellbackground, colframe=cellborder]
\prompt{In}{incolor}{12}{\hspace{4pt}}
\begin{Verbatim}[commandchars=\\\{\}]
\PY{c+c1}{\PYZsh{} REPETIMOS LOS CÁLCULOS PARA 20 GRAFOS ALEATORIOS}

\PY{c+c1}{\PYZsh{} Creamos las listas contenedoras de los resultados de diferentes iteraciones}
\PY{n}{l\PYZus{}centralidad} \PY{o}{=} \PY{p}{[}\PY{p}{]}
\PY{n}{l\PYZus{}cercania} \PY{o}{=} \PY{p}{[}\PY{p}{]}
\PY{n}{l\PYZus{}nodo\PYZus{}max\PYZus{}centralidad} \PY{o}{=} \PY{p}{[}\PY{p}{]}
\PY{n}{l\PYZus{}nodo\PYZus{}max\PYZus{}cercania} \PY{o}{=} \PY{p}{[}\PY{p}{]}
\PY{n}{l\PYZus{}betweenness} \PY{o}{=} \PY{p}{[}\PY{p}{]}
\PY{n}{l\PYZus{}nodo\PYZus{}max\PYZus{}bt} \PY{o}{=} \PY{p}{[}\PY{p}{]}
\PY{n}{l\PYZus{}clustering} \PY{o}{=} \PY{p}{[}\PY{p}{]}
\PY{n}{l\PYZus{}nodo\PYZus{}max\PYZus{}clust} \PY{o}{=} \PY{p}{[}\PY{p}{]}
\PY{n}{l\PYZus{}kcores} \PY{o}{=} \PY{p}{[}\PY{p}{]}
\PY{n}{l\PYZus{}min\PYZus{}path} \PY{o}{=} \PY{p}{[}\PY{p}{]}
\PY{n}{l\PYZus{}num\PYZus{}comp} \PY{o}{=} \PY{p}{[}\PY{p}{]}
\PY{n}{l\PYZus{}diameter} \PY{o}{=} \PY{p}{[}\PY{p}{]}

\PY{c+c1}{\PYZsh{} Generamos un bucle para poder hacer métricas estadísticas de los parámetros del grafo aleatorio}
\PY{k}{for} \PY{n}{i} \PY{o+ow}{in} \PY{n+nb}{range}\PY{p}{(}\PY{l+m+mi}{0}\PY{p}{,} \PY{l+m+mi}{20}\PY{p}{)}\PY{p}{:}
    \PY{n}{g} \PY{o}{=} \PY{n}{nx}\PY{o}{.}\PY{n}{gnm\PYZus{}random\PYZus{}graph}\PY{p}{(}\PY{n}{orden}\PY{p}{,} \PY{n}{tam}\PY{p}{,} \PY{n}{seed}\PY{o}{=}\PY{n}{i}\PY{p}{)}
    
    \PY{c+c1}{\PYZsh{} Centralidad}
    \PY{n}{l\PYZus{}centralidad}\PY{o}{.}\PY{n}{append}\PY{p}{(}\PY{n+nb}{sum}\PY{p}{(}\PY{n}{nx}\PY{o}{.}\PY{n}{degree\PYZus{}centrality}\PY{p}{(}\PY{n}{g}\PY{p}{)}\PY{o}{.}\PY{n}{values}\PY{p}{(}\PY{p}{)}\PY{p}{)}\PY{p}{)}
    \PY{n}{l\PYZus{}nodo\PYZus{}max\PYZus{}centralidad}\PY{o}{.}\PY{n}{append}\PY{p}{(}\PY{n+nb}{max}\PY{p}{(}\PY{n}{nx}\PY{o}{.}\PY{n}{degree\PYZus{}centrality}\PY{p}{(}\PY{n}{g}\PY{p}{)}\PY{o}{.}\PY{n}{items}\PY{p}{(}\PY{p}{)}\PY{p}{,} \PY{n}{key} \PY{o}{=} \PY{k}{lambda} \PY{n}{pareja}\PY{p}{:} \PY{n}{pareja}\PY{p}{[}\PY{l+m+mi}{1}\PY{p}{]}\PY{p}{)}\PY{p}{[}\PY{l+m+mi}{1}\PY{p}{]}\PY{p}{)}
    
    \PY{c+c1}{\PYZsh{} Cercanía}
    \PY{n}{l\PYZus{}cercania}\PY{o}{.}\PY{n}{append}\PY{p}{(}\PY{n+nb}{sum}\PY{p}{(}\PY{n}{nx}\PY{o}{.}\PY{n}{closeness\PYZus{}centrality}\PY{p}{(}\PY{n}{g}\PY{p}{)}\PY{o}{.}\PY{n}{values}\PY{p}{(}\PY{p}{)}\PY{p}{)}\PY{p}{)}
    \PY{n}{l\PYZus{}nodo\PYZus{}max\PYZus{}cercania}\PY{o}{.}\PY{n}{append}\PY{p}{(}\PY{n+nb}{max}\PY{p}{(}\PY{n}{nx}\PY{o}{.}\PY{n}{closeness\PYZus{}centrality}\PY{p}{(}\PY{n}{g}\PY{p}{)}\PY{o}{.}\PY{n}{items}\PY{p}{(}\PY{p}{)}\PY{p}{,} \PY{n}{key} \PY{o}{=} \PY{k}{lambda} \PY{n}{pareja}\PY{p}{:} \PY{n}{pareja}\PY{p}{[}\PY{l+m+mi}{1}\PY{p}{]}\PY{p}{)}\PY{p}{[}\PY{l+m+mi}{1}\PY{p}{]}\PY{p}{)}

    \PY{c+c1}{\PYZsh{} Betweenness}
    \PY{n}{l\PYZus{}betweenness}\PY{o}{.}\PY{n}{append}\PY{p}{(}\PY{n+nb}{sum}\PY{p}{(}\PY{n}{nx}\PY{o}{.}\PY{n}{betweenness\PYZus{}centrality}\PY{p}{(}\PY{n}{g}\PY{p}{)}\PY{o}{.}\PY{n}{values}\PY{p}{(}\PY{p}{)}\PY{p}{)} \PY{o}{/} \PY{n}{orden}\PY{p}{)}
    \PY{n}{l\PYZus{}nodo\PYZus{}max\PYZus{}bt}\PY{o}{.}\PY{n}{append}\PY{p}{(}\PY{n+nb}{max}\PY{p}{(}\PY{n}{nx}\PY{o}{.}\PY{n}{betweenness\PYZus{}centrality}\PY{p}{(}\PY{n}{g}\PY{p}{)}\PY{o}{.}\PY{n}{items}\PY{p}{(}\PY{p}{)}\PY{p}{,} \PY{n}{key} \PY{o}{=} \PY{k}{lambda} \PY{n}{pareja}\PY{p}{:} \PY{n}{pareja}\PY{p}{[}\PY{l+m+mi}{1}\PY{p}{]}\PY{p}{)}\PY{p}{)}

    \PY{c+c1}{\PYZsh{} Indice de clusterizacion}
    \PY{n}{l\PYZus{}clustering}\PY{o}{.}\PY{n}{append}\PY{p}{(}\PY{n}{nx}\PY{o}{.}\PY{n}{average\PYZus{}clustering}\PY{p}{(}\PY{n}{g}\PY{p}{)}\PY{p}{)}
    \PY{n}{l\PYZus{}nodo\PYZus{}max\PYZus{}clust}\PY{o}{.}\PY{n}{append}\PY{p}{(}\PY{n+nb}{max}\PY{p}{(}\PY{n}{nx}\PY{o}{.}\PY{n}{clustering}\PY{p}{(}\PY{n}{g}\PY{p}{)}\PY{o}{.}\PY{n}{items}\PY{p}{(}\PY{p}{)}\PY{p}{,} \PY{n}{key} \PY{o}{=} \PY{k}{lambda} \PY{n}{pareja}\PY{p}{:} \PY{n}{pareja}\PY{p}{[}\PY{l+m+mi}{1}\PY{p}{]}\PY{p}{)}\PY{p}{[}\PY{l+m+mi}{1}\PY{p}{]}\PY{p}{)}

    \PY{c+c1}{\PYZsh{} max k\PYZhy{}core}
    \PY{n}{l\PYZus{}kcores}\PY{o}{.}\PY{n}{append}\PY{p}{(}\PY{n+nb}{max}\PY{p}{(}\PY{n}{nx}\PY{o}{.}\PY{n}{core\PYZus{}number}\PY{p}{(}\PY{n}{g}\PY{p}{)}\PY{o}{.}\PY{n}{values}\PY{p}{(}\PY{p}{)}\PY{p}{)}\PY{p}{)}

    \PY{c+c1}{\PYZsh{} Camino caracteristico}
    \PY{n}{l\PYZus{}min\PYZus{}path}\PY{o}{.}\PY{n}{append}\PY{p}{(}\PY{n}{nx}\PY{o}{.}\PY{n}{average\PYZus{}shortest\PYZus{}path\PYZus{}length}\PY{p}{(}\PY{n+nb}{max}\PY{p}{(}\PY{n}{nx}\PY{o}{.}\PY{n}{connected\PYZus{}component\PYZus{}subgraphs}\PY{p}{(}\PY{n}{g}\PY{p}{)}\PY{p}{,} \PY{n}{key} \PY{o}{=} \PY{n+nb}{len}\PY{p}{)}\PY{p}{)}\PY{p}{)}

    \PY{c+c1}{\PYZsh{} Numero de componente y diametro de la componente gigante}
    \PY{n}{l\PYZus{}num\PYZus{}comp}\PY{o}{.}\PY{n}{append}\PY{p}{(}\PY{n}{nx}\PY{o}{.}\PY{n}{number\PYZus{}connected\PYZus{}components}\PY{p}{(}\PY{n}{g}\PY{p}{)}\PY{p}{)}
    \PY{n}{l\PYZus{}diameter}\PY{o}{.}\PY{n}{append}\PY{p}{(}\PY{n}{nx}\PY{o}{.}\PY{n}{diameter}\PY{p}{(}\PY{n+nb}{max}\PY{p}{(}\PY{n}{nx}\PY{o}{.}\PY{n}{connected\PYZus{}component\PYZus{}subgraphs}\PY{p}{(}\PY{n}{g}\PY{p}{)}\PY{p}{,} \PY{n}{key} \PY{o}{=} \PY{n+nb}{len}\PY{p}{)}\PY{p}{)}\PY{p}{)}

    
    
\PY{c+c1}{\PYZsh{} Guardamos los resultados de la media y des. std en un diccionario para acceder posteriormente}
\PY{n}{dic\PYZus{}params} \PY{o}{=} \PY{p}{\PYZob{}}\PY{p}{\PYZcb{}}

\PY{c+c1}{\PYZsh{} Centralidad}
\PY{n}{dic\PYZus{}params}\PY{p}{[}\PY{l+s+s2}{\PYZdq{}}\PY{l+s+s2}{Degree Centrality}\PY{l+s+s2}{\PYZdq{}}\PY{p}{]} \PY{o}{=} \PY{n}{np}\PY{o}{.}\PY{n}{mean}\PY{p}{(}\PY{n}{np}\PY{o}{.}\PY{n}{array}\PY{p}{(}\PY{n}{l\PYZus{}centralidad}\PY{p}{)}\PY{p}{)}\PY{p}{,} \PY{n}{np}\PY{o}{.}\PY{n}{std}\PY{p}{(}\PY{n}{np}\PY{o}{.}\PY{n}{array}\PY{p}{(}\PY{n}{l\PYZus{}centralidad}\PY{p}{)}\PY{p}{)}
\PY{n}{dic\PYZus{}params}\PY{p}{[}\PY{l+s+s2}{\PYZdq{}}\PY{l+s+s2}{Nodo max Degree Centrality}\PY{l+s+s2}{\PYZdq{}}\PY{p}{]} \PY{o}{=} \PY{n}{np}\PY{o}{.}\PY{n}{mean}\PY{p}{(}\PY{n}{np}\PY{o}{.}\PY{n}{array}\PY{p}{(}\PY{n}{l\PYZus{}nodo\PYZus{}max\PYZus{}centralidad}\PY{p}{)}\PY{p}{)}\PY{p}{,} \PY{n}{np}\PY{o}{.}\PY{n}{std}\PY{p}{(}\PY{n}{np}\PY{o}{.}\PY{n}{array}\PY{p}{(}\PY{n}{l\PYZus{}nodo\PYZus{}max\PYZus{}centralidad}\PY{p}{)}\PY{p}{)}

\PY{c+c1}{\PYZsh{} Cercanía}
\PY{n}{dic\PYZus{}params}\PY{p}{[}\PY{l+s+s2}{\PYZdq{}}\PY{l+s+s2}{Cercania}\PY{l+s+s2}{\PYZdq{}}\PY{p}{]} \PY{o}{=} \PY{n}{np}\PY{o}{.}\PY{n}{mean}\PY{p}{(}\PY{n}{np}\PY{o}{.}\PY{n}{array}\PY{p}{(}\PY{n}{l\PYZus{}cercania}\PY{p}{)}\PY{p}{)}\PY{p}{,} \PY{n}{np}\PY{o}{.}\PY{n}{std}\PY{p}{(}\PY{n}{np}\PY{o}{.}\PY{n}{array}\PY{p}{(}\PY{n}{l\PYZus{}cercania}\PY{p}{)}\PY{p}{)}
\PY{n}{dic\PYZus{}params}\PY{p}{[}\PY{l+s+s2}{\PYZdq{}}\PY{l+s+s2}{Nodo max cercania}\PY{l+s+s2}{\PYZdq{}}\PY{p}{]} \PY{o}{=} \PY{n}{np}\PY{o}{.}\PY{n}{mean}\PY{p}{(}\PY{n}{np}\PY{o}{.}\PY{n}{array}\PY{p}{(}\PY{n}{l\PYZus{}nodo\PYZus{}max\PYZus{}cercania}\PY{p}{)}\PY{p}{)}\PY{p}{,} \PY{n}{np}\PY{o}{.}\PY{n}{std}\PY{p}{(}\PY{n}{np}\PY{o}{.}\PY{n}{array}\PY{p}{(}\PY{n}{l\PYZus{}nodo\PYZus{}max\PYZus{}cercania}\PY{p}{)}\PY{p}{)}

\PY{c+c1}{\PYZsh{} Betweenness}
\PY{n}{dic\PYZus{}params}\PY{p}{[}\PY{l+s+s2}{\PYZdq{}}\PY{l+s+s2}{Betweenness}\PY{l+s+s2}{\PYZdq{}}\PY{p}{]} \PY{o}{=} \PY{n}{np}\PY{o}{.}\PY{n}{mean}\PY{p}{(}\PY{n}{np}\PY{o}{.}\PY{n}{array}\PY{p}{(}\PY{n}{l\PYZus{}betweenness}\PY{p}{)}\PY{p}{)}\PY{p}{,} \PY{n}{np}\PY{o}{.}\PY{n}{std}\PY{p}{(}\PY{n}{np}\PY{o}{.}\PY{n}{array}\PY{p}{(}\PY{n}{l\PYZus{}betweenness}\PY{p}{)}\PY{p}{)}
\PY{n}{dic\PYZus{}params}\PY{p}{[}\PY{l+s+s2}{\PYZdq{}}\PY{l+s+s2}{Nodo max betweenness}\PY{l+s+s2}{\PYZdq{}}\PY{p}{]} \PY{o}{=} \PY{n}{np}\PY{o}{.}\PY{n}{mean}\PY{p}{(}\PY{n}{np}\PY{o}{.}\PY{n}{array}\PY{p}{(}\PY{n}{l\PYZus{}nodo\PYZus{}max\PYZus{}bt}\PY{p}{)}\PY{p}{)}\PY{p}{,} \PY{n}{np}\PY{o}{.}\PY{n}{std}\PY{p}{(}\PY{n}{np}\PY{o}{.}\PY{n}{array}\PY{p}{(}\PY{n}{l\PYZus{}nodo\PYZus{}max\PYZus{}bt}\PY{p}{)}\PY{p}{)}

\PY{c+c1}{\PYZsh{} Indice de clusterizacion}
\PY{n}{dic\PYZus{}params}\PY{p}{[}\PY{l+s+s2}{\PYZdq{}}\PY{l+s+s2}{Clustering}\PY{l+s+s2}{\PYZdq{}}\PY{p}{]} \PY{o}{=} \PY{n}{np}\PY{o}{.}\PY{n}{mean}\PY{p}{(}\PY{n}{np}\PY{o}{.}\PY{n}{array}\PY{p}{(}\PY{n}{l\PYZus{}clustering}\PY{p}{)}\PY{p}{)}\PY{p}{,} \PY{n}{np}\PY{o}{.}\PY{n}{std}\PY{p}{(}\PY{n}{np}\PY{o}{.}\PY{n}{array}\PY{p}{(}\PY{n}{l\PYZus{}clustering}\PY{p}{)}\PY{p}{)}
\PY{n}{dic\PYZus{}params}\PY{p}{[}\PY{l+s+s2}{\PYZdq{}}\PY{l+s+s2}{Nodo max clustering}\PY{l+s+s2}{\PYZdq{}}\PY{p}{]} \PY{o}{=} \PY{n}{np}\PY{o}{.}\PY{n}{mean}\PY{p}{(}\PY{n}{np}\PY{o}{.}\PY{n}{array}\PY{p}{(}\PY{n}{l\PYZus{}nodo\PYZus{}max\PYZus{}clust}\PY{p}{)}\PY{p}{)}\PY{p}{,} \PY{n}{np}\PY{o}{.}\PY{n}{std}\PY{p}{(}\PY{n}{np}\PY{o}{.}\PY{n}{array}\PY{p}{(}\PY{n}{l\PYZus{}nodo\PYZus{}max\PYZus{}clust}\PY{p}{)}\PY{p}{)}

\PY{c+c1}{\PYZsh{} max k\PYZhy{}core}
\PY{n}{dic\PYZus{}params}\PY{p}{[}\PY{l+s+s2}{\PYZdq{}}\PY{l+s+s2}{Max k\PYZhy{}core}\PY{l+s+s2}{\PYZdq{}}\PY{p}{]} \PY{o}{=} \PY{n}{np}\PY{o}{.}\PY{n}{mean}\PY{p}{(}\PY{n}{np}\PY{o}{.}\PY{n}{array}\PY{p}{(}\PY{n}{l\PYZus{}kcores}\PY{p}{)}\PY{p}{)}\PY{p}{,} \PY{n}{np}\PY{o}{.}\PY{n}{std}\PY{p}{(}\PY{n}{np}\PY{o}{.}\PY{n}{array}\PY{p}{(}\PY{n}{l\PYZus{}kcores}\PY{p}{)}\PY{p}{)}

\PY{c+c1}{\PYZsh{} Camino caracteristico}
\PY{n}{dic\PYZus{}params}\PY{p}{[}\PY{l+s+s2}{\PYZdq{}}\PY{l+s+s2}{Shortest path}\PY{l+s+s2}{\PYZdq{}}\PY{p}{]} \PY{o}{=} \PY{n}{np}\PY{o}{.}\PY{n}{mean}\PY{p}{(}\PY{n}{np}\PY{o}{.}\PY{n}{array}\PY{p}{(}\PY{n}{l\PYZus{}min\PYZus{}path}\PY{p}{)}\PY{p}{)}\PY{p}{,} \PY{n}{np}\PY{o}{.}\PY{n}{std}\PY{p}{(}\PY{n}{np}\PY{o}{.}\PY{n}{array}\PY{p}{(}\PY{n}{l\PYZus{}min\PYZus{}path}\PY{p}{)}\PY{p}{)}

\PY{c+c1}{\PYZsh{} Numero de componente y diametro de la componente gigante}
\PY{n}{dic\PYZus{}params}\PY{p}{[}\PY{l+s+s2}{\PYZdq{}}\PY{l+s+s2}{Componentes}\PY{l+s+s2}{\PYZdq{}}\PY{p}{]} \PY{o}{=} \PY{n}{np}\PY{o}{.}\PY{n}{mean}\PY{p}{(}\PY{n}{np}\PY{o}{.}\PY{n}{array}\PY{p}{(}\PY{n}{l\PYZus{}num\PYZus{}comp}\PY{p}{)}\PY{p}{)}\PY{p}{,} \PY{n}{np}\PY{o}{.}\PY{n}{std}\PY{p}{(}\PY{n}{np}\PY{o}{.}\PY{n}{array}\PY{p}{(}\PY{n}{l\PYZus{}num\PYZus{}comp}\PY{p}{)}\PY{p}{)}
\PY{n}{dic\PYZus{}params}\PY{p}{[}\PY{l+s+s2}{\PYZdq{}}\PY{l+s+s2}{Diámetro}\PY{l+s+s2}{\PYZdq{}}\PY{p}{]} \PY{o}{=} \PY{n}{np}\PY{o}{.}\PY{n}{mean}\PY{p}{(}\PY{n}{np}\PY{o}{.}\PY{n}{array}\PY{p}{(}\PY{n}{l\PYZus{}diameter}\PY{p}{)}\PY{p}{)}\PY{p}{,} \PY{n}{np}\PY{o}{.}\PY{n}{std}\PY{p}{(}\PY{n}{np}\PY{o}{.}\PY{n}{array}\PY{p}{(}\PY{n}{l\PYZus{}diameter}\PY{p}{)}\PY{p}{)}

\PY{c+c1}{\PYZsh{} Visualizamos los resultados de los parámetros. }
\PY{n+nb}{print}\PY{p}{(}\PY{n}{dic\PYZus{}params}\PY{p}{)}
\end{Verbatim}
\end{tcolorbox}

    \begin{Verbatim}[commandchars=\\\{\}]
\{'Degree Centrality': (2.378066378066355, 1.9100999153570945e-15), 'Nodo max
Degree Centrality': (0.006493506493506494, 0.0006453298636363029), 'Cercania':
(135.34516157436457, 2.3001931602143246), 'Nodo max cercania':
(0.1455585998306277, 0.0038772866573120023), 'Betweenness':
(0.003852702791742197, 0.0001306928874608582), 'Nodo max betweenness':
(328.47126017957453, 425.60895451056115), 'Clustering': (0.0009396727601486074,
0.001109340196485008), 'Nodo max clustering': (0.4133333333333334,
0.4019950248448356), 'Max k-core': (2.0, 0.0), 'Shortest path':
(7.9675033029459525, 0.10250774207511149), 'Componentes': (146.55,
7.651633812461231), 'Diámetro': (19.1, 1.337908816025965)\}
\end{Verbatim}

    \(\\\)

\textbf{5. Indica si ambos grafos son conexos}

Según la definición de conexidad, un grafo es conexo si para cada par de
nodos del grafo existe al menos un camino que los une (paseo de un un
nodo \(u\) a un nodo \(v\) en el cual todos los vértices
\(\{v_0, v_1,...,v_k\}\) son distintos).

Esta característica del grafo puede conocerse a partir de la función
``is\_connected()'' de NetworkX

    \begin{tcolorbox}[breakable, size=fbox, boxrule=1pt, pad at break*=1mm,colback=cellbackground, colframe=cellborder]
\prompt{In}{incolor}{13}{\hspace{4pt}}
\begin{Verbatim}[commandchars=\\\{\}]
\PY{n+nb}{print}\PY{p}{(}\PY{l+s+s2}{\PYZdq{}}\PY{l+s+s2}{Aleatorio (G\PYZus{}AL) \PYZhy{}\PYZhy{}\PYZgt{}}\PY{l+s+s2}{\PYZdq{}}\PY{p}{,} \PY{n}{nx}\PY{o}{.}\PY{n}{is\PYZus{}connected}\PY{p}{(}\PY{n}{G\PYZus{}AL}\PY{p}{)}\PY{p}{,}
      \PY{l+s+s2}{\PYZdq{}}\PY{l+s+se}{\PYZbs{}n}\PY{l+s+s2}{Número de componentes (G\PYZus{}AL):}\PY{l+s+s2}{\PYZdq{}}\PY{p}{,} \PY{n}{nx}\PY{o}{.}\PY{n}{number\PYZus{}connected\PYZus{}components}\PY{p}{(}\PY{n}{G\PYZus{}AL}\PY{p}{)}\PY{p}{,} 
      \PY{l+s+s2}{\PYZdq{}}\PY{l+s+se}{\PYZbs{}n}\PY{l+s+se}{\PYZbs{}n}\PY{l+s+s2}{Grafo inicial (G\PYZus{}CE) \PYZhy{}\PYZhy{}\PYZgt{} }\PY{l+s+s2}{\PYZdq{}}\PY{p}{,} \PY{n}{nx}\PY{o}{.}\PY{n}{is\PYZus{}connected}\PY{p}{(}\PY{n}{G\PYZus{}CE}\PY{p}{)}\PY{p}{,}
      \PY{l+s+s2}{\PYZdq{}}\PY{l+s+se}{\PYZbs{}n}\PY{l+s+s2}{Número de componentes (G\PYZus{}CE):}\PY{l+s+s2}{\PYZdq{}}\PY{p}{,} \PY{n}{nx}\PY{o}{.}\PY{n}{number\PYZus{}connected\PYZus{}components}\PY{p}{(}\PY{n}{G\PYZus{}CE}\PY{p}{)}\PY{p}{)}
\end{Verbatim}
\end{tcolorbox}

    \begin{Verbatim}[commandchars=\\\{\}]
Aleatorio (G\_AL) --> False
Número de componentes (G\_AL): 142

Grafo inicial (G\_CE) -->  False
Número de componentes (G\_CE): 89
\end{Verbatim}

    Ninguno de los grafos es conexo.

    \textbf{6. ¿Cuál es el nodo con mayor grado en cada grafo?}

El grado de los nodos de un grafo (número de vecinos adyacentes que
posee cada nodo) se puede conocer con el atributo ``G.degree()'' del
objeto creado por NetworkX.

    \begin{tcolorbox}[breakable, size=fbox, boxrule=1pt, pad at break*=1mm,colback=cellbackground, colframe=cellborder]
\prompt{In}{incolor}{14}{\hspace{4pt}}
\begin{Verbatim}[commandchars=\\\{\}]
\PY{c+c1}{\PYZsh{}\PYZsh{} Generamos un diccionario con el nodo como clave y el grado como valor}
\PY{n}{degree\PYZus{}dic} \PY{o}{=} \PY{n+nb}{dict}\PY{p}{(}\PY{n}{G\PYZus{}CE}\PY{o}{.}\PY{n}{degree}\PY{p}{(}\PY{p}{)}\PY{p}{)}

\PY{c+c1}{\PYZsh{}\PYZsh{} Generamos una tupla con la pareja clave\PYZhy{}valor, y extraemos la pareja con mayor valor (grado)}
\PY{n+nb}{print}\PY{p}{(}\PY{n+nb}{max}\PY{p}{(}\PY{n}{degree\PYZus{}dic}\PY{o}{.}\PY{n}{items}\PY{p}{(}\PY{p}{)}\PY{p}{,} \PY{n}{key} \PY{o}{=} \PY{k}{lambda} \PY{n}{k}\PY{p}{:} \PY{n}{k}\PY{p}{[}\PY{l+m+mi}{1}\PY{p}{]}\PY{p}{)}\PY{p}{)}
\end{Verbatim}
\end{tcolorbox}

    \begin{Verbatim}[commandchars=\\\{\}]
('T08G11.5', 131)
\end{Verbatim}

    Encontramos la proteína T08G11.5 con un grado muy alto (131). Esto puede
indicar que es una proteína involucrada en un gran número de procesos.

    \textbf{7. ¿Cuál es la máxima distancia entre dos nodos del grafo
(diámetro del grafo)?}

El diámetro del grafo puede obtenerse con la función ``diameter()'' de
NetworkX.

    \begin{tcolorbox}[breakable, size=fbox, boxrule=1pt, pad at break*=1mm,colback=cellbackground, colframe=cellborder]
\prompt{In}{incolor}{15}{\hspace{4pt}}
\begin{Verbatim}[commandchars=\\\{\}]
\PY{k}{try}\PY{p}{:}
    \PY{n+nb}{print}\PY{p}{(}\PY{n}{nx}\PY{o}{.}\PY{n}{diameter}\PY{p}{(}\PY{n}{G\PYZus{}CE}\PY{p}{)}\PY{p}{)} \PY{c+c1}{\PYZsh{} If the diameter does not exit give an error show it}
    
\PY{k}{except}\PY{p}{:}
    \PY{n+nb}{print}\PY{p}{(}\PY{l+s+s2}{\PYZdq{}}\PY{l+s+s2}{El diámetro del grafo no se puede conocer dado que se trata de un grafo no conexo}\PY{l+s+s2}{\PYZdq{}}\PY{p}{)}
    \PY{n}{diametro\PYZus{}G\PYZus{}CE} \PY{o}{=} \PY{n}{nx}\PY{o}{.}\PY{n}{diameter}\PY{p}{(}\PY{n+nb}{max}\PY{p}{(}\PY{n}{nx}\PY{o}{.}\PY{n}{connected\PYZus{}component\PYZus{}subgraphs}\PY{p}{(}\PY{n}{G\PYZus{}CE}\PY{p}{)}\PY{p}{,} \PY{n}{key} \PY{o}{=} \PY{n+nb}{len}\PY{p}{)}\PY{p}{)}
    \PY{n+nb}{print}\PY{p}{(}\PY{l+s+s2}{\PYZdq{}}\PY{l+s+s2}{En su lugar se muestra el diámetro de la mayor componente del grafo CaernoElegans:}\PY{l+s+s2}{\PYZdq{}}\PY{p}{,} \PY{n}{diametro\PYZus{}G\PYZus{}CE}\PY{p}{)}
\end{Verbatim}
\end{tcolorbox}

    \begin{Verbatim}[commandchars=\\\{\}]
El diámetro del grafo no se puede conocer dado que se trata de un grafo no
conexo
En su lugar se muestra el diámetro de la mayor componente del grafo
CaernoElegans: 22
\end{Verbatim}

    Dado que en el grafo aleatorio el grado medio es superior a 1
\((<k> > 1)\) aparece un cluster gigante, siendo el diámetro del clúster
gigante el diámetro del grafo aleatorio.

    \begin{tcolorbox}[breakable, size=fbox, boxrule=1pt, pad at break*=1mm,colback=cellbackground, colframe=cellborder]
\prompt{In}{incolor}{16}{\hspace{4pt}}
\begin{Verbatim}[commandchars=\\\{\}]
\PY{n+nb}{print}\PY{p}{(}\PY{l+s+s2}{\PYZdq{}}\PY{l+s+s2}{El diámetro promedio de la componente gigante del grafo aleatorio es:}\PY{l+s+s2}{\PYZdq{}}\PY{p}{,} \PY{n}{dic\PYZus{}params}\PY{p}{[}\PY{l+s+s2}{\PYZdq{}}\PY{l+s+s2}{Diámetro}\PY{l+s+s2}{\PYZdq{}}\PY{p}{]}\PY{p}{[}\PY{l+m+mi}{0}\PY{p}{]}\PY{p}{,} \PY{l+s+s2}{\PYZdq{}}\PY{l+s+s2}{con una desviación estándar de}\PY{l+s+s2}{\PYZdq{}}\PY{p}{,}
     \PY{n}{dic\PYZus{}params}\PY{p}{[}\PY{l+s+s2}{\PYZdq{}}\PY{l+s+s2}{Diámetro}\PY{l+s+s2}{\PYZdq{}}\PY{p}{]}\PY{p}{[}\PY{l+m+mi}{1}\PY{p}{]}\PY{p}{)}
\end{Verbatim}
\end{tcolorbox}

    \begin{Verbatim}[commandchars=\\\{\}]
El diámetro promedio de la componente gigante del grafo aleatorio es: 19.1 con
una desviación estándar de 1.337908816025965
\end{Verbatim}

    El diametro es interesante conocerlo porque es una cota superior al
camino característico, puesto que el camino característico nunca puede
ser superior al diámetro al igual que una media no puede ser superior al
mayor valor de la muestra. Por lo tanto, si el diámetro es pequeño se
comprobará que el camino característico es pequeño.

    \hypertarget{apartado-3-distribuciuxf3n-del-grado-de-los-nodos}{%
\subsection{Apartado 3: Distribución del grado de los
nodos}\label{apartado-3-distribuciuxf3n-del-grado-de-los-nodos}}

    \textbf{1. Visualizad la distribución del grado de los nodos de ambos
grafos}

Primero mostramos la distribución para el grafo CaernoElegans y después
para un grafo aleatorio de mismo orden y magnitud.

    \begin{tcolorbox}[breakable, size=fbox, boxrule=1pt, pad at break*=1mm,colback=cellbackground, colframe=cellborder]
\prompt{In}{incolor}{17}{\hspace{4pt}}
\begin{Verbatim}[commandchars=\\\{\}]
\PY{c+c1}{\PYZsh{}\PYZsh{} Distribucion del grado de los nodos del grafo inicial}
\PY{n}{L}\PY{o}{=}\PY{n}{nx}\PY{o}{.}\PY{n}{degree\PYZus{}histogram}\PY{p}{(}\PY{n}{G\PYZus{}CE}\PY{p}{)}

\PY{c+c1}{\PYZsh{} tamaño figura}
\PY{n}{plt}\PY{o}{.}\PY{n}{figure}\PY{p}{(}\PY{n}{figsize} \PY{o}{=} \PY{p}{(}\PY{l+m+mi}{12}\PY{p}{,} \PY{l+m+mi}{8}\PY{p}{)}\PY{p}{)}
\PY{n}{plt}\PY{o}{.}\PY{n}{bar}\PY{p}{(}\PY{n+nb}{range}\PY{p}{(}\PY{n+nb}{len}\PY{p}{(}\PY{n}{L}\PY{p}{)}\PY{p}{)}\PY{p}{,}\PY{n}{L}\PY{p}{,} \PY{n}{color}\PY{o}{=}\PY{l+s+s1}{\PYZsq{}}\PY{l+s+s1}{lightblue}\PY{l+s+s1}{\PYZsq{}}\PY{p}{,} \PY{n}{width} \PY{o}{=} \PY{l+m+mi}{1}\PY{p}{,} \PY{n}{edgecolor} \PY{o}{=} \PY{l+s+s2}{\PYZdq{}}\PY{l+s+s2}{black}\PY{l+s+s2}{\PYZdq{}}\PY{p}{,} \PY{n}{align} \PY{o}{=} \PY{l+s+s1}{\PYZsq{}}\PY{l+s+s1}{center}\PY{l+s+s1}{\PYZsq{}}\PY{p}{)}

\PY{c+c1}{\PYZsh{} titulo}
\PY{n}{plt}\PY{o}{.}\PY{n}{title}\PY{p}{(}\PY{l+s+s2}{\PYZdq{}}\PY{l+s+s2}{Distribución de grado del grafo CaernoElegans}\PY{l+s+s2}{\PYZdq{}}\PY{p}{,} \PY{n}{fontsize} \PY{o}{=} \PY{l+m+mi}{20}\PY{p}{)}

\PY{c+c1}{\PYZsh{} ejes X y Y}
\PY{n}{plt}\PY{o}{.}\PY{n}{ylabel}\PY{p}{(}\PY{l+s+s2}{\PYZdq{}}\PY{l+s+s2}{Número de nodos}\PY{l+s+s2}{\PYZdq{}}\PY{p}{,} \PY{n}{fontsize} \PY{o}{=} \PY{l+m+mi}{16}\PY{p}{)}
\PY{n}{plt}\PY{o}{.}\PY{n}{xlabel}\PY{p}{(}\PY{l+s+s2}{\PYZdq{}}\PY{l+s+s2}{Grado (k)}\PY{l+s+s2}{\PYZdq{}}\PY{p}{,} \PY{n}{fontsize} \PY{o}{=} \PY{l+m+mi}{16}\PY{p}{)}
\PY{n}{plt}\PY{o}{.}\PY{n}{xticks}\PY{p}{(}\PY{n}{fontsize} \PY{o}{=} \PY{l+m+mi}{12}\PY{p}{)}
\PY{n}{plt}\PY{o}{.}\PY{n}{yticks}\PY{p}{(}\PY{n}{fontsize} \PY{o}{=} \PY{l+m+mi}{12}\PY{p}{)}

\PY{n}{plt}\PY{o}{.}\PY{n}{show}\PY{p}{(}\PY{p}{)}
\end{Verbatim}
\end{tcolorbox}

    \begin{center}
    \adjustimage{max size={0.9\linewidth}{0.9\paperheight}}{output_42_0.png}
    \end{center}
    { \hspace*{\fill} \\}
    
    \begin{tcolorbox}[breakable, size=fbox, boxrule=1pt, pad at break*=1mm,colback=cellbackground, colframe=cellborder]
\prompt{In}{incolor}{18}{\hspace{4pt}}
\begin{Verbatim}[commandchars=\\\{\}]
\PY{c+c1}{\PYZsh{}\PYZsh{} Distribucion del grado de los nodos del grafo aleatorio}
\PY{n}{L}\PY{o}{=}\PY{n}{nx}\PY{o}{.}\PY{n}{degree\PYZus{}histogram}\PY{p}{(}\PY{n}{G\PYZus{}AL}\PY{p}{)}

\PY{c+c1}{\PYZsh{} tamaño figura}
\PY{n}{plt}\PY{o}{.}\PY{n}{figure}\PY{p}{(}\PY{n}{figsize} \PY{o}{=} \PY{p}{(}\PY{l+m+mi}{12}\PY{p}{,} \PY{l+m+mi}{8}\PY{p}{)}\PY{p}{)}
\PY{n}{plt}\PY{o}{.}\PY{n}{bar}\PY{p}{(}\PY{n+nb}{range}\PY{p}{(}\PY{n+nb}{len}\PY{p}{(}\PY{n}{L}\PY{p}{)}\PY{p}{)}\PY{p}{,}\PY{n}{L}\PY{p}{,} \PY{n}{color}\PY{o}{=}\PY{l+s+s1}{\PYZsq{}}\PY{l+s+s1}{lightblue}\PY{l+s+s1}{\PYZsq{}}\PY{p}{,} \PY{n}{width} \PY{o}{=} \PY{l+m+mi}{1}\PY{p}{,} \PY{n}{edgecolor} \PY{o}{=} \PY{l+s+s2}{\PYZdq{}}\PY{l+s+s2}{black}\PY{l+s+s2}{\PYZdq{}}\PY{p}{,} \PY{n}{align} \PY{o}{=} \PY{l+s+s1}{\PYZsq{}}\PY{l+s+s1}{center}\PY{l+s+s1}{\PYZsq{}}\PY{p}{)}

\PY{c+c1}{\PYZsh{} titulo}
\PY{n}{plt}\PY{o}{.}\PY{n}{title}\PY{p}{(}\PY{l+s+s2}{\PYZdq{}}\PY{l+s+s2}{Distribución de grado del grafo aleatorio}\PY{l+s+s2}{\PYZdq{}}\PY{p}{,} \PY{n}{fontsize} \PY{o}{=} \PY{l+m+mi}{20}\PY{p}{)}

\PY{c+c1}{\PYZsh{} ejes X y Y}
\PY{n}{plt}\PY{o}{.}\PY{n}{ylabel}\PY{p}{(}\PY{l+s+s2}{\PYZdq{}}\PY{l+s+s2}{Número de nodos}\PY{l+s+s2}{\PYZdq{}}\PY{p}{,} \PY{n}{fontsize} \PY{o}{=} \PY{l+m+mi}{16}\PY{p}{)}
\PY{n}{plt}\PY{o}{.}\PY{n}{xlabel}\PY{p}{(}\PY{l+s+s2}{\PYZdq{}}\PY{l+s+s2}{Grado (k)}\PY{l+s+s2}{\PYZdq{}}\PY{p}{,} \PY{n}{fontsize} \PY{o}{=} \PY{l+m+mi}{16}\PY{p}{)}
\PY{n}{plt}\PY{o}{.}\PY{n}{xticks}\PY{p}{(}\PY{n}{fontsize} \PY{o}{=} \PY{l+m+mi}{12}\PY{p}{)}
\PY{n}{plt}\PY{o}{.}\PY{n}{yticks}\PY{p}{(}\PY{n}{fontsize} \PY{o}{=} \PY{l+m+mi}{12}\PY{p}{)}

\PY{c+c1}{\PYZsh{} Dibujamos el grado medio}

\PY{n}{plt}\PY{o}{.}\PY{n}{show}\PY{p}{(}\PY{p}{)}
\end{Verbatim}
\end{tcolorbox}

    \begin{center}
    \adjustimage{max size={0.9\linewidth}{0.9\paperheight}}{output_43_0.png}
    \end{center}
    { \hspace*{\fill} \\}
    
    \begin{tcolorbox}[breakable, size=fbox, boxrule=1pt, pad at break*=1mm,colback=cellbackground, colframe=cellborder]
\prompt{In}{incolor}{19}{\hspace{4pt}}
\begin{Verbatim}[commandchars=\\\{\}]
\PY{n+nb}{print}\PY{p}{(}\PY{n}{nx}\PY{o}{.}\PY{n}{info}\PY{p}{(}\PY{n}{G\PYZus{}CE}\PY{p}{)}\PY{p}{)}
\PY{n+nb}{print}\PY{p}{(}\PY{p}{)}
\PY{n+nb}{print}\PY{p}{(}\PY{n}{nx}\PY{o}{.}\PY{n}{info}\PY{p}{(}\PY{n}{G\PYZus{}AL}\PY{p}{)}\PY{p}{)}
\end{Verbatim}
\end{tcolorbox}

    \begin{Verbatim}[commandchars=\\\{\}]
Name: CaernoElegans
Type: Graph
Number of nodes: 1387
Number of edges: 1648
Average degree:   2.3764

Name: Aleatorio
Type: Graph
Number of nodes: 1387
Number of edges: 1648
Average degree:   2.3764
\end{Verbatim}

    \(\\\)

\textbf{2. ¿Son iguales las gráficas de distribución de grados de ambos
grafos? ¿Qué conclusión sacas de lo anterior?}

Claramente se observa que ambos gráficos no siguen la misma distribución
del grado de sus nodos, aunque ambos poseen el mismo número de nodos y
de ramas y grado medio.

En el caso del grafo CaernoElegans vemos que hay algunos nodos con un
grado alto (por ejemplo anteriormente se comprobó que el nodo con mayor
grado presentaba un grado de 131) y un gran número de nodos con grado
bajo, que en el caso del grafo aleatorio no se observa. Este
comportamiento demuestra que nuestro grafo inicial no se comporta como
un grafo aleatorio y, por tanto, podemos asegurar que no se trata de un
grafo aleatorio.

    \(\\\)

\textbf{3. Dibuja ahora la distribución del grado de los nodos de la red
de interacción de proteínas usando escala logarítmica en ambos ejes
añade para ellos dos líneas de código para cambiar el tipo de escala en
cada eje:}

plt.xscale(``log'', nonposx = `clip')

plt.yscale(``log'', nonposy = `clip')

    \begin{tcolorbox}[breakable, size=fbox, boxrule=1pt, pad at break*=1mm,colback=cellbackground, colframe=cellborder]
\prompt{In}{incolor}{20}{\hspace{4pt}}
\begin{Verbatim}[commandchars=\\\{\}]
\PY{n}{AL} \PY{o}{=} \PY{n}{nx}\PY{o}{.}\PY{n}{degree\PYZus{}histogram}\PY{p}{(}\PY{n}{G\PYZus{}AL}\PY{p}{)}
\PY{n}{CE} \PY{o}{=} \PY{n}{nx}\PY{o}{.}\PY{n}{degree\PYZus{}histogram}\PY{p}{(}\PY{n}{G\PYZus{}CE}\PY{p}{)}

\PY{c+c1}{\PYZsh{} Tamaño de la figura}
\PY{n}{plt}\PY{o}{.}\PY{n}{figure}\PY{p}{(}\PY{n}{figsize} \PY{o}{=} \PY{p}{(}\PY{l+m+mi}{12}\PY{p}{,} \PY{l+m+mi}{8}\PY{p}{)}\PY{p}{)}

\PY{c+c1}{\PYZsh{} Distribución de grado de grafo aleatorio}
\PY{n}{plt}\PY{o}{.}\PY{n}{plot}\PY{p}{(}\PY{n+nb}{range}\PY{p}{(}\PY{n+nb}{len}\PY{p}{(}\PY{n}{AL}\PY{p}{)}\PY{p}{)}\PY{p}{,} \PY{n}{AL}\PY{p}{,} \PY{n}{linestyle}\PY{o}{=}\PY{l+s+s1}{\PYZsq{}}\PY{l+s+s1}{\PYZhy{}\PYZhy{}}\PY{l+s+s1}{\PYZsq{}}\PY{p}{,} \PY{n}{marker}\PY{o}{=}\PY{l+s+s1}{\PYZsq{}}\PY{l+s+s1}{o}\PY{l+s+s1}{\PYZsq{}}\PY{p}{,} \PY{n}{color}\PY{o}{=}\PY{l+s+s1}{\PYZsq{}}\PY{l+s+s1}{tab:orange}\PY{l+s+s1}{\PYZsq{}}\PY{p}{,} \PY{n}{label} \PY{o}{=} \PY{l+s+s2}{\PYZdq{}}\PY{l+s+s2}{Aleatorio}\PY{l+s+s2}{\PYZdq{}}\PY{p}{)}
\PY{c+c1}{\PYZsh{} Distribución de grado de grafo de C. elegans}
\PY{n}{plt}\PY{o}{.}\PY{n}{plot}\PY{p}{(}\PY{n+nb}{range}\PY{p}{(}\PY{n+nb}{len}\PY{p}{(}\PY{n}{CE}\PY{p}{)}\PY{p}{)}\PY{p}{,} \PY{n}{CE}\PY{p}{,} \PY{n}{linestyle}\PY{o}{=}\PY{l+s+s1}{\PYZsq{}}\PY{l+s+s1}{\PYZhy{}\PYZhy{}}\PY{l+s+s1}{\PYZsq{}}\PY{p}{,} \PY{n}{marker}\PY{o}{=}\PY{l+s+s1}{\PYZsq{}}\PY{l+s+s1}{o}\PY{l+s+s1}{\PYZsq{}}\PY{p}{,} \PY{n}{color}\PY{o}{=}\PY{l+s+s1}{\PYZsq{}}\PY{l+s+s1}{tab:blue}\PY{l+s+s1}{\PYZsq{}}\PY{p}{,} \PY{n}{label} \PY{o}{=} \PY{l+s+s2}{\PYZdq{}}\PY{l+s+s2}{CaernoElegans}\PY{l+s+s2}{\PYZdq{}}\PY{p}{)}

\PY{c+c1}{\PYZsh{} Titulo}
\PY{n}{plt}\PY{o}{.}\PY{n}{title}\PY{p}{(}\PY{l+s+s2}{\PYZdq{}}\PY{l+s+s2}{Distribución logarítmica de grado}\PY{l+s+s2}{\PYZdq{}}\PY{p}{,} \PY{n}{fontsize} \PY{o}{=} \PY{l+m+mi}{20}\PY{p}{)}

\PY{c+c1}{\PYZsh{} ejes X y Y}
\PY{n}{plt}\PY{o}{.}\PY{n}{ylabel}\PY{p}{(}\PY{l+s+s2}{\PYZdq{}}\PY{l+s+s2}{Número de nodos}\PY{l+s+s2}{\PYZdq{}}\PY{p}{,} \PY{n}{fontsize} \PY{o}{=} \PY{l+m+mi}{16}\PY{p}{)}
\PY{n}{plt}\PY{o}{.}\PY{n}{xlabel}\PY{p}{(}\PY{l+s+s2}{\PYZdq{}}\PY{l+s+s2}{Grado (k)}\PY{l+s+s2}{\PYZdq{}}\PY{p}{,} \PY{n}{fontsize} \PY{o}{=} \PY{l+m+mi}{16}\PY{p}{)}
\PY{n}{plt}\PY{o}{.}\PY{n}{xscale}\PY{p}{(}\PY{l+s+s2}{\PYZdq{}}\PY{l+s+s2}{log}\PY{l+s+s2}{\PYZdq{}}\PY{p}{,} \PY{n}{nonposx} \PY{o}{=} \PY{l+s+s1}{\PYZsq{}}\PY{l+s+s1}{clip}\PY{l+s+s1}{\PYZsq{}}\PY{p}{)}
\PY{n}{plt}\PY{o}{.}\PY{n}{yscale}\PY{p}{(}\PY{l+s+s2}{\PYZdq{}}\PY{l+s+s2}{log}\PY{l+s+s2}{\PYZdq{}}\PY{p}{,} \PY{n}{nonposy} \PY{o}{=} \PY{l+s+s1}{\PYZsq{}}\PY{l+s+s1}{clip}\PY{l+s+s1}{\PYZsq{}}\PY{p}{)}
\PY{n}{plt}\PY{o}{.}\PY{n}{xticks}\PY{p}{(}\PY{n}{fontsize} \PY{o}{=} \PY{l+m+mi}{14}\PY{p}{)}
\PY{n}{plt}\PY{o}{.}\PY{n}{yticks}\PY{p}{(}\PY{n}{fontsize} \PY{o}{=} \PY{l+m+mi}{14}\PY{p}{)}

\PY{c+c1}{\PYZsh{} parametros de los ticks}
\PY{n}{plt}\PY{o}{.}\PY{n}{tick\PYZus{}params}\PY{p}{(}\PY{n}{axis}\PY{o}{=}\PY{l+s+s2}{\PYZdq{}}\PY{l+s+s2}{both}\PY{l+s+s2}{\PYZdq{}}\PY{p}{,} \PY{n}{which} \PY{o}{=} \PY{l+s+s2}{\PYZdq{}}\PY{l+s+s2}{minor}\PY{l+s+s2}{\PYZdq{}}\PY{p}{,} \PY{n}{direction}\PY{o}{=}\PY{l+s+s2}{\PYZdq{}}\PY{l+s+s2}{out}\PY{l+s+s2}{\PYZdq{}}\PY{p}{,} \PY{n}{length} \PY{o}{=} \PY{l+m+mi}{5}\PY{p}{)}
\PY{n}{plt}\PY{o}{.}\PY{n}{tick\PYZus{}params}\PY{p}{(}\PY{n}{axis}\PY{o}{=}\PY{l+s+s2}{\PYZdq{}}\PY{l+s+s2}{both}\PY{l+s+s2}{\PYZdq{}}\PY{p}{,} \PY{n}{which} \PY{o}{=} \PY{l+s+s2}{\PYZdq{}}\PY{l+s+s2}{major}\PY{l+s+s2}{\PYZdq{}}\PY{p}{,} \PY{n}{direction}\PY{o}{=}\PY{l+s+s2}{\PYZdq{}}\PY{l+s+s2}{out}\PY{l+s+s2}{\PYZdq{}}\PY{p}{,} \PY{n}{length} \PY{o}{=} \PY{l+m+mi}{10}\PY{p}{)}

\PY{c+c1}{\PYZsh{} leyenda}
\PY{n}{plt}\PY{o}{.}\PY{n}{legend}\PY{p}{(}\PY{n}{loc} \PY{o}{=} \PY{l+s+s2}{\PYZdq{}}\PY{l+s+s2}{best}\PY{l+s+s2}{\PYZdq{}}\PY{p}{,} \PY{n}{fontsize} \PY{o}{=} \PY{l+m+mi}{16}\PY{p}{)}

\PY{n}{plt}\PY{o}{.}\PY{n}{show}\PY{p}{(}\PY{p}{)}
\end{Verbatim}
\end{tcolorbox}

    \begin{center}
    \adjustimage{max size={0.9\linewidth}{0.9\paperheight}}{output_47_0.png}
    \end{center}
    { \hspace*{\fill} \\}
    
    \(\\\)

\textbf{4. ¿Qué tipo de gráfica obtienes? ¿Podrías calcular
aproximadamente la pendiente de los datos?}

La gráfica del grafo CaernoElegans tiene una región claramente lineal,
fenómeno que no ocurre en la curva generada para el grafo aleatorio.

La aparición de esta región lineal es característica de una red libre de
escala; donde algunos nodos están altamente conectados, mientras que el
grado de conectividad de casi todos los nodos es bastante bajo. Esto
además concuerda con el hecho de que la pendiente sea negativa. Tomando
de forma aproximada dos puntos de la región lineal de la gráfica
obtenemos:

\[ pendiente = \frac{y_2 -y_1}{x_2 - x_1} = \frac{3-1}{0-0.9} = -2.22\]

Adicionalmente, observamos que tras la región rectilínea, se forma una
especie de cola correspondiente a la desaparición de nodos durante la
formación de la red.

Por el contrario, en el caso de la curva asociada al grafo aleatorio, no
encontramos una región lineal. Por tanto, no es posible calcular su
pendiente. Esta representación curvilínea se corresponde a los grafos
aleatorios, donde la frecuencia de nodos disminuye a medida que aumenta
el número de grado debido a que la probabilidad de que se formen ramas
es igual para todos los nodos. Esta representación curvilínea se asocia
con una distribución binomial con una probabilidad de que se genere una
rama \((p)\) muy baja (en nuestro caso \(p \simeq 0.0017\)), es decir,
una distribución de Poisson.

    \(\\\)

\hypertarget{apartado-4-cuxe1lculo-de-paruxe1metros-de-grafos}{%
\subsection{Apartado 4: Cálculo de parámetros de
grafos}\label{apartado-4-cuxe1lculo-de-paruxe1metros-de-grafos}}

\textbf{Calcula para ambos grafos:}

\hypertarget{a-degree_centralityg}{%
\subsubsection{a) degree\_centrality(G)}\label{a-degree_centralityg}}

Dentro de las medidas de centralidad de un grafo encontramos la
centralidad del grado (``degree centrality''), que corresponde al número
de ramas que posee un nodo con los demás
\href{https://es.wikipedia.org/wiki/Centralidad}{{[}1{]}}. Dado que el
grafo no es un grafo dirigido, no hay que diferenciar entre grados de
salida y de entrada. Por lo tanto, para conocer la centralidad se emplea
la función: \emph{degree\_centrality()} de NetworkX.

Se calcula de la siguiente forma:

\[C_{DEG}^v =  \frac{k_v}{|V|}\]

\[\bar{C}_{DEG}^G = \frac{\sum_{v=1}^{|V|}{C_{DEG}^v}}{|V|}\]

Donde: 
\begin{itemize}
    \item \(C_{DEG}^v\) es la centralidad del grado del nodo \(v\)
    \item \(\bar{C}_{DEG}\) es el promedio de la centralidad del grado del grafo \(G\)
    \item \(k_v\) es el grado del nodo \(v\)
    \item \(|V|\) es el orden del grafo
\end{itemize}

La centralidad del grado nos permite conocer como de conectado se
encuentra un nodo con el resto de nodos. Por ejemplo, en el grafo
correspondiente a la red de \emph{C. elegans} se comprueba que el nodo
con mayor grado calculado previamente es el que presenta mayor valor de
centralidad del grado. De este modo, podemos determinar nodos
importantes de la red y potenciales dianas para atacarla.

Observando el resultado, es lógico que ambos gráficos tengan el mismo
valor promedio de centralidad del grado, porque ambos tienen el mismo
número de nodos y de ramas por lo que, aunque la centralidad del grado
de cada nodo individualmente sea diferente, el promedio será el mismo
(la desviación es \(\simeq 0\) en promedio para los grafos aleatorios
generados). No obstante, sí observamos que el grafo de la red de
proteínas de \emph{C. elegans} no se comporta de manera aleatoria, dado
que el valor máximo de centralidad de grado es mayor frente al máximo
del grafo aleatorio (0.09 \textgreater{} 0.006).

    \begin{tcolorbox}[breakable, size=fbox, boxrule=1pt, pad at break*=1mm,colback=cellbackground, colframe=cellborder]
\prompt{In}{incolor}{21}{\hspace{4pt}}
\begin{Verbatim}[commandchars=\\\{\}]
\PY{c+c1}{\PYZsh{} Grafo inicial}
\PY{n}{dic\PYZus{}centralidad} \PY{o}{=} \PY{n}{nx}\PY{o}{.}\PY{n}{degree\PYZus{}centrality}\PY{p}{(}\PY{n}{G\PYZus{}CE}\PY{p}{)} \PY{c+c1}{\PYZsh{} centralidad calculada para todos los nodos}
\PY{c+c1}{\PYZsh{} print (dic\PYZus{}centralidad)}
\PY{c+c1}{\PYZsh{} Promedio}
\PY{n}{promedio\PYZus{}centralidad} \PY{o}{=} \PY{p}{(}\PY{n+nb}{sum}\PY{p}{(}\PY{n}{dic\PYZus{}centralidad}\PY{o}{.}\PY{n}{values}\PY{p}{(}\PY{p}{)}\PY{p}{)} \PY{o}{/} \PY{n}{orden}\PY{p}{)}

\PY{n+nb}{print}\PY{p}{(}\PY{l+s+s2}{\PYZdq{}}\PY{l+s+s2}{Centralidad de grado en la red de proteínas de C. elegans: }\PY{l+s+s2}{\PYZdq{}}\PY{p}{,} \PY{n}{promedio\PYZus{}centralidad}\PY{p}{)}

\PY{c+c1}{\PYZsh{} Nodo con mayor valor de centralidad de grado}
\PY{n}{mayor} \PY{o}{=} \PY{n+nb}{max}\PY{p}{(}\PY{n}{dic\PYZus{}centralidad}\PY{o}{.}\PY{n}{items}\PY{p}{(}\PY{p}{)}\PY{p}{,} \PY{n}{key} \PY{o}{=} \PY{k}{lambda} \PY{n}{pareja}\PY{p}{:} \PY{n}{pareja}\PY{p}{[}\PY{l+m+mi}{1}\PY{p}{]} \PY{p}{)}
\PY{n+nb}{print}\PY{p}{(}\PY{l+s+s2}{\PYZdq{}}\PY{l+s+s2}{Nodo con mayor centralidad de grado en la red de proteínas de C. elegans: }\PY{l+s+si}{\PYZpc{}s}\PY{l+s+s2}{, (}\PY{l+s+si}{\PYZpc{}s}\PY{l+s+s2}{)}\PY{l+s+s2}{\PYZdq{}} \PY{o}{\PYZpc{}}\PY{p}{(}\PY{n}{mayor}\PY{p}{[}\PY{l+m+mi}{0}\PY{p}{]}\PY{p}{,} \PY{n}{mayor}\PY{p}{[}\PY{l+m+mi}{1}\PY{p}{]}\PY{p}{)}\PY{p}{)}
\end{Verbatim}
\end{tcolorbox}

    \begin{Verbatim}[commandchars=\\\{\}]
Centralidad de grado en la red de proteínas de C. elegans:
0.0017145395660175464
Nodo con mayor centralidad de grado en la red de proteínas de C. elegans:
T08G11.5, (0.09451659451659451)
\end{Verbatim}

    \begin{tcolorbox}[breakable, size=fbox, boxrule=1pt, pad at break*=1mm,colback=cellbackground, colframe=cellborder]
\prompt{In}{incolor}{22}{\hspace{4pt}}
\begin{Verbatim}[commandchars=\\\{\}]
\PY{n+nb}{print}\PY{p}{(}\PY{l+s+s2}{\PYZdq{}}\PY{l+s+s2}{Promedio de la centralidad media de grado en el grafo aleatorio: }\PY{l+s+si}{\PYZpc{}s}\PY{l+s+s2}{ con }\PY{l+s+si}{\PYZpc{}s}\PY{l+s+s2}{ desviación estándar}\PY{l+s+s2}{\PYZdq{}}
      \PY{o}{\PYZpc{}}\PY{p}{(}\PY{n}{dic\PYZus{}params}\PY{p}{[}\PY{l+s+s2}{\PYZdq{}}\PY{l+s+s2}{Degree Centrality}\PY{l+s+s2}{\PYZdq{}}\PY{p}{]}\PY{p}{[}\PY{l+m+mi}{0}\PY{p}{]}\PY{p}{,} \PY{n}{dic\PYZus{}params}\PY{p}{[}\PY{l+s+s2}{\PYZdq{}}\PY{l+s+s2}{Degree Centrality}\PY{l+s+s2}{\PYZdq{}}\PY{p}{]}\PY{p}{[}\PY{l+m+mi}{1}\PY{p}{]}\PY{p}{)}\PY{p}{)}

\PY{n+nb}{print}\PY{p}{(}\PY{p}{)}
\PY{n+nb}{print}\PY{p}{(}\PY{l+s+s2}{\PYZdq{}}\PY{l+s+s2}{Centralidad media del nodo con mayor centralidad de grado en el grafo aleatorio: }\PY{l+s+si}{\PYZpc{}s}\PY{l+s+s2}{, con desviación estándar }\PY{l+s+si}{\PYZpc{}s}\PY{l+s+s2}{\PYZdq{}} 
      \PY{o}{\PYZpc{}}\PY{p}{(}\PY{n}{dic\PYZus{}params}\PY{p}{[}\PY{l+s+s2}{\PYZdq{}}\PY{l+s+s2}{Nodo max Degree Centrality}\PY{l+s+s2}{\PYZdq{}}\PY{p}{]}\PY{p}{[}\PY{l+m+mi}{0}\PY{p}{]}\PY{p}{,} \PY{n}{dic\PYZus{}params}\PY{p}{[}\PY{l+s+s2}{\PYZdq{}}\PY{l+s+s2}{Nodo max Degree Centrality}\PY{l+s+s2}{\PYZdq{}}\PY{p}{]}\PY{p}{[}\PY{l+m+mi}{1}\PY{p}{]}\PY{p}{)}\PY{p}{)}
\end{Verbatim}
\end{tcolorbox}

    \begin{Verbatim}[commandchars=\\\{\}]
Promedio de la centralidad media de grado en el grafo aleatorio:
2.378066378066355 con 1.9100999153570945e-15 desviación estándar

Centralidad media del nodo con mayor centralidad de grado en el grafo aleatorio:
0.006493506493506494, con desviación estándar 0.0006453298636363029
\end{Verbatim}

    \(\\\)

\hypertarget{b-closeness_centralityg-cercanuxeda}{%
\subsubsection{b) closeness\_centrality(G)
(cercanía)}\label{b-closeness_centralityg-cercanuxeda}}

La cercanía (inversamente proporcional a la lejanía) es otra medida de
centralidad basada en el cálculo de la suma o bien el promedio de las
distancias más cortas desde un nodo hacia todos los demás.

    \hypertarget{b.1-cuxe1lculo-de-la-cercanuxeda-en-el-grafo-de-caernoelegans-y-el-grafo-aleatorio}{%
\paragraph{b.1) Cálculo de la cercanía en el grafo de CaernoElegans y el
grafo
aleatorio}\label{b.1-cuxe1lculo-de-la-cercanuxeda-en-el-grafo-de-caernoelegans-y-el-grafo-aleatorio}}

    \begin{tcolorbox}[breakable, size=fbox, boxrule=1pt, pad at break*=1mm,colback=cellbackground, colframe=cellborder]
\prompt{In}{incolor}{23}{\hspace{4pt}}
\begin{Verbatim}[commandchars=\\\{\}]
\PY{c+c1}{\PYZsh{} Grafo inicial}
\PY{n}{dic\PYZus{}cercania} \PY{o}{=} \PY{n}{nx}\PY{o}{.}\PY{n}{closeness\PYZus{}centrality}\PY{p}{(}\PY{n}{G\PYZus{}CE}\PY{p}{)}
\PY{n}{promedio\PYZus{}cercania} \PY{o}{=} \PY{p}{(}\PY{n+nb}{sum}\PY{p}{(}\PY{n}{dic\PYZus{}cercania}\PY{o}{.}\PY{n}{values}\PY{p}{(}\PY{p}{)}\PY{p}{)} \PY{o}{/} \PY{n}{orden}\PY{p}{)}
\PY{n+nb}{print}\PY{p}{(}\PY{l+s+s2}{\PYZdq{}}\PY{l+s+s2}{Cercanía en la red de proteínas de C. elegans: }\PY{l+s+s2}{\PYZdq{}}\PY{p}{,} \PY{n}{promedio\PYZus{}cercania}\PY{p}{)}
\PY{n+nb}{print}\PY{p}{(}\PY{p}{)}
\PY{n}{mayor} \PY{o}{=} \PY{n+nb}{max}\PY{p}{(}\PY{n}{dic\PYZus{}cercania}\PY{o}{.}\PY{n}{items}\PY{p}{(}\PY{p}{)}\PY{p}{,} \PY{n}{key} \PY{o}{=} \PY{k}{lambda} \PY{n}{pareja}\PY{p}{:} \PY{n}{pareja}\PY{p}{[}\PY{l+m+mi}{1}\PY{p}{]} \PY{p}{)}
\PY{n+nb}{print}\PY{p}{(}\PY{l+s+s2}{\PYZdq{}}\PY{l+s+s2}{Nodo con mayor centralidad de grado en la red de proteínas de C. elegans: }\PY{l+s+si}{\PYZpc{}s}\PY{l+s+s2}{, (}\PY{l+s+si}{\PYZpc{}s}\PY{l+s+s2}{)}\PY{l+s+s2}{\PYZdq{}} \PY{o}{\PYZpc{}}\PY{p}{(}\PY{n}{mayor}\PY{p}{[}\PY{l+m+mi}{0}\PY{p}{]}\PY{p}{,} \PY{n}{mayor}\PY{p}{[}\PY{l+m+mi}{1}\PY{p}{]}\PY{p}{)}\PY{p}{)}
\end{Verbatim}
\end{tcolorbox}

    \begin{Verbatim}[commandchars=\\\{\}]
Cercanía en la red de proteínas de C. elegans:  0.07114268033538046

Nodo con mayor centralidad de grado en la red de proteínas de C. elegans:
W10C8.2, (0.13008480872167205)
\end{Verbatim}

    \begin{tcolorbox}[breakable, size=fbox, boxrule=1pt, pad at break*=1mm,colback=cellbackground, colframe=cellborder]
\prompt{In}{incolor}{24}{\hspace{4pt}}
\begin{Verbatim}[commandchars=\\\{\}]
\PY{n+nb}{print}\PY{p}{(}\PY{l+s+s2}{\PYZdq{}}\PY{l+s+s2}{Promedio de la cercanía media de grado en el grafo aleatorio: }\PY{l+s+si}{\PYZpc{}s}\PY{l+s+s2}{, con }\PY{l+s+si}{\PYZpc{}s}\PY{l+s+s2}{ desviación estándar}\PY{l+s+s2}{\PYZdq{}}
      \PY{o}{\PYZpc{}}\PY{p}{(}\PY{n}{dic\PYZus{}params}\PY{p}{[}\PY{l+s+s2}{\PYZdq{}}\PY{l+s+s2}{Cercania}\PY{l+s+s2}{\PYZdq{}}\PY{p}{]}\PY{p}{[}\PY{l+m+mi}{0}\PY{p}{]}\PY{p}{,} \PY{n}{dic\PYZus{}params}\PY{p}{[}\PY{l+s+s2}{\PYZdq{}}\PY{l+s+s2}{Cercania}\PY{l+s+s2}{\PYZdq{}}\PY{p}{]}\PY{p}{[}\PY{l+m+mi}{1}\PY{p}{]}\PY{p}{)}\PY{p}{)}
\PY{n+nb}{print}\PY{p}{(}\PY{p}{)}
\PY{n+nb}{print}\PY{p}{(}\PY{l+s+s2}{\PYZdq{}}\PY{l+s+s2}{Centralidad media del nodo con mayor cercanía en el grafo aleatorio: }\PY{l+s+si}{\PYZpc{}s}\PY{l+s+s2}{, con desviación estándar }\PY{l+s+si}{\PYZpc{}s}\PY{l+s+s2}{\PYZdq{}} 
      \PY{o}{\PYZpc{}}\PY{p}{(}\PY{n}{dic\PYZus{}params}\PY{p}{[}\PY{l+s+s2}{\PYZdq{}}\PY{l+s+s2}{Nodo max cercania}\PY{l+s+s2}{\PYZdq{}}\PY{p}{]}\PY{p}{[}\PY{l+m+mi}{0}\PY{p}{]}\PY{p}{,} \PY{n}{dic\PYZus{}params}\PY{p}{[}\PY{l+s+s2}{\PYZdq{}}\PY{l+s+s2}{Nodo max cercania}\PY{l+s+s2}{\PYZdq{}}\PY{p}{]}\PY{p}{[}\PY{l+m+mi}{1}\PY{p}{]}\PY{p}{)}\PY{p}{)}
\end{Verbatim}
\end{tcolorbox}

    \begin{Verbatim}[commandchars=\\\{\}]
Promedio de la cercanía media de grado en el grafo aleatorio:
135.34516157436457, con 2.3001931602143246 desviación estándar

Centralidad media del nodo con mayor cercanía en el grafo aleatorio:
0.1455585998306277, con desviación estándar 0.0038772866573120023
\end{Verbatim}

    La cercanía se expresa como la inversa de la suma de las distancias de
un grafo. Normalmente se muestra normalizada por el número de nodos
diferentes al nodo de partida (\(N-1\), siendo \(N\) el número total de
nodos)\href{https://www.ebi.ac.uk/training/online/course/network-analysis-protein-interaction-data-introduction/building-and-analysing-ppins-1}{{[}1{]}}:

\[ CC_{i} = \frac{N-1}{\sum_{j} d_{i,j}} \]

Donde: 
\begin{itemize}
    \item \(CC_{i}\) es la cercanía
    \item \(i \neq j\)
    \item \(N\) es el número de nodos
    \item \(d_{i,j}\) es la distancia mínima entre el nodo \(i\) y \(j\)
\end{itemize}

Mientras mayor sea el valor de la cercanía, se puede decir que el nodo
está más «cercano» al resto de nodos de la red. De este modo, la
cercanía de un nodo refleja la accesibilidad de un nodo y la rapidez con
la que la información se propaga en una red
\href{https://www.ebi.ac.uk/training/online/course/network-analysis-protein-interaction-data-introduction/building-and-analysing-ppins-1}{{[}1{]}}\href{https://es.wikipedia.org/wiki/Centralidad}{{[}2{]}}.

Teniendo esto último en consideración y que ambos grafos poseen el mismo
número de nodos y ramas, el hecho de que el grafo aleatorio presente
mayor cercanía que la red de proteínas de \emph{C. elegans} (0.098
\textgreater{} 0.071) indica que las aristas del grafo CaernoElegans no
se disponen del mismo modo que en los grafos aleatorios y, por tanto,
que presenta un sesgo interesante de estudiar. Esto además también se
observó anteriormente cuando se calculó la distribución del grado de los
nodos en ambos grafos. Igualmente, el hecho de que la red de proteínas
de \emph{C. elegans} presente una cercanía menor nos indica que pueden
existir nodos con un grado elevado con respecto al resto, como se ha
comprobado anteriormente.

    \hypertarget{b.2-demostraciuxf3n-de-que-el-nodo-con-mayor-nuxfamero-de-ramas-no-tiene-porquuxe9-ser-el-nodo-con-mayor-cercanuxeda}{%
\paragraph{b.2) Demostración de que el nodo con mayor número de ramas no
tiene porqué ser el nodo con mayor
cercanía}\label{b.2-demostraciuxf3n-de-que-el-nodo-con-mayor-nuxfamero-de-ramas-no-tiene-porquuxe9-ser-el-nodo-con-mayor-cercanuxeda}}

Es interesante comentar que en ambos grafos, el nodo con mayor cercanía
no es el mismo que el nodo que presentaba mayor centralidad del grado.
Esto es posible dado que en la cercanía se consideran las distancias más
cortas desde un nodo al resto y es posible que éste no sea el que mayor
número de ramas posea, como se muestra a continuación:

    \begin{tcolorbox}[breakable, size=fbox, boxrule=1pt, pad at break*=1mm,colback=cellbackground, colframe=cellborder]
\prompt{In}{incolor}{25}{\hspace{4pt}}
\begin{Verbatim}[commandchars=\\\{\}]
\PY{c+c1}{\PYZsh{} Ejemplo de grafo con nodo con mayor grado diferente al de mayor centralidad}
\PY{n}{H} \PY{o}{=} \PY{n}{nx}\PY{o}{.}\PY{n}{Graph}\PY{p}{(}\PY{p}{)}
\PY{n}{H}\PY{o}{.}\PY{n}{add\PYZus{}nodes\PYZus{}from}\PY{p}{(}\PY{p}{[}\PY{l+s+s1}{\PYZsq{}}\PY{l+s+s1}{1}\PY{l+s+s1}{\PYZsq{}}\PY{p}{,} \PY{l+s+s1}{\PYZsq{}}\PY{l+s+s1}{2}\PY{l+s+s1}{\PYZsq{}}\PY{p}{,} \PY{l+s+s1}{\PYZsq{}}\PY{l+s+s1}{3}\PY{l+s+s1}{\PYZsq{}}\PY{p}{,} \PY{l+s+s1}{\PYZsq{}}\PY{l+s+s1}{4}\PY{l+s+s1}{\PYZsq{}}\PY{p}{,} \PY{l+s+s1}{\PYZsq{}}\PY{l+s+s1}{5}\PY{l+s+s1}{\PYZsq{}}\PY{p}{,} \PY{l+s+s1}{\PYZsq{}}\PY{l+s+s1}{6}\PY{l+s+s1}{\PYZsq{}}\PY{p}{,} \PY{l+s+s1}{\PYZsq{}}\PY{l+s+s1}{7}\PY{l+s+s1}{\PYZsq{}}\PY{p}{,} \PY{l+s+s1}{\PYZsq{}}\PY{l+s+s1}{8}\PY{l+s+s1}{\PYZsq{}}\PY{p}{]}\PY{p}{)}
\PY{n}{H}\PY{o}{.}\PY{n}{add\PYZus{}edges\PYZus{}from}\PY{p}{(}\PY{p}{[}\PY{p}{(}\PY{l+s+s1}{\PYZsq{}}\PY{l+s+s1}{1}\PY{l+s+s1}{\PYZsq{}}\PY{p}{,} \PY{l+s+s1}{\PYZsq{}}\PY{l+s+s1}{4}\PY{l+s+s1}{\PYZsq{}}\PY{p}{)}\PY{p}{,} \PY{p}{(}\PY{l+s+s1}{\PYZsq{}}\PY{l+s+s1}{2}\PY{l+s+s1}{\PYZsq{}}\PY{p}{,} \PY{l+s+s1}{\PYZsq{}}\PY{l+s+s1}{4}\PY{l+s+s1}{\PYZsq{}}\PY{p}{)}\PY{p}{,} \PY{p}{(}\PY{l+s+s1}{\PYZsq{}}\PY{l+s+s1}{3}\PY{l+s+s1}{\PYZsq{}}\PY{p}{,} \PY{l+s+s1}{\PYZsq{}}\PY{l+s+s1}{4}\PY{l+s+s1}{\PYZsq{}}\PY{p}{)}\PY{p}{,} \PY{p}{(}\PY{l+s+s1}{\PYZsq{}}\PY{l+s+s1}{4}\PY{l+s+s1}{\PYZsq{}}\PY{p}{,} \PY{l+s+s1}{\PYZsq{}}\PY{l+s+s1}{5}\PY{l+s+s1}{\PYZsq{}}\PY{p}{)}\PY{p}{,} \PY{p}{(}\PY{l+s+s1}{\PYZsq{}}\PY{l+s+s1}{5}\PY{l+s+s1}{\PYZsq{}}\PY{p}{,} \PY{l+s+s1}{\PYZsq{}}\PY{l+s+s1}{6}\PY{l+s+s1}{\PYZsq{}}\PY{p}{)}\PY{p}{,} \PY{p}{(}\PY{l+s+s1}{\PYZsq{}}\PY{l+s+s1}{6}\PY{l+s+s1}{\PYZsq{}}\PY{p}{,} \PY{l+s+s1}{\PYZsq{}}\PY{l+s+s1}{7}\PY{l+s+s1}{\PYZsq{}}\PY{p}{)}\PY{p}{,} \PY{p}{(}\PY{l+s+s1}{\PYZsq{}}\PY{l+s+s1}{6}\PY{l+s+s1}{\PYZsq{}}\PY{p}{,} \PY{l+s+s1}{\PYZsq{}}\PY{l+s+s1}{8}\PY{l+s+s1}{\PYZsq{}}\PY{p}{)}\PY{p}{]}\PY{p}{)}

\PY{n}{posiciones\PYZus{}nodos} \PY{o}{=} \PY{p}{\PYZob{}}\PY{l+s+s1}{\PYZsq{}}\PY{l+s+s1}{1}\PY{l+s+s1}{\PYZsq{}}\PY{p}{:} \PY{p}{[}\PY{l+m+mf}{0.26081287}\PY{p}{,} \PY{l+m+mf}{0.8751262} \PY{p}{]}\PY{p}{,} 
 \PY{l+s+s1}{\PYZsq{}}\PY{l+s+s1}{2}\PY{l+s+s1}{\PYZsq{}}\PY{p}{:} \PY{p}{[}\PY{o}{\PYZhy{}}\PY{l+m+mf}{0.18917212}\PY{p}{,}  \PY{l+m+mf}{0.7198457} \PY{p}{]}\PY{p}{,}
 \PY{l+s+s1}{\PYZsq{}}\PY{l+s+s1}{3}\PY{l+s+s1}{\PYZsq{}}\PY{p}{:} \PY{p}{[}\PY{l+m+mf}{0.55419382}\PY{p}{,} \PY{l+m+mf}{0.4984472} \PY{p}{]}\PY{p}{,} 
 \PY{l+s+s1}{\PYZsq{}}\PY{l+s+s1}{4}\PY{l+s+s1}{\PYZsq{}}\PY{p}{:} \PY{p}{[}\PY{l+m+mf}{0.13556184}\PY{p}{,} \PY{l+m+mf}{0.45475622}\PY{p}{]}\PY{p}{,} 
 \PY{l+s+s1}{\PYZsq{}}\PY{l+s+s1}{5}\PY{l+s+s1}{\PYZsq{}}\PY{p}{:} \PY{p}{[}\PY{o}{\PYZhy{}}\PY{l+m+mf}{0.02473302}\PY{p}{,} \PY{o}{\PYZhy{}}\PY{l+m+mf}{0.0874654} \PY{p}{]}\PY{p}{,} 
 \PY{l+s+s1}{\PYZsq{}}\PY{l+s+s1}{6}\PY{l+s+s1}{\PYZsq{}}\PY{p}{:} \PY{p}{[}\PY{o}{\PYZhy{}}\PY{l+m+mf}{0.18162813}\PY{p}{,} \PY{o}{\PYZhy{}}\PY{l+m+mf}{0.61044471}\PY{p}{]}\PY{p}{,}
 \PY{l+s+s1}{\PYZsq{}}\PY{l+s+s1}{7}\PY{l+s+s1}{\PYZsq{}}\PY{p}{:} \PY{p}{[}\PY{o}{\PYZhy{}}\PY{l+m+mf}{0.03313124}\PY{p}{,} \PY{o}{\PYZhy{}}\PY{l+m+mf}{1.}        \PY{p}{]}\PY{p}{,}
 \PY{l+s+s1}{\PYZsq{}}\PY{l+s+s1}{8}\PY{l+s+s1}{\PYZsq{}}\PY{p}{:} \PY{p}{[}\PY{o}{\PYZhy{}}\PY{l+m+mf}{0.52190402}\PY{p}{,} \PY{o}{\PYZhy{}}\PY{l+m+mf}{0.85026521}\PY{p}{]}\PY{p}{\PYZcb{}}

\PY{n}{posiciones\PYZus{}labels} \PY{o}{=} \PY{p}{\PYZob{}}\PY{l+s+s1}{\PYZsq{}}\PY{l+s+s1}{1}\PY{l+s+s1}{\PYZsq{}}\PY{p}{:} \PY{p}{[}\PY{l+m+mf}{0.26081287}\PY{p}{,} \PY{l+m+mf}{0.8751262} \PY{p}{]}\PY{p}{,} 
 \PY{l+s+s1}{\PYZsq{}}\PY{l+s+s1}{2}\PY{l+s+s1}{\PYZsq{}}\PY{p}{:} \PY{p}{[}\PY{o}{\PYZhy{}}\PY{l+m+mf}{0.18917212}\PY{p}{,}  \PY{l+m+mf}{0.7198457} \PY{p}{]}\PY{p}{,}
 \PY{l+s+s1}{\PYZsq{}}\PY{l+s+s1}{3}\PY{l+s+s1}{\PYZsq{}}\PY{p}{:} \PY{p}{[}\PY{l+m+mf}{0.55419382}\PY{p}{,} \PY{l+m+mf}{0.4984472} \PY{p}{]}\PY{p}{,} 
 \PY{l+s+s1}{\PYZsq{}}\PY{l+s+s1}{4}\PY{l+s+s1}{\PYZsq{}}\PY{p}{:} \PY{p}{[}\PY{l+m+mf}{0.25556184}\PY{p}{,} \PY{l+m+mf}{0.28475622}\PY{p}{]}\PY{p}{,} 
 \PY{l+s+s1}{\PYZsq{}}\PY{l+s+s1}{5}\PY{l+s+s1}{\PYZsq{}}\PY{p}{:} \PY{p}{[} \PY{l+m+mf}{0.18473302}\PY{p}{,} \PY{o}{\PYZhy{}}\PY{l+m+mf}{0.2074654} \PY{p}{]}\PY{p}{,} 
 \PY{l+s+s1}{\PYZsq{}}\PY{l+s+s1}{6}\PY{l+s+s1}{\PYZsq{}}\PY{p}{:} \PY{p}{[}\PY{o}{\PYZhy{}}\PY{l+m+mf}{0.18162813}\PY{p}{,} \PY{o}{\PYZhy{}}\PY{l+m+mf}{0.58044471}\PY{p}{]}\PY{p}{,}
 \PY{l+s+s1}{\PYZsq{}}\PY{l+s+s1}{7}\PY{l+s+s1}{\PYZsq{}}\PY{p}{:} \PY{p}{[}\PY{o}{\PYZhy{}}\PY{l+m+mf}{0.03313124}\PY{p}{,} \PY{o}{\PYZhy{}}\PY{l+m+mf}{1.}        \PY{p}{]}\PY{p}{,}
 \PY{l+s+s1}{\PYZsq{}}\PY{l+s+s1}{8}\PY{l+s+s1}{\PYZsq{}}\PY{p}{:} \PY{p}{[}\PY{o}{\PYZhy{}}\PY{l+m+mf}{0.52190402}\PY{p}{,} \PY{o}{\PYZhy{}}\PY{l+m+mf}{0.85026521}\PY{p}{]}\PY{p}{\PYZcb{}}


\PY{n}{nx}\PY{o}{.}\PY{n}{draw}\PY{p}{(}\PY{n}{H}\PY{p}{,} \PY{n}{with\PYZus{}labels} \PY{o}{=} \PY{k+kc}{True}\PY{p}{,} \PY{n}{pos} \PY{o}{=} \PY{n}{posiciones\PYZus{}nodos}\PY{p}{)}
\PY{n}{labels}\PY{o}{=}\PY{n}{nx}\PY{o}{.}\PY{n}{draw\PYZus{}networkx\PYZus{}labels}\PY{p}{(}\PY{n}{H}\PY{p}{,} \PY{n}{pos} \PY{o}{=} \PY{n}{posiciones\PYZus{}labels}\PY{p}{,} 
                               \PY{n}{labels} \PY{o}{=} \PY{p}{\PYZob{}}\PY{l+s+s2}{\PYZdq{}}\PY{l+s+s2}{5}\PY{l+s+s2}{\PYZdq{}}\PY{p}{:} \PY{l+s+s2}{\PYZdq{}}\PY{l+s+s2}{Mayor cercanía}\PY{l+s+s2}{\PYZdq{}}\PY{p}{,} \PY{l+s+s2}{\PYZdq{}}\PY{l+s+s2}{4}\PY{l+s+s2}{\PYZdq{}}\PY{p}{:} \PY{l+s+s2}{\PYZdq{}}\PY{l+s+s2}{Mayor grado}\PY{l+s+s2}{\PYZdq{}}\PY{p}{\PYZcb{}}\PY{p}{,}
                               \PY{n}{font\PYZus{}size} \PY{o}{=} \PY{l+m+mi}{14}\PY{p}{,} 
                               \PY{n}{font\PYZus{}color} \PY{o}{=} \PY{l+s+s1}{\PYZsq{}}\PY{l+s+s1}{green}\PY{l+s+s1}{\PYZsq{}}\PY{p}{)}

\PY{n}{dic\PYZus{}cercania} \PY{o}{=} \PY{n}{nx}\PY{o}{.}\PY{n}{closeness\PYZus{}centrality}\PY{p}{(}\PY{n}{H}\PY{p}{)}
\PY{n}{promedio\PYZus{}cercania} \PY{o}{=} \PY{p}{(}\PY{n+nb}{sum}\PY{p}{(}\PY{n}{dic\PYZus{}cercania}\PY{o}{.}\PY{n}{values}\PY{p}{(}\PY{p}{)}\PY{p}{)} \PY{o}{/} \PY{n}{orden}\PY{p}{)}
\PY{n+nb}{print}\PY{p}{(}\PY{l+s+s2}{\PYZdq{}}\PY{l+s+s2}{Promedio de la cercanía : }\PY{l+s+s2}{\PYZdq{}}\PY{p}{,} \PY{n+nb}{round}\PY{p}{(}\PY{n}{promedio\PYZus{}cercania}\PY{p}{,} \PY{l+m+mi}{5}\PY{p}{)}\PY{p}{)}
\end{Verbatim}
\end{tcolorbox}

    \begin{Verbatim}[commandchars=\\\{\}]
Promedio de la cercanía :  0.00255
\end{Verbatim}

    \begin{center}
    \adjustimage{max size={0.9\linewidth}{0.9\paperheight}}{output_58_1.png}
    \end{center}
    { \hspace*{\fill} \\}
    
    \hypertarget{b.3-demostraciuxf3n-de-que-la-cercanuxeda-depende-de-la-distribuciuxf3n-de-las-ramas-en-el}{%
\paragraph{b.3) Demostración de que la cercanía depende de la
distribución de las ramas en
el}\label{b.3-demostraciuxf3n-de-que-la-cercanuxeda-depende-de-la-distribuciuxf3n-de-las-ramas-en-el}}

Igualmente, se observa que la cercanía media de los dos grafos no es la
misma. Esto se debe a que en el cálculo de este parámetro de
centralidad, la distribución de las ramas afecta al resultado final. Por
ello, a pesar de que ambos grafos presentan el mismo número de nodos y
ramas, no presentan la misma distribución, lo cual afecta a los caminos
mínimos que se puedan formar. Este efecto se demuestra en los dos
siguientes grafos, donde la centralidad promedio cambia de \(0.00255\) a
\(0.00259\) tan solo cambiando una rama (manteniéndose el número de
nodos y ramas).

    \begin{tcolorbox}[breakable, size=fbox, boxrule=1pt, pad at break*=1mm,colback=cellbackground, colframe=cellborder]
\prompt{In}{incolor}{26}{\hspace{4pt}}
\begin{Verbatim}[commandchars=\\\{\}]
\PY{n}{H} \PY{o}{=} \PY{n}{nx}\PY{o}{.}\PY{n}{Graph}\PY{p}{(}\PY{p}{)}
\PY{n}{H}\PY{o}{.}\PY{n}{add\PYZus{}nodes\PYZus{}from}\PY{p}{(}\PY{p}{[}\PY{l+s+s1}{\PYZsq{}}\PY{l+s+s1}{1}\PY{l+s+s1}{\PYZsq{}}\PY{p}{,} \PY{l+s+s1}{\PYZsq{}}\PY{l+s+s1}{2}\PY{l+s+s1}{\PYZsq{}}\PY{p}{,} \PY{l+s+s1}{\PYZsq{}}\PY{l+s+s1}{3}\PY{l+s+s1}{\PYZsq{}}\PY{p}{,} \PY{l+s+s1}{\PYZsq{}}\PY{l+s+s1}{4}\PY{l+s+s1}{\PYZsq{}}\PY{p}{,} \PY{l+s+s1}{\PYZsq{}}\PY{l+s+s1}{5}\PY{l+s+s1}{\PYZsq{}}\PY{p}{,} \PY{l+s+s1}{\PYZsq{}}\PY{l+s+s1}{6}\PY{l+s+s1}{\PYZsq{}}\PY{p}{,} \PY{l+s+s1}{\PYZsq{}}\PY{l+s+s1}{7}\PY{l+s+s1}{\PYZsq{}}\PY{p}{,} \PY{l+s+s1}{\PYZsq{}}\PY{l+s+s1}{8}\PY{l+s+s1}{\PYZsq{}}\PY{p}{]}\PY{p}{)}
\PY{n}{H}\PY{o}{.}\PY{n}{add\PYZus{}edges\PYZus{}from}\PY{p}{(}\PY{p}{[}\PY{p}{(}\PY{l+s+s1}{\PYZsq{}}\PY{l+s+s1}{1}\PY{l+s+s1}{\PYZsq{}}\PY{p}{,} \PY{l+s+s1}{\PYZsq{}}\PY{l+s+s1}{4}\PY{l+s+s1}{\PYZsq{}}\PY{p}{)}\PY{p}{,} \PY{p}{(}\PY{l+s+s1}{\PYZsq{}}\PY{l+s+s1}{2}\PY{l+s+s1}{\PYZsq{}}\PY{p}{,} \PY{l+s+s1}{\PYZsq{}}\PY{l+s+s1}{4}\PY{l+s+s1}{\PYZsq{}}\PY{p}{)}\PY{p}{,} \PY{p}{(}\PY{l+s+s1}{\PYZsq{}}\PY{l+s+s1}{3}\PY{l+s+s1}{\PYZsq{}}\PY{p}{,} \PY{l+s+s1}{\PYZsq{}}\PY{l+s+s1}{4}\PY{l+s+s1}{\PYZsq{}}\PY{p}{)}\PY{p}{,} \PY{p}{(}\PY{l+s+s1}{\PYZsq{}}\PY{l+s+s1}{4}\PY{l+s+s1}{\PYZsq{}}\PY{p}{,} \PY{l+s+s1}{\PYZsq{}}\PY{l+s+s1}{5}\PY{l+s+s1}{\PYZsq{}}\PY{p}{)}\PY{p}{,} \PY{p}{(}\PY{l+s+s1}{\PYZsq{}}\PY{l+s+s1}{5}\PY{l+s+s1}{\PYZsq{}}\PY{p}{,} \PY{l+s+s1}{\PYZsq{}}\PY{l+s+s1}{6}\PY{l+s+s1}{\PYZsq{}}\PY{p}{)}\PY{p}{,} \PY{p}{(}\PY{l+s+s1}{\PYZsq{}}\PY{l+s+s1}{6}\PY{l+s+s1}{\PYZsq{}}\PY{p}{,} \PY{l+s+s1}{\PYZsq{}}\PY{l+s+s1}{7}\PY{l+s+s1}{\PYZsq{}}\PY{p}{)}\PY{p}{,} \PY{p}{(}\PY{l+s+s1}{\PYZsq{}}\PY{l+s+s1}{6}\PY{l+s+s1}{\PYZsq{}}\PY{p}{,} \PY{l+s+s1}{\PYZsq{}}\PY{l+s+s1}{8}\PY{l+s+s1}{\PYZsq{}}\PY{p}{)}\PY{p}{]}\PY{p}{)}

\PY{n}{posiciones\PYZus{}nodos} \PY{o}{=} \PY{p}{\PYZob{}}\PY{l+s+s1}{\PYZsq{}}\PY{l+s+s1}{1}\PY{l+s+s1}{\PYZsq{}}\PY{p}{:} \PY{p}{[}\PY{l+m+mf}{0.26081287}\PY{p}{,} \PY{l+m+mf}{0.8751262} \PY{p}{]}\PY{p}{,} 
 \PY{l+s+s1}{\PYZsq{}}\PY{l+s+s1}{2}\PY{l+s+s1}{\PYZsq{}}\PY{p}{:} \PY{p}{[}\PY{o}{\PYZhy{}}\PY{l+m+mf}{0.18917212}\PY{p}{,}  \PY{l+m+mf}{0.7198457} \PY{p}{]}\PY{p}{,}
 \PY{l+s+s1}{\PYZsq{}}\PY{l+s+s1}{3}\PY{l+s+s1}{\PYZsq{}}\PY{p}{:} \PY{p}{[}\PY{l+m+mf}{0.55419382}\PY{p}{,} \PY{l+m+mf}{0.4984472} \PY{p}{]}\PY{p}{,} 
 \PY{l+s+s1}{\PYZsq{}}\PY{l+s+s1}{4}\PY{l+s+s1}{\PYZsq{}}\PY{p}{:} \PY{p}{[}\PY{l+m+mf}{0.13556184}\PY{p}{,} \PY{l+m+mf}{0.45475622}\PY{p}{]}\PY{p}{,} 
 \PY{l+s+s1}{\PYZsq{}}\PY{l+s+s1}{5}\PY{l+s+s1}{\PYZsq{}}\PY{p}{:} \PY{p}{[}\PY{o}{\PYZhy{}}\PY{l+m+mf}{0.02473302}\PY{p}{,} \PY{o}{\PYZhy{}}\PY{l+m+mf}{0.0874654} \PY{p}{]}\PY{p}{,} 
 \PY{l+s+s1}{\PYZsq{}}\PY{l+s+s1}{6}\PY{l+s+s1}{\PYZsq{}}\PY{p}{:} \PY{p}{[}\PY{o}{\PYZhy{}}\PY{l+m+mf}{0.18162813}\PY{p}{,} \PY{o}{\PYZhy{}}\PY{l+m+mf}{0.61044471}\PY{p}{]}\PY{p}{,}
 \PY{l+s+s1}{\PYZsq{}}\PY{l+s+s1}{7}\PY{l+s+s1}{\PYZsq{}}\PY{p}{:} \PY{p}{[}\PY{o}{\PYZhy{}}\PY{l+m+mf}{0.03313124}\PY{p}{,} \PY{o}{\PYZhy{}}\PY{l+m+mf}{1.}        \PY{p}{]}\PY{p}{,}
 \PY{l+s+s1}{\PYZsq{}}\PY{l+s+s1}{8}\PY{l+s+s1}{\PYZsq{}}\PY{p}{:} \PY{p}{[}\PY{o}{\PYZhy{}}\PY{l+m+mf}{0.52190402}\PY{p}{,} \PY{o}{\PYZhy{}}\PY{l+m+mf}{0.85026521}\PY{p}{]}\PY{p}{\PYZcb{}}

\PY{n}{nx}\PY{o}{.}\PY{n}{draw}\PY{p}{(}\PY{n}{H}\PY{p}{,} \PY{n}{with\PYZus{}labels} \PY{o}{=} \PY{k+kc}{True}\PY{p}{,} \PY{n}{pos} \PY{o}{=} \PY{n}{posiciones\PYZus{}nodos}\PY{p}{)}

\PY{n}{dic\PYZus{}cercania} \PY{o}{=} \PY{n}{nx}\PY{o}{.}\PY{n}{closeness\PYZus{}centrality}\PY{p}{(}\PY{n}{H}\PY{p}{)}
\PY{n}{promedio\PYZus{}cercania} \PY{o}{=} \PY{p}{(}\PY{n+nb}{sum}\PY{p}{(}\PY{n}{dic\PYZus{}cercania}\PY{o}{.}\PY{n}{values}\PY{p}{(}\PY{p}{)}\PY{p}{)} \PY{o}{/} \PY{n}{orden}\PY{p}{)}
\PY{n+nb}{print}\PY{p}{(}\PY{l+s+s2}{\PYZdq{}}\PY{l+s+s2}{Promedio de la cercanía : }\PY{l+s+s2}{\PYZdq{}}\PY{p}{,} \PY{n+nb}{round}\PY{p}{(}\PY{n}{promedio\PYZus{}cercania}\PY{p}{,} \PY{l+m+mi}{5}\PY{p}{)}\PY{p}{)}
\end{Verbatim}
\end{tcolorbox}

    \begin{Verbatim}[commandchars=\\\{\}]
Promedio de la cercanía :  0.00255
\end{Verbatim}

    \begin{center}
    \adjustimage{max size={0.9\linewidth}{0.9\paperheight}}{output_60_1.png}
    \end{center}
    { \hspace*{\fill} \\}
    
    \begin{tcolorbox}[breakable, size=fbox, boxrule=1pt, pad at break*=1mm,colback=cellbackground, colframe=cellborder]
\prompt{In}{incolor}{27}{\hspace{4pt}}
\begin{Verbatim}[commandchars=\\\{\}]
\PY{c+c1}{\PYZsh{} Ejemplo de grafo con nodo con mayor grado diferente al de mayor centralidad}
\PY{n}{F} \PY{o}{=} \PY{n}{nx}\PY{o}{.}\PY{n}{Graph}\PY{p}{(}\PY{p}{)}
\PY{n}{F}\PY{o}{.}\PY{n}{add\PYZus{}nodes\PYZus{}from}\PY{p}{(}\PY{p}{[}\PY{l+s+s1}{\PYZsq{}}\PY{l+s+s1}{1}\PY{l+s+s1}{\PYZsq{}}\PY{p}{,} \PY{l+s+s1}{\PYZsq{}}\PY{l+s+s1}{2}\PY{l+s+s1}{\PYZsq{}}\PY{p}{,} \PY{l+s+s1}{\PYZsq{}}\PY{l+s+s1}{3}\PY{l+s+s1}{\PYZsq{}}\PY{p}{,} \PY{l+s+s1}{\PYZsq{}}\PY{l+s+s1}{4}\PY{l+s+s1}{\PYZsq{}}\PY{p}{,} \PY{l+s+s1}{\PYZsq{}}\PY{l+s+s1}{5}\PY{l+s+s1}{\PYZsq{}}\PY{p}{,} \PY{l+s+s1}{\PYZsq{}}\PY{l+s+s1}{6}\PY{l+s+s1}{\PYZsq{}}\PY{p}{,} \PY{l+s+s1}{\PYZsq{}}\PY{l+s+s1}{7}\PY{l+s+s1}{\PYZsq{}}\PY{p}{,} \PY{l+s+s1}{\PYZsq{}}\PY{l+s+s1}{8}\PY{l+s+s1}{\PYZsq{}}\PY{p}{]}\PY{p}{)}
\PY{n}{F}\PY{o}{.}\PY{n}{add\PYZus{}edges\PYZus{}from}\PY{p}{(}\PY{p}{[}\PY{p}{(}\PY{l+s+s1}{\PYZsq{}}\PY{l+s+s1}{1}\PY{l+s+s1}{\PYZsq{}}\PY{p}{,} \PY{l+s+s1}{\PYZsq{}}\PY{l+s+s1}{4}\PY{l+s+s1}{\PYZsq{}}\PY{p}{)}\PY{p}{,} \PY{p}{(}\PY{l+s+s1}{\PYZsq{}}\PY{l+s+s1}{2}\PY{l+s+s1}{\PYZsq{}}\PY{p}{,} \PY{l+s+s1}{\PYZsq{}}\PY{l+s+s1}{4}\PY{l+s+s1}{\PYZsq{}}\PY{p}{)}\PY{p}{,} \PY{p}{(}\PY{l+s+s1}{\PYZsq{}}\PY{l+s+s1}{3}\PY{l+s+s1}{\PYZsq{}}\PY{p}{,} \PY{l+s+s1}{\PYZsq{}}\PY{l+s+s1}{5}\PY{l+s+s1}{\PYZsq{}}\PY{p}{)}\PY{p}{,} \PY{p}{(}\PY{l+s+s1}{\PYZsq{}}\PY{l+s+s1}{4}\PY{l+s+s1}{\PYZsq{}}\PY{p}{,} \PY{l+s+s1}{\PYZsq{}}\PY{l+s+s1}{5}\PY{l+s+s1}{\PYZsq{}}\PY{p}{)}\PY{p}{,} \PY{p}{(}\PY{l+s+s1}{\PYZsq{}}\PY{l+s+s1}{5}\PY{l+s+s1}{\PYZsq{}}\PY{p}{,} \PY{l+s+s1}{\PYZsq{}}\PY{l+s+s1}{6}\PY{l+s+s1}{\PYZsq{}}\PY{p}{)}\PY{p}{,} \PY{p}{(}\PY{l+s+s1}{\PYZsq{}}\PY{l+s+s1}{6}\PY{l+s+s1}{\PYZsq{}}\PY{p}{,} \PY{l+s+s1}{\PYZsq{}}\PY{l+s+s1}{7}\PY{l+s+s1}{\PYZsq{}}\PY{p}{)}\PY{p}{,} \PY{p}{(}\PY{l+s+s1}{\PYZsq{}}\PY{l+s+s1}{6}\PY{l+s+s1}{\PYZsq{}}\PY{p}{,} \PY{l+s+s1}{\PYZsq{}}\PY{l+s+s1}{8}\PY{l+s+s1}{\PYZsq{}}\PY{p}{)}\PY{p}{]}\PY{p}{)}

\PY{n}{posiciones\PYZus{}nodos} \PY{o}{=} \PY{p}{\PYZob{}}\PY{l+s+s1}{\PYZsq{}}\PY{l+s+s1}{1}\PY{l+s+s1}{\PYZsq{}}\PY{p}{:} \PY{p}{[}\PY{l+m+mf}{0.26081287}\PY{p}{,} \PY{l+m+mf}{0.8751262} \PY{p}{]}\PY{p}{,} 
 \PY{l+s+s1}{\PYZsq{}}\PY{l+s+s1}{2}\PY{l+s+s1}{\PYZsq{}}\PY{p}{:} \PY{p}{[}\PY{o}{\PYZhy{}}\PY{l+m+mf}{0.18917212}\PY{p}{,}  \PY{l+m+mf}{0.7198457} \PY{p}{]}\PY{p}{,}
 \PY{l+s+s1}{\PYZsq{}}\PY{l+s+s1}{3}\PY{l+s+s1}{\PYZsq{}}\PY{p}{:} \PY{p}{[}\PY{l+m+mf}{0.15419382}\PY{p}{,} \PY{o}{\PYZhy{}}\PY{l+m+mf}{0.2984472} \PY{p}{]}\PY{p}{,} 
 \PY{l+s+s1}{\PYZsq{}}\PY{l+s+s1}{4}\PY{l+s+s1}{\PYZsq{}}\PY{p}{:} \PY{p}{[}\PY{l+m+mf}{0.13556184}\PY{p}{,} \PY{l+m+mf}{0.45475622}\PY{p}{]}\PY{p}{,} 
 \PY{l+s+s1}{\PYZsq{}}\PY{l+s+s1}{5}\PY{l+s+s1}{\PYZsq{}}\PY{p}{:} \PY{p}{[}\PY{o}{\PYZhy{}}\PY{l+m+mf}{0.02473302}\PY{p}{,} \PY{o}{\PYZhy{}}\PY{l+m+mf}{0.0874654} \PY{p}{]}\PY{p}{,} 
 \PY{l+s+s1}{\PYZsq{}}\PY{l+s+s1}{6}\PY{l+s+s1}{\PYZsq{}}\PY{p}{:} \PY{p}{[}\PY{o}{\PYZhy{}}\PY{l+m+mf}{0.18162813}\PY{p}{,} \PY{o}{\PYZhy{}}\PY{l+m+mf}{0.61044471}\PY{p}{]}\PY{p}{,}
 \PY{l+s+s1}{\PYZsq{}}\PY{l+s+s1}{7}\PY{l+s+s1}{\PYZsq{}}\PY{p}{:} \PY{p}{[}\PY{o}{\PYZhy{}}\PY{l+m+mf}{0.03313124}\PY{p}{,} \PY{o}{\PYZhy{}}\PY{l+m+mf}{1.}        \PY{p}{]}\PY{p}{,}
 \PY{l+s+s1}{\PYZsq{}}\PY{l+s+s1}{8}\PY{l+s+s1}{\PYZsq{}}\PY{p}{:} \PY{p}{[}\PY{o}{\PYZhy{}}\PY{l+m+mf}{0.52190402}\PY{p}{,} \PY{o}{\PYZhy{}}\PY{l+m+mf}{0.85026521}\PY{p}{]}\PY{p}{\PYZcb{}}

\PY{n}{nx}\PY{o}{.}\PY{n}{draw}\PY{p}{(}\PY{n}{F}\PY{p}{,} \PY{n}{with\PYZus{}labels} \PY{o}{=} \PY{k+kc}{True}\PY{p}{,} \PY{n}{pos} \PY{o}{=} \PY{n}{posiciones\PYZus{}nodos}\PY{p}{)}

\PY{n}{dic\PYZus{}cercania} \PY{o}{=} \PY{n}{nx}\PY{o}{.}\PY{n}{closeness\PYZus{}centrality}\PY{p}{(}\PY{n}{F}\PY{p}{)}
\PY{n}{promedio\PYZus{}cercania} \PY{o}{=} \PY{p}{(}\PY{n+nb}{sum}\PY{p}{(}\PY{n}{dic\PYZus{}cercania}\PY{o}{.}\PY{n}{values}\PY{p}{(}\PY{p}{)}\PY{p}{)} \PY{o}{/} \PY{n}{orden}\PY{p}{)}
\PY{n+nb}{print}\PY{p}{(}\PY{l+s+s2}{\PYZdq{}}\PY{l+s+s2}{Promedio de la cercanía: }\PY{l+s+s2}{\PYZdq{}}\PY{p}{,} \PY{n+nb}{round}\PY{p}{(}\PY{n}{promedio\PYZus{}cercania}\PY{p}{,} \PY{l+m+mi}{5}\PY{p}{)}\PY{p}{)}
\end{Verbatim}
\end{tcolorbox}

    \begin{Verbatim}[commandchars=\\\{\}]
Promedio de la cercanía:  0.00259
\end{Verbatim}

    \begin{center}
    \adjustimage{max size={0.9\linewidth}{0.9\paperheight}}{output_61_1.png}
    \end{center}
    { \hspace*{\fill} \\}
    
    Como conclusión podemos decir que el nodo con mayor cercanía es otra
diana susceptible de ser atacada en una red, puesto que ayuda a
determinar el impacto que éste causa dentro del conjunto del que forma
parte \href{https://www.grapheverywhere.com/centralidad/}{{[}3{]}}.

    \(\\\)

\hypertarget{c-beteewnness_centrality}{%
\subsubsection{c)
beteewnness\_centrality()}\label{c-beteewnness_centrality}}

El beteewness de un nodo (\(v\)) se define como la fracción de caminos
mínimos que hay entre el resto de nodos del grafo y que pasan por el
nodo \(v\). En otras palabras, el betweeness es una medida que
cuantifica la frecuencia o el número de veces que un nodo actúa como un
puente a lo largo del camino más corto entre otros dos nodos
\href{https://es.wikipedia.org/wiki/Centralidad}{{[}1{]}}.

Los nodos con una alto betweeness suelen jugar un rol crítico en la
estructura de la red. Esto se debe a que participan como reguladores del
flujo de información en los procesos de difusión e integración de la
información. De este modo, los nodos con un alto betweeness también son
dianas susceptibles de ser atacadas en una red.

Aquí las ramas y nodos son los mismos pero la distribución de las ramas
sí afecta al cálculo, ya que influye en los caminos mínimos que se
puedan formar y a la topología de la red.

La red biológica es algo diferente a la aleatoria, siendo mayor el
betweenness de la aleatoria que la de la red de proteínas. Tiene sentido
que sean diferentes, dado que la red de proteínas presenta un sesgo en
la distribución de sus ramas por lo que el número de caminos mínimos que
pasan por un nodo individual puede ser alto, pero de media, el número de
caminos mínimos que pasan por un punto será menor dado que hay
clústeres, de modo que la conexión entre los clústeres tiene poco
intermediarios. Mientras que en uno aleatorio los caminos mínimos que
incorporen otros nodos intermediarios serán más abundantes dada la
distribución homogénea de las ramas.

    \begin{tcolorbox}[breakable, size=fbox, boxrule=1pt, pad at break*=1mm,colback=cellbackground, colframe=cellborder]
\prompt{In}{incolor}{28}{\hspace{4pt}}
\begin{Verbatim}[commandchars=\\\{\}]
\PY{c+c1}{\PYZsh{} Grafo inicial}
\PY{n}{dic\PYZus{}bet} \PY{o}{=} \PY{n}{nx}\PY{o}{.}\PY{n}{betweenness\PYZus{}centrality}\PY{p}{(}\PY{n}{G\PYZus{}CE}\PY{p}{)}
\PY{n}{promedio\PYZus{}bet} \PY{o}{=} \PY{p}{(}\PY{n+nb}{sum}\PY{p}{(}\PY{n}{dic\PYZus{}bet}\PY{o}{.}\PY{n}{values}\PY{p}{(}\PY{p}{)}\PY{p}{)} \PY{o}{/} \PY{n}{orden}\PY{p}{)}
\PY{n+nb}{print}\PY{p}{(}\PY{l+s+s2}{\PYZdq{}}\PY{l+s+s2}{Beteewness en la red de proteínas de C. elegans: }\PY{l+s+s2}{\PYZdq{}}\PY{p}{,} \PY{n}{promedio\PYZus{}bet}\PY{p}{)}

\PY{n}{mayor} \PY{o}{=} \PY{n+nb}{max}\PY{p}{(}\PY{n}{dic\PYZus{}bet}\PY{o}{.}\PY{n}{items}\PY{p}{(}\PY{p}{)}\PY{p}{,} \PY{n}{key} \PY{o}{=} \PY{k}{lambda} \PY{n}{pareja}\PY{p}{:} \PY{n}{pareja}\PY{p}{[}\PY{l+m+mi}{1}\PY{p}{]} \PY{p}{)}
\PY{n+nb}{print}\PY{p}{(}\PY{l+s+s2}{\PYZdq{}}\PY{l+s+s2}{Nodo con mayor betweeness en la red de proteínas de C. elegans: }\PY{l+s+si}{\PYZpc{}s}\PY{l+s+s2}{, (}\PY{l+s+si}{\PYZpc{}s}\PY{l+s+s2}{)}\PY{l+s+s2}{\PYZdq{}} \PY{o}{\PYZpc{}}\PY{p}{(}\PY{n}{mayor}\PY{p}{[}\PY{l+m+mi}{0}\PY{p}{]}\PY{p}{,} \PY{n}{mayor}\PY{p}{[}\PY{l+m+mi}{1}\PY{p}{]}\PY{p}{)}\PY{p}{)}
\end{Verbatim}
\end{tcolorbox}

    \begin{Verbatim}[commandchars=\\\{\}]
Beteewness en la red de proteínas de C. elegans:  0.0025678228687606676
Nodo con mayor betweeness en la red de proteínas de C. elegans: C23G10.4,
(0.08356781880879106)
\end{Verbatim}

    \begin{tcolorbox}[breakable, size=fbox, boxrule=1pt, pad at break*=1mm,colback=cellbackground, colframe=cellborder]
\prompt{In}{incolor}{29}{\hspace{4pt}}
\begin{Verbatim}[commandchars=\\\{\}]
\PY{n+nb}{print}\PY{p}{(}\PY{l+s+s2}{\PYZdq{}}\PY{l+s+s2}{Promedio del betweenness medioo en el grafo aleatorio: }\PY{l+s+si}{\PYZpc{}s}\PY{l+s+s2}{, con }\PY{l+s+si}{\PYZpc{}s}\PY{l+s+s2}{ desviación estándar}\PY{l+s+s2}{\PYZdq{}}
      \PY{o}{\PYZpc{}}\PY{p}{(}\PY{n}{dic\PYZus{}params}\PY{p}{[}\PY{l+s+s2}{\PYZdq{}}\PY{l+s+s2}{Betweenness}\PY{l+s+s2}{\PYZdq{}}\PY{p}{]}\PY{p}{[}\PY{l+m+mi}{0}\PY{p}{]}\PY{p}{,} \PY{n}{dic\PYZus{}params}\PY{p}{[}\PY{l+s+s2}{\PYZdq{}}\PY{l+s+s2}{Betweenness}\PY{l+s+s2}{\PYZdq{}}\PY{p}{]}\PY{p}{[}\PY{l+m+mi}{1}\PY{p}{]}\PY{p}{)}\PY{p}{)}
\PY{n+nb}{print}\PY{p}{(}\PY{p}{)}
\PY{n+nb}{print}\PY{p}{(}\PY{l+s+s2}{\PYZdq{}}\PY{l+s+s2}{Centralidad media del nodo con mayor betweenness en el grafo aleatorio: }\PY{l+s+si}{\PYZpc{}s}\PY{l+s+s2}{, con desviación estándar }\PY{l+s+si}{\PYZpc{}s}\PY{l+s+s2}{\PYZdq{}} 
      \PY{o}{\PYZpc{}}\PY{p}{(}\PY{n}{dic\PYZus{}params}\PY{p}{[}\PY{l+s+s2}{\PYZdq{}}\PY{l+s+s2}{Nodo max betweenness}\PY{l+s+s2}{\PYZdq{}}\PY{p}{]}\PY{p}{[}\PY{l+m+mi}{0}\PY{p}{]}\PY{p}{,} \PY{n}{dic\PYZus{}params}\PY{p}{[}\PY{l+s+s2}{\PYZdq{}}\PY{l+s+s2}{Nodo max betweenness}\PY{l+s+s2}{\PYZdq{}}\PY{p}{]}\PY{p}{[}\PY{l+m+mi}{1}\PY{p}{]}\PY{p}{)}\PY{p}{)}
\end{Verbatim}
\end{tcolorbox}

    \begin{Verbatim}[commandchars=\\\{\}]
Promedio del betweenness medioo en el grafo aleatorio: 0.003852702791742197, con
0.0001306928874608582 desviación estándar

Centralidad media del nodo con mayor betweenness en el grafo aleatorio:
328.47126017957453, con desviación estándar 425.60895451056115
\end{Verbatim}

    \(\\\)

\hypertarget{ejemplos-donde-se-compara-closeness-y-betweenness-de-grafos-aleatorios-y-redes-de-proteuxednas-procedentes-de-string}{%
\subsubsection{Ejemplos donde se compara ``closeness y betweenness'' de
grafos aleatorios y redes de proteínas procedentes de
STRING}\label{ejemplos-donde-se-compara-closeness-y-betweenness-de-grafos-aleatorios-y-redes-de-proteuxednas-procedentes-de-string}}

A continuación se proponen algunos ejemplos de redes biológicas
obtenidas a partir del reprositorio STRING, sobre los cuales se han
calculado los valores de cercanía y betweennes, y se han generado sus
respectivos grafos aleatorios de mismo orden y tamaño, con los que se
han comparado. Este estudio se ha realizado para comprobar si la
diferencia entre un grafo no aleatorio y un grafo aleatorio en los
valores de ``closeness y betweenness'' presenta grandes diferencias. Nos
ha parecido interesante estudiarlo, dado que el grafo CaernoElegans y el
grafo aleatorio presentan distinto valores de ``closeness y
betweenness'', pero no sabíamos qué grafo tendría que tener un valor
superior o inferior con respecto al otro:

\textbf{Ejemplo 1: Aleatoria vs Red ZNF480
\href{https://string-db.org/cgi/network.pl?taskId=cSfmw1qwj02H}{link}}

    \begin{tcolorbox}[breakable, size=fbox, boxrule=1pt, pad at break*=1mm,colback=cellbackground, colframe=cellborder]
\prompt{In}{incolor}{30}{\hspace{4pt}}
\begin{Verbatim}[commandchars=\\\{\}]
\PY{c+c1}{\PYZsh{} Aleatoria}
\PY{n}{I} \PY{o}{=} \PY{n}{nx}\PY{o}{.}\PY{n}{gnm\PYZus{}random\PYZus{}graph}\PY{p}{(}\PY{l+m+mi}{8}\PY{p}{,} \PY{l+m+mi}{8}\PY{p}{,} \PY{n}{seed} \PY{o}{=} \PY{l+m+mi}{1}\PY{p}{)}

\PY{n}{pos} \PY{o}{=} \PY{p}{\PYZob{}}\PY{l+m+mi}{0}\PY{p}{:}\PY{p}{[}\PY{l+m+mf}{1.00000000e+00}\PY{p}{,} \PY{l+m+mf}{1.83784272e\PYZhy{}08}\PY{p}{]}\PY{p}{,} \PY{l+m+mi}{1}\PY{p}{:}\PY{p}{[}\PY{l+m+mf}{0.70710678}\PY{p}{,} \PY{l+m+mf}{0.70710677}\PY{p}{]}\PY{p}{,} 
       \PY{l+m+mi}{2}\PY{p}{:}\PY{p}{[}\PY{o}{\PYZhy{}}\PY{l+m+mf}{1.73863326e\PYZhy{}08}\PY{p}{,}  \PY{l+m+mf}{9.99999992e\PYZhy{}01}\PY{p}{]}\PY{p}{,} \PY{l+m+mi}{3}\PY{p}{:}\PY{p}{[}\PY{o}{\PYZhy{}}\PY{l+m+mf}{0.70710672}\PY{p}{,}  \PY{l+m+mf}{0.70710677}\PY{p}{]}\PY{p}{,} 
       \PY{l+m+mi}{4}\PY{p}{:}\PY{p}{[}\PY{o}{\PYZhy{}}\PY{l+m+mf}{9.99999947e\PYZhy{}01}\PY{p}{,} \PY{o}{\PYZhy{}}\PY{l+m+mf}{6.90443471e\PYZhy{}08}\PY{p}{]}\PY{p}{,} \PY{l+m+mi}{5}\PY{p}{:}\PY{p}{[}\PY{o}{\PYZhy{}}\PY{l+m+mf}{0.70710678}\PY{p}{,} \PY{o}{\PYZhy{}}\PY{l+m+mf}{0.70710667}\PY{p}{]}\PY{p}{,} 
       \PY{l+m+mi}{6}\PY{p}{:}\PY{p}{[} \PY{l+m+mf}{3.82499349e\PYZhy{}08}\PY{p}{,} \PY{o}{\PYZhy{}}\PY{l+m+mf}{9.99999955e\PYZhy{}01}\PY{p}{]}\PY{p}{,} \PY{l+m+mi}{7}\PY{p}{:}\PY{p}{[} \PY{l+m+mf}{0.70710666}\PY{p}{,} \PY{o}{\PYZhy{}}\PY{l+m+mf}{0.70710685}\PY{p}{]}\PY{p}{\PYZcb{}}

\PY{n}{nx}\PY{o}{.}\PY{n}{draw}\PY{p}{(}\PY{n}{I}\PY{p}{,} \PY{n}{with\PYZus{}labels} \PY{o}{=} \PY{k+kc}{True}\PY{p}{,} \PY{n}{pos} \PY{o}{=} \PY{n}{pos}\PY{p}{)}

\PY{c+c1}{\PYZsh{} Cercanía}
\PY{n}{dic\PYZus{}cercania} \PY{o}{=} \PY{n}{nx}\PY{o}{.}\PY{n}{closeness\PYZus{}centrality}\PY{p}{(}\PY{n}{I}\PY{p}{)}
\PY{n}{promedio\PYZus{}cercania} \PY{o}{=} \PY{p}{(}\PY{n+nb}{sum}\PY{p}{(}\PY{n}{dic\PYZus{}cercania}\PY{o}{.}\PY{n}{values}\PY{p}{(}\PY{p}{)}\PY{p}{)} \PY{o}{/} \PY{n}{orden}\PY{p}{)}
\PY{n+nb}{print}\PY{p}{(}\PY{l+s+s2}{\PYZdq{}}\PY{l+s+s2}{Grafo aleatorio}\PY{l+s+s2}{\PYZdq{}}\PY{p}{)}
\PY{n+nb}{print}\PY{p}{(}\PY{l+s+s2}{\PYZdq{}}\PY{l+s+s2}{Promedio de la cercanía: }\PY{l+s+s2}{\PYZdq{}}\PY{p}{,} \PY{n+nb}{round}\PY{p}{(}\PY{n}{promedio\PYZus{}cercania}\PY{p}{,} \PY{l+m+mi}{5}\PY{p}{)}\PY{p}{)}

\PY{c+c1}{\PYZsh{} Betweenness}
\PY{n}{dic\PYZus{}bet} \PY{o}{=} \PY{n}{nx}\PY{o}{.}\PY{n}{betweenness\PYZus{}centrality}\PY{p}{(}\PY{n}{I}\PY{p}{)}
\PY{n}{promedio\PYZus{}bet} \PY{o}{=} \PY{p}{(}\PY{n+nb}{sum}\PY{p}{(}\PY{n}{dic\PYZus{}bet}\PY{o}{.}\PY{n}{values}\PY{p}{(}\PY{p}{)}\PY{p}{)} \PY{o}{/} \PY{n}{orden}\PY{p}{)}
\PY{n+nb}{print}\PY{p}{(}\PY{l+s+s2}{\PYZdq{}}\PY{l+s+s2}{Promedio del betweenness:}\PY{l+s+s2}{\PYZdq{}}\PY{p}{,} \PY{n+nb}{round}\PY{p}{(}\PY{n}{promedio\PYZus{}bet}\PY{p}{,}\PY{l+m+mi}{5}\PY{p}{)}\PY{p}{)}
\end{Verbatim}
\end{tcolorbox}

    \begin{Verbatim}[commandchars=\\\{\}]
Grafo aleatorio
Promedio de la cercanía:  0.00253
Promedio del betweenness: 0.00134
\end{Verbatim}

    \begin{center}
    \adjustimage{max size={0.9\linewidth}{0.9\paperheight}}{output_67_1.png}
    \end{center}
    { \hspace*{\fill} \\}
    
    \begin{tcolorbox}[breakable, size=fbox, boxrule=1pt, pad at break*=1mm,colback=cellbackground, colframe=cellborder]
\prompt{In}{incolor}{31}{\hspace{4pt}}
\begin{Verbatim}[commandchars=\\\{\}]
\PY{c+c1}{\PYZsh{} Red de proteínas relacionazas con la proteína ZNF480}
\PY{n}{L} \PY{o}{=} \PY{n}{nx}\PY{o}{.}\PY{n}{Graph}\PY{p}{(}\PY{p}{)}
\PY{n}{L}\PY{o}{.}\PY{n}{add\PYZus{}nodes\PYZus{}from}\PY{p}{(}\PY{p}{[}\PY{l+s+s1}{\PYZsq{}}\PY{l+s+s1}{1}\PY{l+s+s1}{\PYZsq{}}\PY{p}{,} \PY{l+s+s1}{\PYZsq{}}\PY{l+s+s1}{2}\PY{l+s+s1}{\PYZsq{}}\PY{p}{,} \PY{l+s+s1}{\PYZsq{}}\PY{l+s+s1}{3}\PY{l+s+s1}{\PYZsq{}}\PY{p}{,} \PY{l+s+s1}{\PYZsq{}}\PY{l+s+s1}{4}\PY{l+s+s1}{\PYZsq{}}\PY{p}{,} \PY{l+s+s1}{\PYZsq{}}\PY{l+s+s1}{5}\PY{l+s+s1}{\PYZsq{}}\PY{p}{,} \PY{l+s+s1}{\PYZsq{}}\PY{l+s+s1}{6}\PY{l+s+s1}{\PYZsq{}}\PY{p}{,} \PY{l+s+s1}{\PYZsq{}}\PY{l+s+s1}{7}\PY{l+s+s1}{\PYZsq{}}\PY{p}{,} \PY{l+s+s1}{\PYZsq{}}\PY{l+s+s1}{8}\PY{l+s+s1}{\PYZsq{}}\PY{p}{]}\PY{p}{)}
\PY{n}{L}\PY{o}{.}\PY{n}{add\PYZus{}edges\PYZus{}from}\PY{p}{(}\PY{p}{[}\PY{p}{(}\PY{l+s+s1}{\PYZsq{}}\PY{l+s+s1}{1}\PY{l+s+s1}{\PYZsq{}}\PY{p}{,} \PY{l+s+s1}{\PYZsq{}}\PY{l+s+s1}{2}\PY{l+s+s1}{\PYZsq{}}\PY{p}{)}\PY{p}{,} \PY{p}{(}\PY{l+s+s1}{\PYZsq{}}\PY{l+s+s1}{1}\PY{l+s+s1}{\PYZsq{}}\PY{p}{,} \PY{l+s+s1}{\PYZsq{}}\PY{l+s+s1}{3}\PY{l+s+s1}{\PYZsq{}}\PY{p}{)}\PY{p}{,} \PY{p}{(}\PY{l+s+s1}{\PYZsq{}}\PY{l+s+s1}{1}\PY{l+s+s1}{\PYZsq{}}\PY{p}{,} \PY{l+s+s1}{\PYZsq{}}\PY{l+s+s1}{4}\PY{l+s+s1}{\PYZsq{}}\PY{p}{)}\PY{p}{,} \PY{p}{(}\PY{l+s+s1}{\PYZsq{}}\PY{l+s+s1}{1}\PY{l+s+s1}{\PYZsq{}}\PY{p}{,} \PY{l+s+s1}{\PYZsq{}}\PY{l+s+s1}{5}\PY{l+s+s1}{\PYZsq{}}\PY{p}{)}\PY{p}{,} \PY{p}{(}\PY{l+s+s1}{\PYZsq{}}\PY{l+s+s1}{1}\PY{l+s+s1}{\PYZsq{}}\PY{p}{,} \PY{l+s+s1}{\PYZsq{}}\PY{l+s+s1}{6}\PY{l+s+s1}{\PYZsq{}}\PY{p}{)}\PY{p}{,} \PY{p}{(}\PY{l+s+s1}{\PYZsq{}}\PY{l+s+s1}{1}\PY{l+s+s1}{\PYZsq{}}\PY{p}{,} \PY{l+s+s1}{\PYZsq{}}\PY{l+s+s1}{7}\PY{l+s+s1}{\PYZsq{}}\PY{p}{)}\PY{p}{,} \PY{p}{(}\PY{l+s+s1}{\PYZsq{}}\PY{l+s+s1}{1}\PY{l+s+s1}{\PYZsq{}}\PY{p}{,} \PY{l+s+s1}{\PYZsq{}}\PY{l+s+s1}{8}\PY{l+s+s1}{\PYZsq{}}\PY{p}{)}\PY{p}{,} \PY{p}{(}\PY{l+s+s1}{\PYZsq{}}\PY{l+s+s1}{2}\PY{l+s+s1}{\PYZsq{}}\PY{p}{,} \PY{l+s+s1}{\PYZsq{}}\PY{l+s+s1}{3}\PY{l+s+s1}{\PYZsq{}}\PY{p}{)}\PY{p}{]}\PY{p}{)}

\PY{n}{pos} \PY{o}{=} \PY{p}{\PYZob{}}\PY{l+s+s1}{\PYZsq{}}\PY{l+s+s1}{1}\PY{l+s+s1}{\PYZsq{}}\PY{p}{:} \PY{p}{[}\PY{o}{\PYZhy{}}\PY{l+m+mf}{0.00162884}\PY{p}{,} \PY{o}{\PYZhy{}}\PY{l+m+mf}{0.01468591}\PY{p}{]}\PY{p}{,} \PY{l+s+s1}{\PYZsq{}}\PY{l+s+s1}{2}\PY{l+s+s1}{\PYZsq{}}\PY{p}{:} \PY{p}{[}\PY{o}{\PYZhy{}}\PY{l+m+mf}{0.22472125}\PY{p}{,}  \PY{l+m+mf}{0.90837056}\PY{p}{]}\PY{p}{,} 
       \PY{l+s+s1}{\PYZsq{}}\PY{l+s+s1}{3}\PY{l+s+s1}{\PYZsq{}}\PY{p}{:} \PY{p}{[}\PY{l+m+mf}{0.30617573}\PY{p}{,} \PY{l+m+mf}{0.88310596}\PY{p}{]}\PY{p}{,} \PY{l+s+s1}{\PYZsq{}}\PY{l+s+s1}{4}\PY{l+s+s1}{\PYZsq{}}\PY{p}{:} \PY{p}{[}\PY{o}{\PYZhy{}}\PY{l+m+mf}{0.94317228}\PY{p}{,}  \PY{l+m+mf}{0.27793508}\PY{p}{]}\PY{p}{,} 
       \PY{l+s+s1}{\PYZsq{}}\PY{l+s+s1}{5}\PY{l+s+s1}{\PYZsq{}}\PY{p}{:} \PY{p}{[}\PY{o}{\PYZhy{}}\PY{l+m+mf}{0.04744948}\PY{p}{,} \PY{o}{\PYZhy{}}\PY{l+m+mf}{1.}        \PY{p}{]}\PY{p}{,} \PY{l+s+s1}{\PYZsq{}}\PY{l+s+s1}{6}\PY{l+s+s1}{\PYZsq{}}\PY{p}{:} \PY{p}{[}\PY{l+m+mf}{0.96677375}\PY{p}{,} \PY{l+m+mf}{0.19258629}\PY{p}{]}\PY{p}{,}
       \PY{l+s+s1}{\PYZsq{}}\PY{l+s+s1}{7}\PY{l+s+s1}{\PYZsq{}}\PY{p}{:} \PY{p}{[}\PY{o}{\PYZhy{}}\PY{l+m+mf}{0.80946116}\PY{p}{,} \PY{o}{\PYZhy{}}\PY{l+m+mf}{0.58527708}\PY{p}{]}\PY{p}{,} \PY{l+s+s1}{\PYZsq{}}\PY{l+s+s1}{8}\PY{l+s+s1}{\PYZsq{}}\PY{p}{:} \PY{p}{[} \PY{l+m+mf}{0.75348354}\PY{p}{,} \PY{o}{\PYZhy{}}\PY{l+m+mf}{0.66203489}\PY{p}{]}\PY{p}{\PYZcb{}}

\PY{n}{nx}\PY{o}{.}\PY{n}{draw}\PY{p}{(}\PY{n}{L}\PY{p}{,} \PY{n}{with\PYZus{}labels} \PY{o}{=} \PY{k+kc}{True}\PY{p}{,} \PY{n}{pos} \PY{o}{=} \PY{n}{pos}\PY{p}{)}

\PY{n+nb}{print}\PY{p}{(}\PY{l+s+s2}{\PYZdq{}}\PY{l+s+s2}{Grafo de la proteína ZNF480 }\PY{l+s+s2}{\PYZdq{}}\PY{p}{)}

\PY{c+c1}{\PYZsh{} Cercanía}
\PY{n}{dic\PYZus{}cercania} \PY{o}{=} \PY{n}{nx}\PY{o}{.}\PY{n}{closeness\PYZus{}centrality}\PY{p}{(}\PY{n}{L}\PY{p}{)}
\PY{n}{promedio\PYZus{}cercania} \PY{o}{=} \PY{p}{(}\PY{n+nb}{sum}\PY{p}{(}\PY{n}{dic\PYZus{}cercania}\PY{o}{.}\PY{n}{values}\PY{p}{(}\PY{p}{)}\PY{p}{)} \PY{o}{/} \PY{n}{orden}\PY{p}{)}
\PY{n+nb}{print}\PY{p}{(}\PY{l+s+s2}{\PYZdq{}}\PY{l+s+s2}{Promedio de la cercanía: }\PY{l+s+s2}{\PYZdq{}}\PY{p}{,} \PY{n+nb}{round}\PY{p}{(}\PY{n}{promedio\PYZus{}cercania}\PY{p}{,} \PY{l+m+mi}{5}\PY{p}{)}\PY{p}{)}

\PY{c+c1}{\PYZsh{} Betweenness}
\PY{n}{dic\PYZus{}bet} \PY{o}{=} \PY{n}{nx}\PY{o}{.}\PY{n}{betweenness\PYZus{}centrality}\PY{p}{(}\PY{n}{L}\PY{p}{)}
\PY{n}{promedio\PYZus{}bet} \PY{o}{=} \PY{p}{(}\PY{n+nb}{sum}\PY{p}{(}\PY{n}{dic\PYZus{}bet}\PY{o}{.}\PY{n}{values}\PY{p}{(}\PY{p}{)}\PY{p}{)} \PY{o}{/} \PY{n}{orden}\PY{p}{)}
\PY{n+nb}{print}\PY{p}{(}\PY{l+s+s2}{\PYZdq{}}\PY{l+s+s2}{Promedio del betweenness:}\PY{l+s+s2}{\PYZdq{}}\PY{p}{,} \PY{n+nb}{round}\PY{p}{(}\PY{n}{promedio\PYZus{}bet}\PY{p}{,}\PY{l+m+mi}{5}\PY{p}{)}\PY{p}{)}
\end{Verbatim}
\end{tcolorbox}

    \begin{Verbatim}[commandchars=\\\{\}]
Grafo de la proteína ZNF480
Promedio de la cercanía:  0.0035
Promedio del betweenness: 0.00069
\end{Verbatim}

    \begin{center}
    \adjustimage{max size={0.9\linewidth}{0.9\paperheight}}{output_68_1.png}
    \end{center}
    { \hspace*{\fill} \\}
    
    \textbf{Ejemplo 2: Aleatoria vs GLIPR2
\href{https://string-db.org/cgi/network.pl?taskId=yRGyrjXoPGkU}{link}}

    \begin{tcolorbox}[breakable, size=fbox, boxrule=1pt, pad at break*=1mm,colback=cellbackground, colframe=cellborder]
\prompt{In}{incolor}{32}{\hspace{4pt}}
\begin{Verbatim}[commandchars=\\\{\}]
\PY{c+c1}{\PYZsh{} Aleatoria}
\PY{n}{I} \PY{o}{=} \PY{n}{nx}\PY{o}{.}\PY{n}{gnm\PYZus{}random\PYZus{}graph}\PY{p}{(}\PY{l+m+mi}{11}\PY{p}{,} \PY{l+m+mi}{14}\PY{p}{,} \PY{n}{seed} \PY{o}{=} \PY{l+m+mi}{1}\PY{p}{)}

\PY{n}{pos} \PY{o}{=} \PY{p}{\PYZob{}}\PY{l+m+mi}{0}\PY{p}{:} \PY{p}{[}\PY{l+m+mf}{1.}\PY{p}{,} \PY{l+m+mf}{0.}\PY{p}{]}\PY{p}{,} \PY{l+m+mi}{1}\PY{p}{:} \PY{p}{[}\PY{l+m+mf}{0.84125352}\PY{p}{,} \PY{l+m+mf}{0.54064077}\PY{p}{]}\PY{p}{,} \PY{l+m+mi}{2}\PY{p}{:} \PY{p}{[}\PY{l+m+mf}{0.41541505}\PY{p}{,} \PY{l+m+mf}{0.90963196}\PY{p}{]}\PY{p}{,}
       \PY{l+m+mi}{3}\PY{p}{:} \PY{p}{[}\PY{o}{\PYZhy{}}\PY{l+m+mf}{0.14231483}\PY{p}{,}  \PY{l+m+mf}{0.98982143}\PY{p}{]}\PY{p}{,} \PY{l+m+mi}{4}\PY{p}{:} \PY{p}{[}\PY{o}{\PYZhy{}}\PY{l+m+mf}{0.65486066}\PY{p}{,}  \PY{l+m+mf}{0.75574964}\PY{p}{]}\PY{p}{,} 
       \PY{l+m+mi}{5}\PY{p}{:} \PY{p}{[}\PY{o}{\PYZhy{}}\PY{l+m+mf}{0.95949297}\PY{p}{,}  \PY{l+m+mf}{0.28173262}\PY{p}{]}\PY{p}{,} \PY{l+m+mi}{6}\PY{p}{:} \PY{p}{[}\PY{o}{\PYZhy{}}\PY{l+m+mf}{0.95949297}\PY{p}{,} \PY{o}{\PYZhy{}}\PY{l+m+mf}{0.28173256}\PY{p}{]}\PY{p}{,} 
       \PY{l+m+mi}{7}\PY{p}{:} \PY{p}{[}\PY{o}{\PYZhy{}}\PY{l+m+mf}{0.65486072}\PY{p}{,} \PY{o}{\PYZhy{}}\PY{l+m+mf}{0.75574958}\PY{p}{]}\PY{p}{,} \PY{l+m+mi}{8}\PY{p}{:} \PY{p}{[}\PY{o}{\PYZhy{}}\PY{l+m+mf}{0.14231501}\PY{p}{,} \PY{o}{\PYZhy{}}\PY{l+m+mf}{0.98982143}\PY{p}{]}\PY{p}{,}
       \PY{l+m+mi}{9}\PY{p}{:} \PY{p}{[} \PY{l+m+mf}{0.41541511}\PY{p}{,} \PY{o}{\PYZhy{}}\PY{l+m+mf}{0.90963196}\PY{p}{]}\PY{p}{,} \PY{l+m+mi}{10}\PY{p}{:} \PY{p}{[} \PY{l+m+mf}{0.84125346}\PY{p}{,} \PY{o}{\PYZhy{}}\PY{l+m+mf}{0.54064089}\PY{p}{]}\PY{p}{\PYZcb{}}

\PY{n}{nx}\PY{o}{.}\PY{n}{draw}\PY{p}{(}\PY{n}{I}\PY{p}{,} \PY{n}{with\PYZus{}labels} \PY{o}{=} \PY{k+kc}{True}\PY{p}{,} \PY{n}{pos} \PY{o}{=} \PY{n}{pos}\PY{p}{)}
\PY{n+nb}{print}\PY{p}{(}\PY{l+s+s2}{\PYZdq{}}\PY{l+s+s2}{Grafo aleatorio}\PY{l+s+s2}{\PYZdq{}}\PY{p}{)}

\PY{c+c1}{\PYZsh{} Cercanía}
\PY{n}{dic\PYZus{}cercania} \PY{o}{=} \PY{n}{nx}\PY{o}{.}\PY{n}{closeness\PYZus{}centrality}\PY{p}{(}\PY{n}{I}\PY{p}{)}
\PY{n}{promedio\PYZus{}cercania} \PY{o}{=} \PY{p}{(}\PY{n+nb}{sum}\PY{p}{(}\PY{n}{dic\PYZus{}cercania}\PY{o}{.}\PY{n}{values}\PY{p}{(}\PY{p}{)}\PY{p}{)} \PY{o}{/} \PY{n}{orden}\PY{p}{)}
\PY{n+nb}{print}\PY{p}{(}\PY{l+s+s2}{\PYZdq{}}\PY{l+s+s2}{Promedio de la cercanía: }\PY{l+s+s2}{\PYZdq{}}\PY{p}{,} \PY{n+nb}{round}\PY{p}{(}\PY{n}{promedio\PYZus{}cercania}\PY{p}{,} \PY{l+m+mi}{5}\PY{p}{)}\PY{p}{)}

\PY{c+c1}{\PYZsh{} Betweenness}
\PY{n}{dic\PYZus{}bet} \PY{o}{=} \PY{n}{nx}\PY{o}{.}\PY{n}{betweenness\PYZus{}centrality}\PY{p}{(}\PY{n}{I}\PY{p}{)}
\PY{n}{promedio\PYZus{}bet} \PY{o}{=} \PY{p}{(}\PY{n+nb}{sum}\PY{p}{(}\PY{n}{dic\PYZus{}bet}\PY{o}{.}\PY{n}{values}\PY{p}{(}\PY{p}{)}\PY{p}{)} \PY{o}{/} \PY{n}{orden}\PY{p}{)}
\PY{n+nb}{print}\PY{p}{(}\PY{l+s+s2}{\PYZdq{}}\PY{l+s+s2}{Promedio del betweenness:}\PY{l+s+s2}{\PYZdq{}}\PY{p}{,} \PY{n+nb}{round}\PY{p}{(}\PY{n}{promedio\PYZus{}bet}\PY{p}{,}\PY{l+m+mi}{5}\PY{p}{)}\PY{p}{)}
\end{Verbatim}
\end{tcolorbox}

    \begin{Verbatim}[commandchars=\\\{\}]
Grafo aleatorio
Promedio de la cercanía:  0.00377
Promedio del betweenness: 0.00104
\end{Verbatim}

    \begin{center}
    \adjustimage{max size={0.9\linewidth}{0.9\paperheight}}{output_70_1.png}
    \end{center}
    { \hspace*{\fill} \\}
    
    \begin{tcolorbox}[breakable, size=fbox, boxrule=1pt, pad at break*=1mm,colback=cellbackground, colframe=cellborder]
\prompt{In}{incolor}{33}{\hspace{4pt}}
\begin{Verbatim}[commandchars=\\\{\}]
\PY{c+c1}{\PYZsh{} Red de proteínas relacionazas con la proteína GLIPR2}
\PY{n}{L} \PY{o}{=} \PY{n}{nx}\PY{o}{.}\PY{n}{Graph}\PY{p}{(}\PY{p}{)}
\PY{n}{L}\PY{o}{.}\PY{n}{add\PYZus{}nodes\PYZus{}from}\PY{p}{(}\PY{p}{[}\PY{l+s+s1}{\PYZsq{}}\PY{l+s+s1}{1}\PY{l+s+s1}{\PYZsq{}}\PY{p}{,} \PY{l+s+s1}{\PYZsq{}}\PY{l+s+s1}{2}\PY{l+s+s1}{\PYZsq{}}\PY{p}{,} \PY{l+s+s1}{\PYZsq{}}\PY{l+s+s1}{3}\PY{l+s+s1}{\PYZsq{}}\PY{p}{,} \PY{l+s+s1}{\PYZsq{}}\PY{l+s+s1}{4}\PY{l+s+s1}{\PYZsq{}}\PY{p}{,} \PY{l+s+s1}{\PYZsq{}}\PY{l+s+s1}{5}\PY{l+s+s1}{\PYZsq{}}\PY{p}{,} \PY{l+s+s1}{\PYZsq{}}\PY{l+s+s1}{6}\PY{l+s+s1}{\PYZsq{}}\PY{p}{,} \PY{l+s+s1}{\PYZsq{}}\PY{l+s+s1}{7}\PY{l+s+s1}{\PYZsq{}}\PY{p}{,} \PY{l+s+s1}{\PYZsq{}}\PY{l+s+s1}{8}\PY{l+s+s1}{\PYZsq{}}\PY{p}{,} \PY{l+s+s1}{\PYZsq{}}\PY{l+s+s1}{9}\PY{l+s+s1}{\PYZsq{}}\PY{p}{,} \PY{l+s+s1}{\PYZsq{}}\PY{l+s+s1}{10}\PY{l+s+s1}{\PYZsq{}}\PY{p}{,} \PY{l+s+s1}{\PYZsq{}}\PY{l+s+s1}{11}\PY{l+s+s1}{\PYZsq{}}\PY{p}{]}\PY{p}{)}
\PY{n}{L}\PY{o}{.}\PY{n}{add\PYZus{}edges\PYZus{}from}\PY{p}{(}\PY{p}{[}\PY{p}{(}\PY{l+s+s1}{\PYZsq{}}\PY{l+s+s1}{1}\PY{l+s+s1}{\PYZsq{}}\PY{p}{,} \PY{l+s+s1}{\PYZsq{}}\PY{l+s+s1}{2}\PY{l+s+s1}{\PYZsq{}}\PY{p}{)}\PY{p}{,} \PY{p}{(}\PY{l+s+s1}{\PYZsq{}}\PY{l+s+s1}{1}\PY{l+s+s1}{\PYZsq{}}\PY{p}{,} \PY{l+s+s1}{\PYZsq{}}\PY{l+s+s1}{3}\PY{l+s+s1}{\PYZsq{}}\PY{p}{)}\PY{p}{,} \PY{p}{(}\PY{l+s+s1}{\PYZsq{}}\PY{l+s+s1}{1}\PY{l+s+s1}{\PYZsq{}}\PY{p}{,} \PY{l+s+s1}{\PYZsq{}}\PY{l+s+s1}{4}\PY{l+s+s1}{\PYZsq{}}\PY{p}{)}\PY{p}{,} \PY{p}{(}\PY{l+s+s1}{\PYZsq{}}\PY{l+s+s1}{1}\PY{l+s+s1}{\PYZsq{}}\PY{p}{,} \PY{l+s+s1}{\PYZsq{}}\PY{l+s+s1}{5}\PY{l+s+s1}{\PYZsq{}}\PY{p}{)}\PY{p}{,} \PY{p}{(}\PY{l+s+s1}{\PYZsq{}}\PY{l+s+s1}{1}\PY{l+s+s1}{\PYZsq{}}\PY{p}{,} \PY{l+s+s1}{\PYZsq{}}\PY{l+s+s1}{6}\PY{l+s+s1}{\PYZsq{}}\PY{p}{)}\PY{p}{,}
                  \PY{p}{(}\PY{l+s+s1}{\PYZsq{}}\PY{l+s+s1}{1}\PY{l+s+s1}{\PYZsq{}}\PY{p}{,} \PY{l+s+s1}{\PYZsq{}}\PY{l+s+s1}{7}\PY{l+s+s1}{\PYZsq{}}\PY{p}{)}\PY{p}{,} \PY{p}{(}\PY{l+s+s1}{\PYZsq{}}\PY{l+s+s1}{1}\PY{l+s+s1}{\PYZsq{}}\PY{p}{,} \PY{l+s+s1}{\PYZsq{}}\PY{l+s+s1}{8}\PY{l+s+s1}{\PYZsq{}}\PY{p}{)}\PY{p}{,} \PY{p}{(}\PY{l+s+s1}{\PYZsq{}}\PY{l+s+s1}{1}\PY{l+s+s1}{\PYZsq{}}\PY{p}{,} \PY{l+s+s1}{\PYZsq{}}\PY{l+s+s1}{9}\PY{l+s+s1}{\PYZsq{}}\PY{p}{)}\PY{p}{,} \PY{p}{(}\PY{l+s+s1}{\PYZsq{}}\PY{l+s+s1}{1}\PY{l+s+s1}{\PYZsq{}}\PY{p}{,} \PY{l+s+s1}{\PYZsq{}}\PY{l+s+s1}{10}\PY{l+s+s1}{\PYZsq{}}\PY{p}{)}\PY{p}{,} \PY{p}{(}\PY{l+s+s1}{\PYZsq{}}\PY{l+s+s1}{1}\PY{l+s+s1}{\PYZsq{}}\PY{p}{,} \PY{l+s+s1}{\PYZsq{}}\PY{l+s+s1}{11}\PY{l+s+s1}{\PYZsq{}}\PY{p}{)}\PY{p}{,} \PY{p}{(}\PY{l+s+s1}{\PYZsq{}}\PY{l+s+s1}{2}\PY{l+s+s1}{\PYZsq{}}\PY{p}{,} \PY{l+s+s1}{\PYZsq{}}\PY{l+s+s1}{3}\PY{l+s+s1}{\PYZsq{}}\PY{p}{)}\PY{p}{,}
                 \PY{p}{(}\PY{l+s+s1}{\PYZsq{}}\PY{l+s+s1}{9}\PY{l+s+s1}{\PYZsq{}}\PY{p}{,} \PY{l+s+s1}{\PYZsq{}}\PY{l+s+s1}{11}\PY{l+s+s1}{\PYZsq{}}\PY{p}{)}\PY{p}{,} \PY{p}{(}\PY{l+s+s1}{\PYZsq{}}\PY{l+s+s1}{10}\PY{l+s+s1}{\PYZsq{}}\PY{p}{,} \PY{l+s+s1}{\PYZsq{}}\PY{l+s+s1}{11}\PY{l+s+s1}{\PYZsq{}}\PY{p}{)}\PY{p}{,} \PY{p}{(}\PY{l+s+s1}{\PYZsq{}}\PY{l+s+s1}{9}\PY{l+s+s1}{\PYZsq{}}\PY{p}{,} \PY{l+s+s1}{\PYZsq{}}\PY{l+s+s1}{10}\PY{l+s+s1}{\PYZsq{}}\PY{p}{)}\PY{p}{]}\PY{p}{)}

\PY{n}{pos} \PY{o}{=} \PY{n}{nx}\PY{o}{.}\PY{n}{circular\PYZus{}layout}\PY{p}{(}\PY{n}{L}\PY{p}{)}
\PY{n}{nx}\PY{o}{.}\PY{n}{draw}\PY{p}{(}\PY{n}{L}\PY{p}{,} \PY{n}{with\PYZus{}labels} \PY{o}{=} \PY{k+kc}{True}\PY{p}{,} \PY{n}{pos} \PY{o}{=} \PY{n}{pos}\PY{p}{)}

\PY{n+nb}{print}\PY{p}{(}\PY{l+s+s2}{\PYZdq{}}\PY{l+s+s2}{Grafo de la proteína GLIPR2}\PY{l+s+s2}{\PYZdq{}}\PY{p}{)}

\PY{c+c1}{\PYZsh{} Cercanía}
\PY{n}{dic\PYZus{}cercania} \PY{o}{=} \PY{n}{nx}\PY{o}{.}\PY{n}{closeness\PYZus{}centrality}\PY{p}{(}\PY{n}{L}\PY{p}{)}
\PY{n}{promedio\PYZus{}cercania} \PY{o}{=} \PY{p}{(}\PY{n+nb}{sum}\PY{p}{(}\PY{n}{dic\PYZus{}cercania}\PY{o}{.}\PY{n}{values}\PY{p}{(}\PY{p}{)}\PY{p}{)} \PY{o}{/} \PY{n}{orden}\PY{p}{)}
\PY{n+nb}{print}\PY{p}{(}\PY{l+s+s2}{\PYZdq{}}\PY{l+s+s2}{Promedio de la cercanía: }\PY{l+s+s2}{\PYZdq{}}\PY{p}{,} \PY{n+nb}{round}\PY{p}{(}\PY{n}{promedio\PYZus{}cercania}\PY{p}{,} \PY{l+m+mi}{5}\PY{p}{)}\PY{p}{)}

\PY{c+c1}{\PYZsh{} Betweenness}
\PY{n}{dic\PYZus{}bet} \PY{o}{=} \PY{n}{nx}\PY{o}{.}\PY{n}{betweenness\PYZus{}centrality}\PY{p}{(}\PY{n}{L}\PY{p}{)}
\PY{n}{promedio\PYZus{}bet} \PY{o}{=} \PY{p}{(}\PY{n+nb}{sum}\PY{p}{(}\PY{n}{dic\PYZus{}bet}\PY{o}{.}\PY{n}{values}\PY{p}{(}\PY{p}{)}\PY{p}{)} \PY{o}{/} \PY{n}{orden}\PY{p}{)}
\PY{n+nb}{print}\PY{p}{(}\PY{l+s+s2}{\PYZdq{}}\PY{l+s+s2}{Promedio del betweenness:}\PY{l+s+s2}{\PYZdq{}}\PY{p}{,} \PY{n+nb}{round}\PY{p}{(}\PY{n}{promedio\PYZus{}bet}\PY{p}{,}\PY{l+m+mi}{5}\PY{p}{)}\PY{p}{)}
\end{Verbatim}
\end{tcolorbox}

    \begin{Verbatim}[commandchars=\\\{\}]
Grafo de la proteína GLIPR2
Promedio de la cercanía:  0.00469
Promedio del betweenness: 0.00066
\end{Verbatim}

    \begin{center}
    \adjustimage{max size={0.9\linewidth}{0.9\paperheight}}{output_71_1.png}
    \end{center}
    { \hspace*{\fill} \\}
    
    \begin{longtable}[]{@{}lllllll@{}}
\toprule
& ZNF480 & Aleatorio & GLIPR2 & Aleatorio & CaernoElegans &
Aleatorio\tabularnewline
\midrule
\endhead
Cercanía & 0.0035 & 0.00253 & 0.00469 & 0.00377 & 0.071142 & 0.097581 ±
0.00165\tabularnewline
Betwenness & 0.00069 & 0.00134 & 0.00066 & 0.00104 & 0.002567 & 0.003852
± 0.000130\tabularnewline
\bottomrule
\end{longtable}

    \hypertarget{conclusiuxf3n-sobre-los-paruxe1metros-clonesness-y-betweenness}{%
\subsubsection{Conclusión sobre los parámetros ``clonesness y
betweenness''}\label{conclusiuxf3n-sobre-los-paruxe1metros-clonesness-y-betweenness}}

A raíz de los resultados obtenidos, queda claro que los grafos NO
aleatorios presentan valores diferentes a sus correspondientes grafos
aleatorios. El sesgo biológico que presentan los grafos de redes de
proteínas provoca que los valores de ``clonesness y betweenness'' sean
diferentes. Esta diferencia puede darse en ambas direcciones en ambos
parámetros, es decir, con un valor superior o inferior de ``clonesness y
betweenness'' con respecto al del grafo aleatorio. En conclusión, son
parámetros que nos permiten conocer si existe un sesgo en nuestro grafo
de estudio e identificar nodos importantes en el mismo.

    \(\\\)

\hypertarget{d-average_clustering}{%
\subsubsection{d) average\_clustering()}\label{d-average_clustering}}

\hypertarget{d.1-cuxe1lculo-del-uxedndice-de-clusterizaciuxf3n}{%
\paragraph{d.1) Cálculo del índice de
clusterización}\label{d.1-cuxe1lculo-del-uxedndice-de-clusterizaciuxf3n}}

El coeficiente de clusterización o agrupamiento (C) es un valor métrico
local que mide el nivel de agrupamiento de los nodos. Este parámetro nos
permitirá conocer cómo de agrupados se encuentran los nodos en el grafo
CaernoElegans, y si existe alguna diferencia con los grafos aleatorios.

    \begin{tcolorbox}[breakable, size=fbox, boxrule=1pt, pad at break*=1mm,colback=cellbackground, colframe=cellborder]
\prompt{In}{incolor}{34}{\hspace{4pt}}
\begin{Verbatim}[commandchars=\\\{\}]
\PY{c+c1}{\PYZsh{} Grafo inicial}
\PY{n}{c} \PY{o}{=} \PY{n}{nx}\PY{o}{.}\PY{n}{average\PYZus{}clustering}\PY{p}{(}\PY{n}{G\PYZus{}CE}\PY{p}{)}
\PY{n+nb}{print}\PY{p}{(}\PY{l+s+s2}{\PYZdq{}}\PY{l+s+s2}{Índice de clusterización de la red de proteínas de C. elegans:}\PY{l+s+s2}{\PYZdq{}}\PY{p}{,} \PY{n}{c}\PY{p}{)}
\PY{n}{mayor} \PY{o}{=} \PY{n+nb}{max}\PY{p}{(}\PY{n}{nx}\PY{o}{.}\PY{n}{clustering}\PY{p}{(}\PY{n}{G\PYZus{}CE}\PY{p}{)}\PY{o}{.}\PY{n}{items}\PY{p}{(}\PY{p}{)}\PY{p}{,} \PY{n}{key} \PY{o}{=} \PY{k}{lambda} \PY{n}{pareja}\PY{p}{:} \PY{n}{pareja}\PY{p}{[}\PY{l+m+mi}{1}\PY{p}{]}\PY{p}{)}
\PY{n+nb}{print}\PY{p}{(}\PY{l+s+s2}{\PYZdq{}}\PY{l+s+s2}{El nodo con mayor índice de clusterización es: }\PY{l+s+si}{\PYZpc{}s}\PY{l+s+s2}{, (}\PY{l+s+si}{\PYZpc{}s}\PY{l+s+s2}{)}\PY{l+s+s2}{\PYZdq{}} \PY{o}{\PYZpc{}}\PY{p}{(}\PY{n}{mayor}\PY{p}{[}\PY{l+m+mi}{0}\PY{p}{]}\PY{p}{,} \PY{n}{mayor}\PY{p}{[}\PY{l+m+mi}{1}\PY{p}{]}\PY{p}{)}\PY{p}{)}

\PY{k}{try}\PY{p}{:}
    \PY{n}{L} \PY{o}{=} \PY{n}{nx}\PY{o}{.}\PY{n}{average\PYZus{}shortest\PYZus{}path\PYZus{}length}\PY{p}{(}\PY{n}{G\PYZus{}CE}\PY{p}{)}
    \PY{n+nb}{print}\PY{p}{(}\PY{l+s+s2}{\PYZdq{}}\PY{l+s+s2}{Camino mínimo de la red de proteínas de C. elegans:}\PY{l+s+s2}{\PYZdq{}}\PY{p}{,} \PY{n}{L}\PY{p}{)}
\PY{k}{except}\PY{p}{:}
    \PY{n+nb}{print} \PY{p}{(}\PY{l+s+s2}{\PYZdq{}}\PY{l+s+se}{\PYZbs{}n}\PY{l+s+s2}{No puede calcularse el camíno mínimo de un grafo no conexo}\PY{l+s+s2}{\PYZdq{}}\PY{p}{)}
\end{Verbatim}
\end{tcolorbox}

    \begin{Verbatim}[commandchars=\\\{\}]
Índice de clusterización de la red de proteínas de C. elegans:
0.07570841434149081
El nodo con mayor índice de clusterización es: Y38C1AA.2, (1.0)

No puede calcularse el camíno mínimo de un grafo no conexo
\end{Verbatim}

    \begin{tcolorbox}[breakable, size=fbox, boxrule=1pt, pad at break*=1mm,colback=cellbackground, colframe=cellborder]
\prompt{In}{incolor}{35}{\hspace{4pt}}
\begin{Verbatim}[commandchars=\\\{\}]
\PY{c+c1}{\PYZsh{} Aleatorio}
\PY{n+nb}{print}\PY{p}{(}\PY{l+s+s2}{\PYZdq{}}\PY{l+s+s2}{Promedio del índice de clústering medio de grado en el grafo aleatorio: }\PY{l+s+si}{\PYZpc{}s}\PY{l+s+s2}{, con }\PY{l+s+si}{\PYZpc{}s}\PY{l+s+s2}{ desviación estándar}\PY{l+s+s2}{\PYZdq{}}
      \PY{o}{\PYZpc{}}\PY{p}{(}\PY{n}{dic\PYZus{}params}\PY{p}{[}\PY{l+s+s2}{\PYZdq{}}\PY{l+s+s2}{Clustering}\PY{l+s+s2}{\PYZdq{}}\PY{p}{]}\PY{p}{,} \PY{n}{dic\PYZus{}params}\PY{p}{[}\PY{l+s+s2}{\PYZdq{}}\PY{l+s+s2}{Clustering}\PY{l+s+s2}{\PYZdq{}}\PY{p}{]}\PY{p}{[}\PY{l+m+mi}{1}\PY{p}{]}\PY{p}{)}\PY{p}{)}
\PY{n+nb}{print}\PY{p}{(}\PY{p}{)}
\PY{n+nb}{print}\PY{p}{(}\PY{l+s+s2}{\PYZdq{}}\PY{l+s+s2}{Clustering media del nodo con mayor índice de clustering en el grafo aleatorio: }\PY{l+s+si}{\PYZpc{}s}\PY{l+s+s2}{, con desviación estándar }\PY{l+s+si}{\PYZpc{}s}\PY{l+s+s2}{\PYZdq{}} 
      \PY{o}{\PYZpc{}}\PY{p}{(}\PY{n}{dic\PYZus{}params}\PY{p}{[}\PY{l+s+s2}{\PYZdq{}}\PY{l+s+s2}{Nodo max clustering}\PY{l+s+s2}{\PYZdq{}}\PY{p}{]}\PY{p}{[}\PY{l+m+mi}{0}\PY{p}{]}\PY{p}{,} \PY{n}{dic\PYZus{}params}\PY{p}{[}\PY{l+s+s2}{\PYZdq{}}\PY{l+s+s2}{Nodo max clustering}\PY{l+s+s2}{\PYZdq{}}\PY{p}{]}\PY{p}{[}\PY{l+m+mi}{1}\PY{p}{]}\PY{p}{)}\PY{p}{)}
\end{Verbatim}
\end{tcolorbox}

    \begin{Verbatim}[commandchars=\\\{\}]
Promedio del índice de clústering medio de grado en el grafo aleatorio:
(0.0009396727601486074, 0.001109340196485008), con 0.001109340196485008
desviación estándar

Clustering media del nodo con mayor índice de clustering en el grafo aleatorio:
0.4133333333333334, con desviación estándar 0.4019950248448356
\end{Verbatim}

    Una red biológica tiene un índice de clusterización alto, pero las redes
aleatorias tienen un índice de clusterización bajo (dado que las ramas
se distribuyen de forma aleatoria y es más difícil que aparezcan
agrupaciones de nodos). De este modo, podemos comprobar de nuevo que la
red de proteínas de \emph{C. elegans} efectivamente corresponde a una
red no aleatoria.

Igualmente, sería interesante conocer el camino característico de
CaernoElegans para ver si se trata de una red de mundo pequeño, las
cuales presentan un índice de clusterización alto (superior al de un
grafo aleatorio) y un camino característico bajo (similar al de un grafo
aleatorio).

    Adicionalmente, se ha comprobado que en un grafo aleatorio la
probabilidad de que dos vecinos de un nodo dado esten conectados es
igual a la que dos nodos elegidos al azar esten conectados (\(p\) =
0.0017, anteriormente calculada), por lo que:

\[C_{aleatorio} \simeq p \simeq \frac{<k>}{N-1} \rightarrow 0.00094 \simeq 0.0017\]

    \(\\\)

\hypertarget{d.2-cuxe1lculo-del-camino-muxednimo-de-la-componente-mayor}{%
\paragraph{d.2) Cálculo del camino mínimo de la componente
mayor}\label{d.2-cuxe1lculo-del-camino-muxednimo-de-la-componente-mayor}}

A pesar de que no se puede calcular el camino mínimo para un grafo no
conexo, sí podemos conocer la componente conexa de mayor tamaño y
calcular en ella su camino mínimo, observando que son muy similares
entre el grafo CaernoElegans y los grafos aleatorios:

    \begin{tcolorbox}[breakable, size=fbox, boxrule=1pt, pad at break*=1mm,colback=cellbackground, colframe=cellborder]
\prompt{In}{incolor}{36}{\hspace{4pt}}
\begin{Verbatim}[commandchars=\\\{\}]
\PY{c+c1}{\PYZsh{} Grafo C. elegans}

\PY{c+c1}{\PYZsh{} Número de componentes conexas.}
\PY{n}{a} \PY{o}{=} \PY{n}{nx}\PY{o}{.}\PY{n}{number\PYZus{}connected\PYZus{}components}\PY{p}{(}\PY{n}{G\PYZus{}CE}\PY{p}{)}
\PY{n+nb}{print} \PY{p}{(}\PY{l+s+s2}{\PYZdq{}}\PY{l+s+s2}{Número de componentes conexas del grafo de C. elegans:}\PY{l+s+s2}{\PYZdq{}}\PY{p}{,} \PY{n}{a}\PY{p}{)}

\PY{c+c1}{\PYZsh{} Para conocer el camino mínimo de la más grande obtengo como subgrafo la componente de mayor tamaño}
\PY{n}{max\PYZus{}componente} \PY{o}{=} \PY{n+nb}{max}\PY{p}{(}\PY{n}{nx}\PY{o}{.}\PY{n}{connected\PYZus{}component\PYZus{}subgraphs}\PY{p}{(}\PY{n}{G\PYZus{}CE}\PY{p}{)}\PY{p}{,} \PY{n}{key} \PY{o}{=} \PY{n+nb}{len}\PY{p}{)}
\PY{n}{min\PYZus{}path\PYZus{}max\PYZus{}componente} \PY{o}{=} \PY{n}{nx}\PY{o}{.}\PY{n}{average\PYZus{}shortest\PYZus{}path\PYZus{}length}\PY{p}{(}\PY{n}{max\PYZus{}componente}\PY{p}{)}
\PY{n+nb}{print} \PY{p}{(}\PY{l+s+s2}{\PYZdq{}}\PY{l+s+se}{\PYZbs{}n}\PY{l+s+s2}{Camino mínimo de la componente más grande de C. elegans:}\PY{l+s+s2}{\PYZdq{}}\PY{p}{,} \PY{n}{min\PYZus{}path\PYZus{}max\PYZus{}componente}\PY{p}{)}
\end{Verbatim}
\end{tcolorbox}

    \begin{Verbatim}[commandchars=\\\{\}]
Número de componentes conexas del grafo de C. elegans: 89

Camino mínimo de la componente más grande de C. elegans: 7.922564808498197
\end{Verbatim}

    \begin{tcolorbox}[breakable, size=fbox, boxrule=1pt, pad at break*=1mm,colback=cellbackground, colframe=cellborder]
\prompt{In}{incolor}{37}{\hspace{4pt}}
\begin{Verbatim}[commandchars=\\\{\}]
\PY{n+nb}{print}\PY{p}{(}\PY{l+s+s2}{\PYZdq{}}\PY{l+s+s2}{Promedio del número de componentes conexas del grafo aleatorio: }\PY{l+s+si}{\PYZpc{}s}\PY{l+s+s2}{, con }\PY{l+s+si}{\PYZpc{}s}\PY{l+s+s2}{ desviación estándar}\PY{l+s+s2}{\PYZdq{}}
      \PY{o}{\PYZpc{}}\PY{p}{(}\PY{n}{dic\PYZus{}params}\PY{p}{[}\PY{l+s+s2}{\PYZdq{}}\PY{l+s+s2}{Componentes}\PY{l+s+s2}{\PYZdq{}}\PY{p}{]}\PY{p}{[}\PY{l+m+mi}{0}\PY{p}{]}\PY{p}{,} \PY{n}{dic\PYZus{}params}\PY{p}{[}\PY{l+s+s2}{\PYZdq{}}\PY{l+s+s2}{Componentes}\PY{l+s+s2}{\PYZdq{}}\PY{p}{]}\PY{p}{[}\PY{l+m+mi}{1}\PY{p}{]}\PY{p}{)}\PY{p}{)}
\PY{n+nb}{print}\PY{p}{(}\PY{p}{)}
\PY{n+nb}{print}\PY{p}{(}\PY{l+s+s2}{\PYZdq{}}\PY{l+s+s2}{Promedio del camino mínimo medio de grafo aleatorio: }\PY{l+s+si}{\PYZpc{}s}\PY{l+s+s2}{, con }\PY{l+s+si}{\PYZpc{}s}\PY{l+s+s2}{ desviación estándar}\PY{l+s+s2}{\PYZdq{}}
      \PY{o}{\PYZpc{}}\PY{p}{(}\PY{n}{dic\PYZus{}params}\PY{p}{[}\PY{l+s+s2}{\PYZdq{}}\PY{l+s+s2}{Shortest path}\PY{l+s+s2}{\PYZdq{}}\PY{p}{]}\PY{p}{[}\PY{l+m+mi}{0}\PY{p}{]}\PY{p}{,} \PY{n}{dic\PYZus{}params}\PY{p}{[}\PY{l+s+s2}{\PYZdq{}}\PY{l+s+s2}{Shortest path}\PY{l+s+s2}{\PYZdq{}}\PY{p}{]}\PY{p}{[}\PY{l+m+mi}{1}\PY{p}{]}\PY{p}{)}\PY{p}{)}
\end{Verbatim}
\end{tcolorbox}

    \begin{Verbatim}[commandchars=\\\{\}]
Promedio del número de componentes conexas del grafo aleatorio: 146.55, con
7.651633812461231 desviación estándar

Promedio del camino mínimo medio de grafo aleatorio: 7.9675033029459525, con
0.10250774207511149 desviación estándar
\end{Verbatim}

    \(\\\)

\hypertarget{e-el-muxe1ximo-k-para-el-cual-existe-un-k-core}{%
\subsubsection{e) El máximo k para el cual existe un k
core}\label{e-el-muxe1ximo-k-para-el-cual-existe-un-k-core}}

Un núcleo de grado k es el máximo subgrafo en el cual todos los puntos
son adyacentes a al menos otros k puntos, y fue propuesto por Seidman
(1983) \href{https://webs.ucm.es/info/pecar/Glosa.htm}{{[}4{]}}. Este
parámetro es un método muy bueno para comprobar si nuestro grado de
estudio se comporta igual que los grafos aleatorios.

La medida de \emph{k-core} apoya el resultado obtenido en el cálculo del
índice de clusterización de ambos grafos. Aunque el índice de
clusterización de la red de proteínas de \emph{C. elegans} no sea muy
alta (\textless{} 0.8), sí es mayor que la del grafo aleatorio (0.075708
\textgreater{} 0.0005827). Por ello, que el máximo k-core encontrado en
la red de proteínas de \emph{C. elegans} (k-core = 6) sea superior a la
del grafo aleatorio ( k-core = 2 con desv std = 0) refleja que las ramas
se distribuyen en la red de proteínas de \emph{C. elegans} de forma
sesgada y no aleatoria, pudiendo concluirse de nuevo que la red de
\emph{C. elegans} no es una red aleatoria.

    \begin{tcolorbox}[breakable, size=fbox, boxrule=1pt, pad at break*=1mm,colback=cellbackground, colframe=cellborder]
\prompt{In}{incolor}{38}{\hspace{4pt}}
\begin{Verbatim}[commandchars=\\\{\}]
\PY{n}{k\PYZus{}core} \PY{o}{=} \PY{n}{nx}\PY{o}{.}\PY{n}{core\PYZus{}number}\PY{p}{(}\PY{n}{G\PYZus{}CE}\PY{p}{)}

\PY{c+c1}{\PYZsh{} Nodo con mayor valor de centralidad de grado}
\PY{n}{mayor} \PY{o}{=} \PY{n+nb}{max}\PY{p}{(}\PY{n}{k\PYZus{}core}\PY{o}{.}\PY{n}{values}\PY{p}{(}\PY{p}{)}\PY{p}{)}
\PY{n+nb}{print}\PY{p}{(}\PY{l+s+s2}{\PYZdq{}}\PY{l+s+s2}{Máximo k\PYZhy{}core de la red de proteínas de C. elegans: }\PY{l+s+si}{\PYZpc{}d}\PY{l+s+s2}{\PYZdq{}} \PY{o}{\PYZpc{}} \PY{n}{mayor}\PY{p}{)}
\end{Verbatim}
\end{tcolorbox}

    \begin{Verbatim}[commandchars=\\\{\}]
Máximo k-core de la red de proteínas de C. elegans: 6
\end{Verbatim}

    \begin{tcolorbox}[breakable, size=fbox, boxrule=1pt, pad at break*=1mm,colback=cellbackground, colframe=cellborder]
\prompt{In}{incolor}{39}{\hspace{4pt}}
\begin{Verbatim}[commandchars=\\\{\}]
\PY{n+nb}{print}\PY{p}{(}\PY{l+s+s2}{\PYZdq{}}\PY{l+s+s2}{Promedio del k\PYZhy{}core medio en el grafo aleatorio: }\PY{l+s+si}{\PYZpc{}s}\PY{l+s+s2}{, con }\PY{l+s+si}{\PYZpc{}s}\PY{l+s+s2}{ desviación estándar}\PY{l+s+s2}{\PYZdq{}}
      \PY{o}{\PYZpc{}}\PY{p}{(}\PY{n}{dic\PYZus{}params}\PY{p}{[}\PY{l+s+s2}{\PYZdq{}}\PY{l+s+s2}{Max k\PYZhy{}core}\PY{l+s+s2}{\PYZdq{}}\PY{p}{]}\PY{p}{[}\PY{l+m+mi}{0}\PY{p}{]}\PY{p}{,} \PY{n}{dic\PYZus{}params}\PY{p}{[}\PY{l+s+s2}{\PYZdq{}}\PY{l+s+s2}{Max k\PYZhy{}core}\PY{l+s+s2}{\PYZdq{}}\PY{p}{]}\PY{p}{[}\PY{l+m+mi}{1}\PY{p}{]}\PY{p}{)}\PY{p}{)}
\end{Verbatim}
\end{tcolorbox}

    \begin{Verbatim}[commandchars=\\\{\}]
Promedio del k-core medio en el grafo aleatorio: 2.0, con 0.0 desviación
estándar
\end{Verbatim}

    \(\\\)

\hypertarget{comparaciuxf3n-con-grafos-regulares}{%
\subsection{Comparación con grafos
regulares}\label{comparaciuxf3n-con-grafos-regulares}}

    Como hemos visto con los resultados anteriores, el grafo de
CaernoElegans presenta un índice de clusterización superior un camino
característico muy similar al del aleatorio. Estas características nos
indican que se trata de una red de mundo pequeño. Las redes de mundo
pequeño se encuentran a caballo entre redes aleatorias y redes
regulares, por lo que resultaría interesante estudiar también las
características de una red regular.

Para que el grafo regular sea lo más parecido a nuestro grafo de estudio
de \emph{C. elegans} se emplean los siguientes parámetros:

\begin{itemize}
\item
  \(p = 0\) para que la probabilidad de formar atajos sea 0 y se genere
  un grafo completamente regular con el que poder comparar nuestro grafo
  de \emph{C. elegans}
\item
  \(N = 1387\) para que tenga el mismo número de nodos
\end{itemize}

Se generará el grafo regular siguiendo la siguiente condición:

\[ N >> k >> log(N) \rightarrow 1387 >> k >> 3.14\]

La condición \(N >> k\) nos asegura que el grafo sea disperso. La
condición \(k >> log(N)\) nos asegura que el grafo sea conexo. Como el
número de vecinos en un grafo regular tiene que ser un número entero,
seleccionaremos el techo de 3.14, es decir, \(k = 4\).

    \begin{tcolorbox}[breakable, size=fbox, boxrule=1pt, pad at break*=1mm,colback=cellbackground, colframe=cellborder]
\prompt{In}{incolor}{40}{\hspace{4pt}}
\begin{Verbatim}[commandchars=\\\{\}]
\PY{c+c1}{\PYZsh{} Emplearemos p = 0 para que el grafo sea regular y  no haya atajos}
\PY{n}{G\PYZus{}RE} \PY{o}{=} \PY{n}{nx}\PY{o}{.}\PY{n}{watts\PYZus{}strogatz\PYZus{}graph}\PY{p}{(}\PY{l+m+mi}{1387}\PY{p}{,}\PY{l+m+mi}{4}\PY{p}{,}\PY{l+m+mi}{0}\PY{p}{)}
\PY{n+nb}{print}\PY{p}{(}\PY{n}{nx}\PY{o}{.}\PY{n}{info}\PY{p}{(}\PY{n}{G\PYZus{}RE}\PY{p}{)}\PY{p}{)}
\PY{n+nb}{print}\PY{p}{(}\PY{l+s+s2}{\PYZdq{}}\PY{l+s+s2}{C:}\PY{l+s+s2}{\PYZdq{}}\PY{p}{,} \PY{n}{nx}\PY{o}{.}\PY{n}{average\PYZus{}clustering}\PY{p}{(}\PY{n}{G\PYZus{}RE}\PY{p}{)}\PY{p}{)}
\PY{n+nb}{print}\PY{p}{(}\PY{l+s+s2}{\PYZdq{}}\PY{l+s+s2}{L:}\PY{l+s+s2}{\PYZdq{}}\PY{p}{,} \PY{n}{nx}\PY{o}{.}\PY{n}{average\PYZus{}shortest\PYZus{}path\PYZus{}length}\PY{p}{(}\PY{n}{G\PYZus{}RE}\PY{p}{)}\PY{p}{)}
\end{Verbatim}
\end{tcolorbox}

    \begin{Verbatim}[commandchars=\\\{\}]
Name:
Type: Graph
Number of nodes: 1387
Number of edges: 2774
Average degree:   4.0000
C: 0.5
L: 173.75036075036076
\end{Verbatim}

    \(\\\) \#\# Conclusiones

    \begin{longtable}[]{@{}lll@{}}
\toprule
& CaernoElegans & Aleatorio\tabularnewline
\midrule
\endhead
\(L\) & 7.922564 & 7.967503 ± 0.102507\tabularnewline
\(C\) & 0.075708 & 0.000939 ± 0.0011093\tabularnewline
Degree\_Centrality & 0.001714 & 0.001714 ± 1.362e-18\tabularnewline
Closeness\_Centrality & 0.071142 & 0.097581 ± 0.00165\tabularnewline
Betweness\_Centrality & 0.002567 & 0.003852 ± 0.000130\tabularnewline
Max \(k\)-core & 6 & 2 ± 0.0\tabularnewline
Dispersión/Densidad & 0.00171 & ---\tabularnewline
\bottomrule
\end{longtable}

    Con las métricas obtenidas en el apartado 2 del estudio del grafo de
CaernoElegans llegamos a la conclusión que la red de proteínas del
organismo \emph{Caenorhabditis elegans} constituyen un grafo disperso,
no dirigido, no conexo compuesto por 1387 nodos y 1648 ramas y con un
diámetro (22) superior a los grafos aleatorios (18.5 \(\pm\) 0.5).

En lo que se refiere al apartado 3, a pesar de presentar el mismo grado
medio, el estudio y comparación de la distribución de grado de los nodos
del grafo CaernoElegans y los grafos aleatorios nos permite llegar a la
conclusión de que la red de proteínas de \emph{Caenorhabditis elegans}
no es un grafo aleatorio y sí un grafo libre de escala.

Igualmente, en la tabla superior se han recopilado los resultados de los
cálculos de los distintos parámetros realizados en el apartado 4. Estos
resultados apoyan el hecho de que el grafo CaernoElegans no es un grafo
aleatorio y nos permiten identificar nodos importantes y susceptibles de
ser atacados en la red de CaernoElegans. Por ejemplo, el nodo con mayor
grado y mayor centralidad (proteína T08G11.5) es una buena diana contra
la que dirigir un ataque a la red, ya que se trata de un nodo importante
que se encuentra relacionada con un gran número de proteínas. Del mismo
modo, las proteínas W10C8.2 y C23G10.4 presentan el mayor valor de
cercanía y ``betweenness'', respectivamente. En consecuencia, también
son nodos claves en la integración de la información en la red de
proteínas y contra las que se podría dirigir un ataque. La comparación
de otros cálculos, como el número de componentes conexas que conforman
al grafo CaernoElegans (89) frente al grafo aleatorio (147.0 \(\pm\)
5.0), y la presencia de un 6-core en la red de proteínas ausente en los
grafos aleatorios (2-kore \(\pm\) 0.0) corroboran que la red de
proteínas presenta un sesgo que la diferencia de las redes aleatorias.

Finalmente, el estudio del índice de clusterización (\(C\)) junto con el
camino característico (\(L\)) calculado para la componente conexa más
grande del grafo CaernoElegans, nos permite establecer que se trata de
un grafo de mundo pequeño. Esta conclusión se obtiene observando los
valores recogidos en la tabla superior, donde se puede observar que el
índice de clusterización del grafo CaernoElegans es superior al grafo
aleatorio (0.075708 \textgreater{} 0.000939 ± 0.0011093), pero ambas
comparten el mismo camino característico (7.922564 \(\simeq\) 7.967503 ±
0.102507). Este rasgo es característicos de las redes de mundo pequeño
que comparten rasgos de grafos aleatorios y grafos regulares.


    % Add a bibliography block to the postdoc
    
    
    
    \end{document}
